\subsection{Probleme bei der Initialisierung von Daten}

Die Initialisierung der Daten auf der Plattform erfolgt mithilfe
eines Java Skripts, das die Daten in den JBoss-Server injiziert.
Dies bringt jedoch einige Probleme mit sich, die den Testzyklus
verlangsamen. Beim Testen einer Anwendung muss der Tester sehr
oft zu einer Ausgangssituation zurückkehren, um seine Tests erneut
auszuführen, was mit dem Initializer nicht möglich ist. Es ist daher
technisch nicht möglich, denselben Benutzer mehrmals für bestimmte
Tests zu verwenden, da die Daten nicht mehr neu sind und beschädigt
wurden. Aus diesem Grund muss der Tester nach jedem Test eine Datei
testData.csv (\Cref{fig:testData}) erstellen und diese Benutzer
verwenden. Dies verkompliziert den Testprozess. Die Tatsache, dass
man nicht zur Ausgangssituation zurückkehren kann (die Daten nicht
zurücksetzen kann), verhindert, dass man die volle Kontrolle über
das Schreiben der Tests hat. Das Testen von bestimmten Fällen ist
noch schwieriger zu reproduzieren. Die fehlende Kontrolle über das
Erzeugen und Löschen von Daten nimmt dem Tester die Möglichkeit,
die Anwendung vollständig zu testen.

Es ist wichtig zu beachten, dass der Initializer, der für die
Testinfrastruktur verwendet wird, noch in der ersten Version
vorliegt und noch wenig entwickelt ist. Es könnte vermieden werden,
dass bei jedem Start neue Benutzer generiert werden, wenn es die
Möglichkeit gäbe, die Testdaten zurückzusetzen. Eine Lösung könnte
auch darin bestehen, eine lokale Testdatenbank zu erstellen, auf die
von einem Docker-Container aus zugegriffen werden kann.
