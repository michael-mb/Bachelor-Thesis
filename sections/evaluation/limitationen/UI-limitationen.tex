\subsection{Probleme beim UI-Testing}


Die Entwicklung von UI-Tests erforderte, UI-Tests für \gls{jexam_new}
bereitzustellen und gleichzeitig \gls{jexam_2009} zu testen. Die beiden
Anwendungen haben eine fast identische grafische Benutzeroberfläche,
aber es gibt deutliche Unterschiede im HTML-Code (siehe Abbildung).
Selenium stützt sich beim Auffinden von Elementen hauptsächlich auf
diesen HTML-Code. Dies könnte für die Entwicklung von Tests
problematisch werden. Tests, die für \gls{jexam_2009} funktionieren,
funktionieren  nicht automatisch  für \gls{jexam_new}. Sie müssen daher
angepasst werden.


Dieses Problem kann leicht gelöst werden, wenn die Entwickler
einen Selenium-Entwicklungsansatz verwenden. Dieser Ansatz
besteht darin, dass man entwickelt, während man weiß, dass man
Selenium-Tests zum Testen schreiben wird. Entwickler, die diesen
Ansatz verwenden, fügen ihren anklickbaren Elementen (Hyperlinks,
Buttons, Formularelementen) HTML-Attribute (id, name, value) hinzu,
die einen einfachen Zugriff auf ein HTML-WebElement ermöglichen.
Dies erleichtert die Entwicklung von Tests mit Selenium erheblich.
Die Entwickler sollten sich auch bemühen, die gleiche
Frontend-Struktur in beiden Anwendungen beizubehalten. Auf diese
Weise müssen die Tester einen Test nur einmal schreiben und er
kann auf beiden Plattformen funktionieren.

Unter den Erwartungen an UI-Tests gibt es einen bestimmten Test,
der mit Selenium nicht durchführbar ist. Es handelt sich dabei um
den Abruf der Noten als PDF. Selenium ist begrenzt und hat keine
Möglichkeit, diese Aufgabe zu realisieren.

