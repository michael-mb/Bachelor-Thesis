\subsection{Leistungsprobleme}

Die Testinfrastruktur ist nicht frei von Fehlern. Sie stammen nicht
direkt aus der Infrastruktur, sondern aus den verschiedenen
Anwendungen, aus denen sie besteht (\Gls{jexam_2009}, \Gls{jexam_new} und vor
allem JBOSS). Bei der Durchführung von Tests kann es zu Inkonsistenzen
und Fehlern auf der Ebene des jBoss-Servers kommen. Manchmal werden
die Daten nicht korrekt angezeigt. Dies führt dazu, dass der
JBOSS-Server in einem Docker-Container nicht stabil ist. Um dieses
Problem zu beheben, wäre die Lösung, jExam für eine Testversion von
JBOSS zu trennen. Dies würde die Testinfrastruktur weniger komplex
und stabiler machen.

Das zweite Problem ist, dass die Testinfrastruktur schwerfällig ist
und sehr viele Ressourcen auf dem Host-Computer beansprucht. Um dieses
Problem zu lösen, sollte die Infrastruktur durch eine bessere Nutzung
von Docker-Compose oder sogar durch eine mögliche Integration mit dem
Kubernetes-Tool optimiert werden.

