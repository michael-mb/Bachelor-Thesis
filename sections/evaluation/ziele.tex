\section{Erreichte Ziele}

In einem ersten Schritt sollten die Funktionen von \gls{jexam_2009}
getestet werden. Danach sollte herausgefunden werden, ob es möglich 
ist, die Tests von \gls{jexam_2009} zum Testen von \gls{jexam_new} 
wiederzuverwenden und sogar eine Testsuite für \gls{jexam_new} 
bereitzustellen, bevor die Entwicklung abgeschlossen ist. Dies wurde
durch die Verwendung des Selenium-Tools ermöglicht, da 
\gls{jexam_2009} und \gls{jexam_new} praktisch die gleiche grafische 
Benutzeroberfläche haben. Die in \autoref{subsec:jexam-ui-tests} beschriebenen
Funktionen wurden für \gls{jexam_2009} getestet und nur die Login-Funktion
für \gls{jexam_new} (weil die anderen Funktionen noch nicht entwickelt
wurden). Der gleiche Login-Test funktioniert also für beide
Plattformen. Das Ziel für UI-Tests wurde erreicht.

Zweitens wurde die Frage gestellt, inwieweit es möglich ist,
Sicherheitstests zu integrieren. Diese Frage wurde in
\autoref{sec:entwicklung-von-sicherheitstests} behandelt und es
zeigte sich, dass es tatsächlich möglich ist,
Sicherheitstests zu integrieren. Zu diesem Zweck wurden
Schwachstellen-Scanner eingebaut, die beide Versionen von jExam
auf ausnutzbare Sicherheitslücken scannen.


Drittens ging es um die Frage, ob Bottleneck-Performance in beiden
Versionen von jExam festgestellt werden konnte. Dies war ein schwer
zu lösendes Problem, da sich \gls{jexam_new} noch in der Entwicklung
befindet. Das jMeter-Tool wurde jedoch in die Testinfrastruktur
integriert und mit Skripten versehen, um beide Versionen zu testen.
Bei der Analyse der von diesen Skripten erstellten Berichte wurden
kürzere Antwortzeiten für \gls{jexam_new} festgestellt, was auf eine
mögliche Verbesserung hindeuten könnte. Die verwendeten jMeter-Skripte
sind jedoch recht einfach und eignen sich nicht zum genauen Testen
von Anwendungen. Sie sollten daher verbessert werden.


Die Testinfrastruktur wurde so konzipiert, dass sie auf allen
Systemen tragbar, leicht installierbar, vollautomatisch und für
die Durchführung von Tests konfigurierbar ist. Daher kann man
sagen, dass die Testinfrastruktur die in
\autoref{sec:anforderungen-an-die-jexam-testsuite} beschriebenen
Anforderungen erfüllt.




