\section{Zusammenfassung}

Das Ziel dieser Arbeit ist die Entwicklung einer WebTestsuite für die jExam-Plattform.
Dies wurde in die Entwicklung einer Testinfrastruktur übersetzt. Diese Testinfrastruktur
sollte nicht nur nicht-funktionale Tests wie Performance- und Sicherheitstests, sondern
auch funktionale Tests beinhalten. Darüber hinaus sollte die Infrastruktur vollautomatisch,
leicht wartbar und einfach zu installieren sein. Da sich die neue Version von jExam noch in
der Entwicklung befindet, sollten die Tests der Infrastruktur bereits für die neue Version
vorbereitet sein, während sie noch auf der alten Version laufen.

Um dieses Ziel zu erreichen, mussten wir zunächst die Grundlagen der Arbeit schaffen,
indem wir die wichtigsten Begriffe und Konzepte definierten, die zum Verständnis der
Arbeit beitragen können. Danach folgte eine Analysephase, in der das zu lösende Problem
analysiert und die Ziele definiert wurden, die bei der Entwicklung erreicht werden sollten.
Nach dieser Phase folgte die Implementierungsphase, in der die Testinfrastruktur im Detail
beschrieben und die verschiedenen Werkzeuge, die zu ihrer Erstellung verwendet wurden,
vorgestellt wurden. Schließlich folgt die Auswertungsphase (Evaluation), in der konkret
analysiert wird, ob die Ziele der Infrastruktur erreicht wurden.

Mit Hilfe von Docker, Maven und anderen Werkzeugen wurde eine Infrastruktur aufgebaut,
die beide Versionen von jExam mit dem ZAP-Proxy-Tool auf Sicherheitslücken untersuchen
kann. Diese Infrastruktur ermöglicht auch, die Leistung der beiden Versionen auf
verschiedenen Ebenen mit dem integrierten JMeter-Tool zu testen. Funktionale Tests
wurden mit Selenium, einem der beliebtesten und von Entwicklern am häufigsten
verwendeten Testwerkzeuge, durchgeführt. Die Infrastruktur ermöglicht die automatische
Ausführung von Tests, sobald sie geschrieben und in die Plattform integriert wurden.
Zu den Herausforderungen dieser Arbeit gehört die Erstellung einer Dokumentation.
Diese Aufgabe wurde ebenfalls erfüllt. Die Dokumentation ist im Anhang dieser
Arbeit zu finden. Die ursprünglichen Ziele, die in der Einleitung zu dieser Arbeit
definiert wurden, wurden erreicht.
