\section{Ausblick}

Die Testinfrastruktur in ihrem derzeitigen Zustand dient lediglich als
Basis für die Entwicklung von jExam-Tests. Die Sicherheits- und
Performancetests, die integriert wurden, sind ziemlich minimalistisch
und sollten weiter verbessert werden. Dasselbe gilt für die UI-Tests,
die derzeit nur einige Funktionen der Plattform testen. Für eine bessere
Abdeckung ist es jedoch wichtig, mehr Tests zu schreiben und alle Funktionen
abzudecken. Eine so sensible Plattform wie jExam sollte in allen Details
getestet werden.

Es gibt jedoch noch einige Probleme, die unabhängig von der Testinfrastruktur
an sich sind und die  gelöst werden sollten, um die Testinfrastruktur erheblich
zu verbessern. Ein Beispiel hierfür ist die Abhängigkeit der Testinfrastruktur
vom JBoss-Server, die häufig zu Dateninkonsistenzproblemen führt, da dieser
Server Verbindungen zu entfernten Datenbanken herstellt. Dieses Problem und
mögliche Lösungen wurden in \autoref{ch:evaluation} behandelt.

Die jExam-Testinfrastruktur ist relativ jung und könnte noch erheblich verbessert
werden. Nicht nur die bestehenden Tests können weiter verbessert werden, sondern
es besteht auch die Möglichkeit, neue Testarten zu integrieren. Dazu gehören
Stresstests, Volumetests, Usabilitytests und andere Arten von funktionalen
Tests, die sich von UI-Tests unterscheiden. Die Möglichkeiten sind sehr breit
gefächert und Docker könnte an einem bestimmten Punkt auch eingeschränkt werden.
Dies würde eine Tür für den Einsatz von Tools wie Kubernetes öffnen, das ein
Open-Source-System zur Automatisierung der Bereitstellung, Skalierung und
Verwaltung von containerisierten Anwendungen ist (vgl. \cite{kubernetes}).