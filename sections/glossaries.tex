\newglossaryentry{TestSuite}
{name=Test-Suite , description={Die Zusammenstellung (Aggregation) mehrerer Testf\"alle f\"ur den Test einer Komponente oder eines Systems, bei der Nachbedingungen des einen Tests als Vorbedingungen des folgenden Tests genutzt werden k\"onnen.}}

\newglossaryentry{jexam_2009}
{name=jExam 2009 ,description={Alte Version von jExam}}

\newglossaryentry{jexam_new}
{name=jExam New ,description={Neue Version von jExam, die im Moment in Entwicklung steht}}

\newglossaryentry{frontend}
{name=Frontend ,description={Benutzeroberfl\"ache, die z. B. in Form einer grafischen Benutzeroberfl\"ache (englisch graphical user interface, kurz GUI) oder mittels Bildschirmmasken implementiert sein kann.}}

\newglossaryentry{backend}
{name=Backend ,description={bezeichnet in der Regel den funktionalen Teil eines digitalen Produkts wie beispielsweise Webseiten oder Apps und bezieht sich meist auf Client Server.}}

\newglossaryentry{prod}
{name=Produktion , description={Produktionsumgebung ist ein Begriff, der vor allem von Entwicklern verwendet wird, um die Umgebung zu beschreiben, in der Software und andere Produkte tats\"achlich f\"ur ihre beabsichtigte Verwendung durch Endbenutzer in Betrieb genommen werden.}}

\newglossaryentry{bug}
{name=Bug ,description={Programmfehler oder Softwarefehler oder Software-Anomalie}}

\newglossaryentry{debug}
{name=Debugging ,description={Debugging ist das Auffinden und Beheben von Fehlern in einem Programm, die den korrekten Betrieb einer Software verhindern.}}

\newglossaryentry{bottleneck}
{name=Bottleneck, description={In der Softwareentwicklung spricht man von einem Bottleneck (Engpass), wenn die Kapazit\"at einer Anwendung oder eines Computersystems durch eine einzelne Komponente begrenzt wird, wie der Hals einer Flasche, der den gesamten Wasserfluss verlangsamt. Der Bottleneck (Engpass) hat den geringsten Durchsatz aller Teile des Transaktionspfads.}}

\newglossaryentry{cve}
{name=CVE, description={Common Vulnerabilities and Exposures auf Englisch, bezeichnet eine \"offentliche Liste von IT-Sicherheitsl\"ucken. Wenn man von einer CVE spricht, bezieht man sich in der Regel auf eine Sicherheitsl\"ucke, der eine CVE-Kennung zugewiesen wurde.}}

\newglossaryentry{crawlen}
{name=crawlen, description={Prozess der automatischen Entdeckung neuer Ressourcen in einer Anwendung}}

\newglossaryentry{selenese}
{name=Selenese, description={Programmiersprache, die zum Schreiben von Selenium-Befehlen verwendet wird. Diese Selenium-Befehle werden dann zum Testen von Webanwendungen verwendet. Basierend auf den HTML-Tags der UI-Elemente kann man deren Existenz \"uberpr\"ufen. Befehle helfen Selenium zu verstehen, welche Aktionen oder Operationen ausgef\"uhrt werden sollen.}}
