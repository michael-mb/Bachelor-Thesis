\subsubsection{Factory}

Page Factory ist eine Klasse, die von Selenium WebDriver bereitgestellt
wird, um das Page Object Model zu implementieren. Das Page Object
Repository wird mit Hilfe des Page Factory-Konzepts von den Testmethoden
getrennt. Page Factory bietet Annotationen, um Elemente zu initialisieren und sie
anschaulich und lesbar macht. Zu den Vorteilen seiner Verwendung gehören:

\begin{enumerate}
    \item \textbf{Sauberer Code}: Das definierte Webelement wird von den Methoden
    getrennt, um eine Webseite in einer Page Object sauber und
    aufgeräumt zu gestalten.
    \item \textbf{Lesbar und beschreibend}: Ein Webelement wird als Variable
    (bekannt als Object Field) deklariert, und die Field-Annotation
    (@FindBy siehe \Cref{fig:page-exp}) wird verwendet, um den Namen, den Typ und
    die Position des Elements zu beschreiben. Auf diese Weise können
    definierte Webelement anhand seiner Annotationen wie Name, Typ
    usw. leicht identifiziert werden.
    \item \textbf{Einfache Wartbarkeit}: Das definierte Webelement kann ohne
    Neudefinition überall in der Page Object Klasse und den Unterklassen
    verwendet werden. Das bedeutet, dass ein bestimmtes Webelement
    mehrmals verwendet werden kann, aber nur an einer Stelle
    definiert ist.
\end{enumerate}


Im nächsten Teil werden die Implementierung der Tests und die verwendeten
Methoden genauer beschrieben.



