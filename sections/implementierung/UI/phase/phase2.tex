\subsubsection{Phase 2: Erstellung eines Testszenarios}

In dieser Phase muss der Tester ein Ausführungsszenario für seinen 
Test erstellen. Es geht darum, die zu testende Funktion manuell
auf der Plattform auszuführen und die Ausführungsschritte zu 
dokumentieren, die in den nächsten Schritten automatisch wiederholt
werden. Dieser Prozess könnte auch in schriftlicher Form erfolgen
(normalerweise bei langen Szenarien). Diese Phase dient dazu, den 
Test im Detail zu planen und so die spätere Umsetzung zu erleichtern.
Im Falle des Registrierungstests ist das Szenario wie folgt festgelegt:

\begin{enumerate}
    \item Gehen Sie auf die Login-Seite und füllen Sie das Formular mit
    der nicht registrierten Matrikelnummer aus.
    \item Klicken Sie auf den Login-Button und Sie werden zur
    Registrierungsseite weitergeleitet.
    \item Füllen Sie das Registrierungsformular aus und klicken Sie
    auf Registrieren.
    \item Weiterleitung zur Seite Privacy Policy und Zustimmung zu
    den allgemeinen Nutzungsbedingungen
    (prüfen Sie, ob der TestUser automatisch eingeloggt wird).
    \item Logout und Login erneut mit den Zugangsdaten und der
    Matrikelnummer, die Sie bei der Registrierung verwendet haben.
    \item Wenn alles richtig funktioniert: Die Registrierung war erfolgreich.
\end{enumerate}