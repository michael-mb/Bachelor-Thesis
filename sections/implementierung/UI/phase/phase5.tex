\subsubsection{Phase 5: Implementierung der tests nach dem szenario}

Diese Phase konzentriert sich auf die Umsetzung des in Phase 2
beschriebenen Szenarios. Es handelt sich um das Schreiben des Tests
an sich. In dieser Phase wird der Tester die Aktionen eines Nutzers
nacheinander ausführen.

Im Fall des Registrierungstests geht es zunächst darum, sich mit einer
Matrikelnummer zu registrieren, die nicht in der Datenbank verzeichnet
ist. Dann wird das Registrierungsformular mit den zuvor festgelegten
Daten ausgefüllt. Im nächsten Schritt werden die Allgemeinen
Geschäftsbedingungen akzeptiert und der Testuser dann einloggt.
Um zu überprüfen, ob die Registrierung erfolgreich war, muss der
Tester einen Login-Test mit den Daten simulieren, die er bei der
Registrierung verwendet hat (siehe Quellcode \ref{lst:reg-test}). Zwischen all
diesen Schritten werden Verifikationsprozesse durchgeführt
(z. B. ob eine Seite korrekt initialisiert wurde).

Es ist wichtig, Logs und ExtentLogger zu verwenden, um jeden Schritt
der Tests zu dokumentieren. Dadurch kann bei der Erstellung des
Ausführungsberichts nachvollzogen werden, welche Schritte während
des Tests durchgeführt wurden. Außerdem können diese Logs zeigen,
ob die Tests abgeschlossen wurden oder ob während der Ausführung
ein Fehler aufgetreten ist.


\begin{lstlisting}[label={lst:reg-test}, caption={Registration Test Quellcode}]

    // Registration Test Quellcode

    @Test
    @Description("New User Registration")
    public void doRegistrationTest(){
        ExtentReport.createTest("Test User Registration");
        ExtentLogger.info("⛔⛔ If you dont see \"Test Passed\" at the end, the Test didn't passed !");

        TestUser user = TestUser.REGISTRATION_USER;
        RegistrationPage registrationPage = startRegistration(user.username, user.password);
        Assert.assertTrue(registrationPage.isInitialized(),"Registrationpage was not correctly initialized");
        ExtentLogger.info("Registration Page is correctly initialized !");

        RegistrationForm form = new RegistrationForm(user.username+"@", user.username+"@", "f",
                "DE", user.username, "30010", "Bachelor Informatik", "1533186");
        ContractPage contractPage = registrationPage.registerUser(form);
        Assert.assertTrue(contractPage.isInitialized(), "Contractpage should be initialized !");
        ExtentLogger.info("Contract Page is initialized");

        MainPage mainPage = contractPage.acceptPrivacyPolicy();
        ExtentLogger.info("Privacy Policy was accepted");

        ExtentLogger.info("Try logout and login !");
        LoginPage loginPage = mainPage.selectLogoutLink();
        Assert.assertTrue(loginPage.isInitialized() , "Expected to land on login page, after log in, but landed on another page");
        ExtentLogger.info("Logout was successful, got on login page");
        log.info("Logout was successful, got on login page");

        // New Login with New Identifier
        mainPage = regularLogin(user.username , user.password);
        Assert.assertTrue(mainPage.isInitialized() , "Expected to land on main page, after log in, but landed on another page");
        ExtentLogger.info("Login was successful, got on main page");
        log.info("Login was successful, got on main page");

        loginPage = mainPage.selectLogoutLink();
        Assert.assertTrue(loginPage.isInitialized() , "Expected to land on login page, after log in, but landed on another page");
        ExtentLogger.info("Logout was successful, got on login page");
        log.info("Logout was successful, got on login page");

        ExtentLogger.info("Registration was successful");
        log.info("Registration was successful");
        ExtentLogger.pass("Test passed");
    }
\end{lstlisting}

Es gibt noch viele weitere Funktionen und Klassen, die in dieser
Arbeit nicht beschrieben werden können. Die Details des Schreibens
und aller Funktionen werden jedoch in der Dokumentation der
Testinfrastruktur ausführlich erläutert.