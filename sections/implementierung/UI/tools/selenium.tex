\subsubsection{Selenium Webdriver}

Selenium ist ein Open-Source-projekt für eine Reihe von Tools und Bibliotheken
zur Unterstützung der Webbrowser-Automatisierung (vgl. \cite{selenium-survey}).
Selenium bietet  Werkzeuge zur Erstellung funktionaler Tests, ohne dass eine
Testskriptsprache erlernt werden muss. Außerdem bietet es eine testspezifische
Sprache (Selenese), mit der Tests in einer Reihe beliebter Programmiersprachen
geschrieben werden können, darunter JavaScript (Node.js), C#, Groovy, Java,
Perl, PHP, Python, Ruby und Scala. Die Tests können dann mit den meisten
modernen Webbrowsern ausgeführt werden. Selenium läuft auf Windows, Linux
und macOS. Selenium besteht aus mehreren Komponenten, von denen jede eine
bestimmte Rolle bei der Entwicklung der Testautomatisierung von Webanwendungen
übernimmt. Zu diesen Komponenten gehören: die Selenium IDE, Selenium Client API,
Selenium Remote Control, Selenium Grid und schließlich der Selenium WebDriver,
der in dieser Arbeit verwendet wird.

Das Kernelement von Selenium ist Selenium WebDriver, eine Schnittstelle zum
Schreiben von Anweisungen, die in verschiedenen Browsern austauschbar sind.
Selenium WebDriver nimmt Befehle entgegen (die in \gls{selenese} oder über eine
Client-API gesendet werden) und sendet sie an einen Browser. Dies wird durch
einen browserspezifischen Browsertreiber implementiert, der Befehle an einen
Browser sendet und Ergebnisse abruft. Die meisten Browsertreiber starten
tatsächlich eine Browseranwendung (z. B. Firefox, Google Chrome, Internet
Explorer, Safari oder Microsoft Edge) und greifen darauf zu. Selenium WebDriver
benötigt keinen speziellen Server, um Tests auszuführen. Stattdessen startet
der WebDriver direkt eine Browserinstanz und steuert sie.  Wo immer möglich,
verwendet WebDriver native Funktionen auf Betriebssystemebene und nicht
browserbasierte JavaScript-Befehle zur Steuerung des Browsers. Dadurch werden
Probleme mit subtilen Unterschieden zwischen nativen und JavaScript-Befehlen,
einschließlich Sicherheitseinschränkungen, vermieden (vgl. \cite{Stewart2016}).
Selenium WebDriver ist vollständig implementiert und wird in JavaScript
(Node.js), Python, Ruby, Java, Kotlin und C# unterstützt.


Selenium ist derzeit das von Testern am meisten geliebte Framework für
UI-Testing (vgl. \cite{selenium-survey} , S. 08-09). Es hat viele Vorteile,
wie z.B. die \"Ubertragbarkeit auf alle Systeme, die einfache Integration mit
andere Technologien, eine große und dynamische Entwicklergemeinschaft sowie
die Dokumentation und die Fülle an verfügbaren Ressourcen. Im Rahmen dieser
Arbeit wird der Selenium Web Driver mit der Programmiersprache Java verwendet.