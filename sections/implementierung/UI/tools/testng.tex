\subsubsection{TestNG}

TestNG ist ein Open-Source-Testautomatisierungs-Framework für Java. Es wird
nach dem Vorbild von JUnit und NUnit entwickelt. Einige fortgeschrittene und
nützliche Funktionen, die TestNG bietet, machen es zu einem robusteren
Framework im Vergleich zu seinen Mitbewerbern (vgl. \cite{browserstack}).
Das NG in TestNG steht für "Next Generation". Es wird immer häufiger von
Entwicklern und Testern bei der Erstellung von Testfällen verwendet, da es
einfach ist, mehrere Annotationen, Gruppierungen, Abhängigkeiten,
Priorisierungen und Parametrierungsfunktionen zu verwenden. Durch die
Beseitigung der meisten Einschränkungen des älteren Frameworks gibt TestNG
dem Entwickler die Möglichkeit, flexiblere und leistungsfähigere Tests zu
schreiben.

Der Hauptgrund, warum TestNG für die Entwicklung der Tests in dieser
Arbeit ausgewählt wurde, ist in erster Linie die Tatsache, dass es im
Vergleich zu seinem Konkurrenten JUnit mehr nützliche Funktionen
bietet (vgl. \cite{browserstack}):

\begin{enumerate}
    \item Annotationen von TestNG sind im Vergleich zu JUnit
    einfacher zu verstehen
    \item TestNG erfordert im Gegensatz zu JUnit keine
    obligatorische Deklaration von @BeforeClass und @AfterClass
    \item Die Funktion der Parametrisierung, die TestNG bietet, ist
    bequemer und einfacher durch den Datenprovider zu nutzen.
    \item Funktionen wie Priorisierung und Gruppierung von Tests,
    die von TestNG bereitgestellt werden, machen es im Vergleich zu
    JUnit realistischer und leichter anpassbar.
    \item TestNG bietet im Vergleich zu JUnit in mehrfacher
    Hinsicht die Möglichkeit der parallelen Testausführung.

\end{enumerate}

Neben diesen Vorteilen gegenüber JUnit lässt sich TestNG leicht in das
Selenium-Framework integrieren. Dies macht es zu einem der am häufigsten
verwendeten Werkzeuge für das Schreiben von Selenium-Testskripten
(vgl. \cite{selenium-survey}, S.13).
