\subsubsection{Verlust der Vertraulichkeit von Daten}

Der Verlust der Vertraulichkeit von  Daten tritt auf, wenn eine
Organisation unwissentlich sensible Daten exponiert oder wenn ein
Sicherheitsvorfall dazu führt, dass sensible Daten versehentlich
oder rechtswidrig zerstört, verloren, verändert, unbefugt
offengelegt oder unbefugt darauf zugegriffen wird. Eine solche
Datenexposition kann das Ergebnis eines unzureichenden Schutzes einer
Datenbank, einer Fehlkonfiguration bei der Erstellung neuer Instanzen
von Datenspeichern, einer unsachgemäßen Nutzung von Datensystemen sein.

Klassische Beispiele dafür sind in Klartext gespeicherte Daten, wie
Passwörter oder Kreditkartendaten, fehlendes HTTPS auf
authentifizierten Webseiten, und Hash-Passwörter, die ohne zugefügtes
“Salz” erstellt wurden. Salz bezeichnet eine zufällige Zeichenfolge,
die den Klardaten vor der Verschlüsselung hinzugefügt werden. Dadurch
kann anhand identischer Hashwerte nicht darauf geschlossen werden,
dass es sich um dieselben Daten handelt.