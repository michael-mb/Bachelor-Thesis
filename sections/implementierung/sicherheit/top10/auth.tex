\subsubsection{Fehler in der Authentifizierung}

Wenn die Authentifizierungsfunktionen  einer Anwendung nicht korrekt
implementiert sind, können Angreifer Passwörter oder Sitzungs-IDs
kompromittieren oder andere Implementierungsfehler ausnutzen, indem
sie die Anmeldedaten anderer Benutzer verwenden. Diese Schwachstelle
wird als ``Fehler in der Authentifizierung'' (Broken authentication
in Englisch) bezeichnet. Es wird in der Regel durch schlecht
implementierte Authentifizierungs- und Sitzungsverwaltungsfunktionen
verursacht. In diesem Fall Angriffe zielen darauf ab, die Kontrolle
über ein oder mehrere Benutzerkonten zu erlangen, indem der Angreifer
die gleichen Privilegien erhält wie der angegriffene Benutzer.
Es wird von ``Fehler in der Authentifizierung'' besprochen ,
wenn Angreifer in der Lage sind, Passwörter, Schlüssel oder
Sitzungs-Tokens, Benutzerkontoinformationen und andere
Details zu kompromittieren, um Benutzeridentitäten zu übernehmen.
Zu den üblichen Risikofaktoren gehören:


\begin{enumerate}
    \item Vorhersehbare Anmeldekennungen
    \item Benutzerauthentifizierungskennungen, die nicht geschützt sind, wenn sie gespeichert werden.
    \item Sitzungs-IDs, die in der URL offengelegt werden (z. B. durch URL-Rewriting).
    \item Sitzungs-IDs, die anfällig für Session-Fixing-Angriffe sind.
    \item Sitzungswert, der nach dem Abmelden nicht unterbrochen oder ungültig gemacht wird.
    \item Sitzungskennungen, die nach einer erfolgreichen Anmeldung nicht erneuert werden.
    \item Passwörter, Sitzungskennungen und andere Identifikationsinformationen, die über unverschlüsselte Verbindungen gesendet werden.
\end{enumerate}