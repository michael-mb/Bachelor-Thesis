\subsubsection{Cross-Site-Scripting (XSS)}

Cross-Site-Scripting (XSS)-Angriffe sind eine Art von
Injektion, bei der bösartige Skripte in Websites
eingeschleust werden (vgl. \cite{xss}). Cross-Site
Scripting (XSS)-Angriffe treten auf, wenn :

\begin{enumerate}
    \item Daten gelangen über eine nicht vertrauenswürdige
    Quelle, meist eine Webanfrage, in eine Webanwendung.
    \item Die Daten sind in einem dynamischen Inhalt
    enthalten, der an einen Webnutzer gesendet wird,
    ohne auf bösartige Inhalte überprüft zu werden.
\end{enumerate}

Der Browser des Benutzers hat keine Möglichkeit zu erkennen,
dass das Skript nicht vertrauenswürdig ist, und führt es
aus. Da er glaubt, dass das Skript aus einer
vertrauenswürdigen Quelle stammt, kann das bösartige
Skript auf alle Cookies, Sitzungstoken oder andere
sensible Informationen zugreifen, die vom Browser
gespeichert und mit dieser Website verwendet werden.
Diese Skripte können sogar den Inhalt der HTML-Seite
umschreiben.


Der bösartige Inhalt, der an den Webbrowser gesendet
wird, hat häufig die Form eines JavaScript-Segments,
kann aber auch HTML, Flash oder jede andere Art von
Code enthalten, die der Browser ausführen kann. Die
Vielfalt der XSS-basierten Angriffe ist nahezu unbegrenzt,
aber sie umfassen in der Regel :

\begin{enumerate}
    \item Die Übermittlung privater Daten wie Cookies oder
    andere Sitzungsinformationen an den Angreifer.
    \item Die Umleitung des Opfers auf Webinhalte,
    die vom Angreifer kontrolliert werden.
    \item die Ausführung anderer bösartiger Operationen auf
    dem Rechner des Benutzers unter der verwundbaren Website.
\end{enumerate}
