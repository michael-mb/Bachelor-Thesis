\subsubsection{Nutzung von Komponenten mit bekannten Schwachstellen}

Bekannte Schwachstellen sind Sicherheitslücken, die entweder vom
Entwickler/Verkäufer der verwendeten Produkte, vom
Benutzer/Entwickler oder vom Hacker/Intrusor/Angreifer identifiziert
wurden. Um bekannte Sicherheitslücken auszunutzen, identifizieren Hacker
eine schwache Systemkomponente, indem sie das System mithilfe
automatisierter Tools analysieren (häufiger, da diese Hacking-Tools
online verfügbar sind) oder die Komponenten manuell analysieren
(seltener, da dies fortgeschrittenere Fähigkeiten erfordert).


Fast jede Anwendung weist mindestens einige Schwachstellen auf,
die auf Schwächen in den Komponenten oder Bibliotheken zurückzuführen
sind, von denen die Anwendung abhängt (vgl. \cite{owasp-comp}).
Manchmal sind die Schwachstellen absichtlich eingebaut. Anbieter sind
dafür bekannt, Backdoor-Schwachstellen in ihren Systemen zu hinterlassen,
um nach dem Einsatz des Systems aus der Ferne auf dieses zugreifen zu
können. Die meisten Schwachstellen sind jedoch unbeabsichtigt. Sie sind
auf Sicherheitslücken zurückzuführen, die dem Produktdesign inhärent
sind (vgl. \cite{owasp-comp}).

