\subsubsection{Fehler in der Zugriffskontrolle}

Fehler in der Zugriffskontrolle ist eine Sicherheitslücke,
die es Angreifern ermöglicht, Berechtigungsgarantien zu
umgehen und Tätigkeiten auszuführen, als wären sie
privilegierte Benutzer.

Die Zugriffskontrolle sorgt dafür, dass Benutzer nicht
außerhalb der ihnen zugewiesenen Befugnisse handeln
können. Fehler führen in der Regel zur unbefugten
Offenlegung von Informationen, zur Änderung oder Zerstörung
aller Daten oder zur Ausführung einer Geschäftsfunktion
außerhalb der Grenzen des Benutzers. Häufige Schwachstellen
bei der Zugriffskontrolle sind:

\begin{enumerate}
    \item Umgehung von Zugriffskontrollprüfungen durch
    Änderung der URL, des internen Anwendungsstatus oder
    der HTML-Seite oder einfach durch Verwendung eines
    benutzerdefinierten API-Angriffstools.

    \item Ermöglichung der Änderung des Primärschlüssels
    in den Datensatz eines anderen Benutzers, wodurch
    die Anzeige oder Bearbeitung des Kontos eines anderen
    Benutzers ermöglicht wird.

    \item Ausweitung der Rechte. Als Benutzer handeln, ohne
    eingeloggt zu sein, oder als Administrator handeln, wenn
    man als Benutzer eingeloggt ist.

    \item Manipulation von Metadaten, wie z. B. die
    Wiedergabe oder Manipulation eines JSON Web Token
    (JWT)-Zugangskontrolltokens oder eines Cookies oder
    eines versteckten Feldes, das manipuliert wurde, um
    die Privilegien zu erhöhen, oder der Missbrauch der
    JWT-Ungültigkeitserklärung.

    \item CORS-Fehlkonfiguration ermöglicht
    nicht autorisierten API-Zugriff.

    \item Erzwingen des Navigierens auf authentifizierten
    Seiten als nicht authentifizierter Benutzer oder
    auf privilegierten Seiten als Standardbenutzer.
    Zugriff auf API mit fehlenden Zugriffskontrollen
    für POST, PUT und DELETE.
\end{enumerate}
