\subsubsection{Injektion}

Eine Injektion ist eine Schwachstelle, die einem Angreifer ermöglicht,
bösartigen Code durch eine Anwendung auf ein anderes System zu
schleusen. Dabei können sowohl Backend-Systeme als auch andere
Clients, die mit der angreifbaren Anwendung verbunden sind, gefährdet
werden. Zu den Auswirkungen dieser Angriffe gehören:

\begin{enumerate}
    \item Einem Angreifer erlauben, Betriebssystemaufrufe auf einem
    Zielrechner auszuführen.
    \item Einem Angreifer ermöglichen, Backend-Datenspeicher
    zu kompromittieren.
    \item Einem Angreifer ermöglichen, die Sitzungen anderer
    Benutzer zu kompromittieren oder umzuleiten.
    \item Einem Angreifer erlauben, Aktionen im Namen anderer
    Benutzer oder Dienste zu erzwingen.
\end{enumerate}

Viele Webanwendungen hängen von Betriebssystemfunktionen, externen
Programmen und der Verarbeitung von Datenabfragen ab, die von
Benutzern eingereicht werden. Wenn eine Webanwendung Informationen
aus einer HTTP-Anfrage als Teil einer externen Anfrage weitergibt,
sollten Sie eine Möglichkeit zur Überprüfung und Validierung der
Nachricht einrichten. Andernfalls kann ein Angreifer spezielle
(Meta-)Zeichen, bösartige Befehle/Codes oder Befehlsmodifikatoren in
die Nachricht einfügen. Diese Angriffe sind zwar nicht schwer
auszuführen, aber es gibt immer mehr Tools, die nach diesen
Fehlern suchen. Ein Angreifer kann diese Techniken nutzen, um den
Inhalt Ihrer Datenbank zu erhalten, zu beschädigen oder zu zerstören,
Backend-Systeme zu kompromittieren oder andere Benutzer anzugreifen.
Erfolgreiche Injektionsangriffe können ein System vollständig
gefährden oder zerstören. Es ist wichtig, auf diese Arten von
Angriffen zu testen und sich dagegen zu schützen.

Als Beispiel kann man die SQL-Injektion nennen. Diese ist eine
besonders weit verbreitete und gefährliche Form der Injektion.
Um einen SQL-Injektion-Fehler auszunutzen, muss ein Angreifer
einen Parameter finden, den die Webanwendung an eine
Datenbankinteraktion weitergibt. Ein Angreifer kann dann
bösartige SQL-Befehle in den Inhalt des Parameters einbetten.
Das bringt die Webanwendung dazu, eine bösartige Abfrage an die
Datenbank weiterzuleiten. SQL-Abfragen könnten durch Hinzufügen
zusätzlicher "Einschränkungen" zu einer Anweisung (z. B. OR 1=1)
geändert werden, um Zugriff auf nicht autorisierte Daten zu
erhalten oder diese zu ändern.


