\subsubsection{Sicherheitsrelevante Fehlkonfiguration}

Sicherheitsrelevante Fehlkonfigurationen sind 
Sicherheitskontrollen, die ungenau konfiguriert oder
unsicher sind. Dadurch werden Ihre Systeme und Daten 
gef\"ahrdet. Grunds\"atzlich kann jede schlecht dokumentierte 
Konfigurations\"anderung, Standardeinstellung oder ein 
technisches Problem bei einer beliebigen Komponente zu 
einer Fehlkonfiguration f\"uhren.

Eine Fehlkonfiguration kann aus einer Vielzahl von 
Gr\"unden auftreten. Moderne Netzwerkinfrastrukturen sind
\"au{\ss}erst komplex und zeichnen sich durch st\"andige
Ver\"anderungen aus. Organisationen k\"onnen leicht 
entscheidende Sicherheitseinstellungen \"ubersehen, 
insbesondere bei neuen Netzwerkger\"aten, die 
Standardkonfigurationen beibehalten k\"onnen. Wenn
sichere Konfigurationen f\"ur Zugangspunkte eingerichtet 
werden, m\"ussen Konfigurationen mit Sicherheitskontrollen
h\"aufig \"uberpr\"uft werden, um unvermeidliche 
Konfigurationsabweichungen zu erkennen. Systeme
\"andern sich, neue Ger\"ate werden in das Netzwerk
eingef\"uhrt, Patches werden eingespielt - all dies
tr\"agt zu fehlerhaften Konfigurationen bei.

