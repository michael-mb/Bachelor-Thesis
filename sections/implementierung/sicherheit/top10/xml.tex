\subsubsection{XML External Entities (XML)}

\acs{xml} External Entity Injection (auch bekannt als \acs{xxe}) ist eine
Sicherheitslücke, die einem Angreifer ermöglicht, in die Verarbeitung
von \acs{xml}-Daten durch eine Anwendung einzugreifen. Dieser Angriff
erfolgt, wenn die \acs{xml}-Eingabe, die einen Verweis auf eine externe
Entität enthält, von einem schwach konfigurierten \acs{xml}-Parser
verarbeitet wird. Dieser Angriff kann zur Offenlegung vertraulicher
Daten, Denial of Service, serverseitiger Anfragefälschung,
Portanalyse aus der Sicht des Computers, auf dem sich der Parser
befindet, und anderen Auswirkungen auf das System führen. Um \acs{xxe}
Injections zu verhindern, sollen folgende Punkte beachtet werden:


\begin{enumerate}
    \item Einfachere Datenformate wie JSON zur Datenübertragung verwenden.
    \item Das Verbot in allen \acs{xml}-Parsern, \acs{xml}-Entitäten
     und \acs{dtd}s verändern. \acs{dtd} steht für Document Type Definition und
     bezeichnet einen XML Dokument, in welchem XML-Entities
     definiert werden. Auf dieser Weise wird vermieden,
     dass ein Benutzer eigenen Entities erstellt.
    \item Deaktivierung des Hochladens von XML-Dateien
\end{enumerate}
