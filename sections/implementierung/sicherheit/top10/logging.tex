\subsubsection{Unzureichendes Logging und Monitoring}

Protokolle (Logs) geben Einblick in die Aktivit\"aten einer
Organisation. Die erstellten Protokolle und Pr\"ufpfade 
erm\"oglichen einer Organisation die Fehlerbehebung, die Verfolgung von
Ereignissen, die Erkennung von Zwischenf\"allen und die Einhaltung
gesetzlicher Vorschriften. Unzureichende Protokollierung und \"uberwachung 
bedeutet, dass sicherheitskritische Informationen nicht protokolliert
werden oder dass das richtige Protokollformat, der richtige Kontext,
die richtige Speicherung, die richtige Sicherheit und eine rechtzeitige
Reaktion fehlen, um einen Vorfall oder eine Sicherheitsverletzung zu 
erkennen.

Laut dem IBM-Bericht \"uber Datenschutzverletzungen aus dem Jahr 2020 dauert
es durchschnittlich 280 Tage, bis eine Datenschutzverletzung entdeckt und
einged\"ammt wird (vgl. \cite{ibm-sec}, S. 11). Protokolle sind ein wichtiger
Bestandteil der Reaktion auf Vorf\"alle. Ein Unternehmen kann von einer 
Sicherheitsverletzung \"uberrascht werden, die unentdeckt bleiben kann 
und irreparable rechtliche, finanzielle und regulatorische Folgen 
haben kann. Eine ordnungsgem\"a{\ss}e Protokollverwaltung sorgt f\"ur eine
schnellere Erkennung und Eind\"ammung von Sicherheitsverletzungen und
spart dem Unternehmen Zeit, Geld und Ansehen.