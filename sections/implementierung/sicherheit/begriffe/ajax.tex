\subsubsection{\acs{ajax} Spider}

Der \acs{ajax} Spider ist ein Add-on für einen Crawler namens Crawljax.
Das Add-on richtet einen lokalen Proxy in ZAP ein, um mit Crawljax
zu kommunizieren. Mit dem \acs{ajax} Spider ist es möglich Webanwendungen,
die  \acs{ajax} benutzen, in weit größerer Tiefe \gls{crawlen} (Prozess der
automatischen Entdeckung neuer Ressourcen in einer Anwendung) als mit
dem nativen Spider. \acs{ajax} ist eine Reihe von Webentwicklungstechniken, die
verschiedene Webtechnologien auf der Client-Seite verwenden, um
asynchrone Webanwendungen zu erstellen. Mit \acs{ajax} können Webanwendungen
asynchron (im Hintergrund) Daten von einem Server senden und abrufen,
ohne die Anzeige und das Verhalten der bestehenden Seite zu beeinträchtigen.
Diese Computerarchitektur ermöglicht den Aufbau von Webanwendungen und
dynamischen, interaktiven Webseiten.\acs{ajax} spider wird beim Testen von Webanwendungen
empfohlen, die \acs{ajax} nutzen. Für eine vollständige Abdeckung einer
Webanwendung (z. B. um HTML-Kommentare abzudecken) soll auch den
nativen Spider verwendet werden (vgl. \cite{ajax}).
