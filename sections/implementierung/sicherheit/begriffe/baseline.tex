\subsubsection{ZAP Baseline}

Dieses Skript führt zuerst einen ZAP-Spider und dann ein Passiv Scanning
anhand der Daten aus dem Spider aus. Dies ist ein schneller und gründlicher
Prozess, der es ermöglicht, schnell zu erkennen, ob es ernsthafte
Schwachstellen gibt, die leicht ausgenutzt werden können. Schließlich erstellt
das Skript einen Bericht (siehe \Cref{fig:baseline}), der von den Testern sorgfältig geprüft werden muss.
Das Skript führt keinen echten "Angriff" durch und läuft nur für eine
relativ kurze Zeit (höchstens ein paar Minuten).


\begin{lstlisting}[language=Dockerfile,label={lst:baseline},caption={ZAP Baseline Ausführungsbefehl}]
command: [ "./wait-for-it.sh", "web:8080", "bash" ,"-c",
"zap-baseline.py -t http://web:8080 -r owaspReport.html" ]

# Web:8080: URL der Neuen Version von jExam, die sich im
Container namens Web befindet.
# ./wait-for-it.sh: Script, das einen Healt-Check
ausführt, um herauszufinden, ob die parametrisierte
Anwendung (In diesem Fall Web:8080) bereits gestartet ist.
# owaspReport.html: Webseite, die am Ende
der Testdurchführung generiert wird.
\end{lstlisting}
