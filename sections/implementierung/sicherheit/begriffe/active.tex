\subsubsection{Active Scanning}

Active Scanning ist eine Methode zur Erkennung von Schwachstellen, bei
der versucht wird, potenzielle Schwachstellen mithilfe bekannter Angriffe
auf ausgewählte Ziele zu finden. Im Gegensatz zum passiven Scanning ist es
nicht risikolos. Es kann potenziell zu schwerwiegenden Problemen auf einem
Webserver führen. Aus diesem Grund wird Testern empfohlen, es nur für ihre
eigenen Anwendungen zu verwenden. Active Scanning ermöglicht es, mehrere
große Sicherheitslücken zu entdecken, die in einer Anwendung vorhanden
sein können. Zu den bekanntesten gehören :

\begin{enumerate}
    \item SQL Injection
    \item Directory Browsing
    \item CRLF Injection
    \item Cross Site Scripting (persistent und Reflected)
    \item Command Injection
    \item .htaccess Information Leak
\end{enumerate}

Es gibt auch viele andere, die in dieser Arbeit nicht behandelt werden.
Informationen zu diesen Sicherheitslücken sind jedoch auf der Website
von Zap Proxy zu finden (vgl. \cite{activ}).

