\subsubsection{Passive Scanning}

Passive Scanning ist eine Methode zur Erkennung von Schwachstellen, die auf
Informationen basiert, die aus Netzwerkdaten gesammelt werden. Beim Sammeln
dieser Informationen gibt es keine direkte Interaktion mit der Zielanwendung
(deshalb passiv). Beim passiven Scannen wird der gesamte Datenverkehr zwischen
dem Browser und der Website gelesen und aufgezeichnet, d. h. die POSTs/GETs
und ihre Antworten. ZAP analysiert diese Daten und sucht in seiner
Angriffsbibliothek nach bekannten Problemen. Passives Scannen führt keine
Angriffe aus und wird daher als harmlos angesehen. Beim passiven Scannen
können viele mögliche Fehler und Schwachstellen in einer Webanwendung
entdeckt werden.  Zu den bekanntesten gehören :

\begin{enumerate}
    \item Cross Domain Script Inclusion
    \item Cross Domain Misconfiguration
    \item X-Debug-Token Information Leak
    \item Username Hash Found
    \item Insecure Authentication
    \item Information Disclosure: Suspicious Comments
    \item Information Disclosure: Referrer
    \item Information Disclosure: In URL
    \item Information Disclosure: Debug Errors
    \item CSRF Countermeasures
\end{enumerate}

Diese Sicherheitslücken werden in dieser Arbeit nicht näher erläutert.
Informationen zu diesen Sicherheitslücken sind jedoch auf der Website von Zap Proxy
zu finden (vgl. \cite{passiv}).