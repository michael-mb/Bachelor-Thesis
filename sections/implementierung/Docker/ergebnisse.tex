\subsection{Erreichte Ergebnisse}

Durch die Verwendung von Docker wurde eine Infrastruktur aufgebaut,
die es ermöglichte, beide Versionen von jExam automatisch zu testen.
Diese Infrastruktur bietet die Möglichkeit, Sicherheits-,
Performance- und funktionale Tests durchzuführen. Dank dieser
Infrastruktur muss der Tester nichts mehr lokal auf seinem
Computer konfigurieren. Sobald die Infrastruktur installiert
ist, muss er nur noch wissen, wie die verschiedenen
Docker-Compose-Befehle funktionieren, um die Dienste zu starten,
die er verwenden möchte. Details zum Code und zur Nutzung der
Infrastruktur sind in der Dokumentation beschrieben.

Die Infrastruktur weist jedoch noch einige Unzulänglichkeiten auf,
die behoben werden sollten. Im nächsten Kapitel werden diese
Unzulänglichkeiten und die möglichen Fehler, die bei der Durchführung
von Tests auftreten können, ausführlich beschrieben. Zweitens
werden mögliche Lösungen und Perspektiven für die Verbesserung der
Infrastruktur und der Testwerkzeuge diskutiert.
