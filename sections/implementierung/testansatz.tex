\section{Vorstellung der Testansatzes}

Die jExam-Webanwendung muss auf drei verschiedenen Ebenen getestet
werden: Funktionalitäten, Performance und Sicherheit. Es ist jedoch
nicht möglich, diese drei Ebenen in einer einzigen Testsuite zu
kombinieren. Dies erfordert die Einrichtung einer Infrastruktur,
die die Erstellung und Ausführung der Tests verwaltet. Die Anwendung,
die verwendet wird, um die verschiedenen Dienste zu trennen, ist Docker \cite{docker}
(wird in den folgenden Kapiteln behandelt). Docker  wird nicht nur für
die Trennung der Testebenen verwendet. Sie ermöglicht es auch, beide
Versionen von jExam zu deployen, sodass sie effizient und lokal
getestet werden können. Über diese Infrastruktur wird es möglich
sein, Tests auszuführen und am Ende ihrer Ausführung Berichte und
Metriken zu erhalten. Über diese Infrastruktur wird es möglich sein,
Tests auszuführen und am Ende ihrer Ausführung Berichte und Metriken
zu erhalten. Diese ermöglichen , die Leistung der beiden Plattformen
zu beobachten und zu vergleichen. In den nächsten Kapiteln werden
alle diese Konzepte im Detail behandelt.

