\section{Webanwendungen}

Das weite Feld der Softwareentwicklung umfasst auch die
Entwicklung von Webanwendungen. Lange Zeit wurden
Anwendungen als kompakte, installierbare Programme
verkauft. Dabei handelt es sich um die so genannten
klassischen Anwendungen oder Computeranwendungen, die
lokal auf einem Computer, Mobiltelefon oder Tablet
installiert werden müssen. Im Gegensatz zu herkömmlichen
Anwendungen werden Webanwendungen nicht lokal auf dem Gerät
des Nutzers installiert, sondern auf einem Server, so dass
sie über eine bestimmte \acs{url} zugänglich sind.

Nach Kappel et al. ist eine Webanwendung ein Softwaresystem,
das auf den Spezifikationen des World Wide Web Consortiums
(\acs{w3c}) basiert und Webressourcen bereitstellt, die über
eine Benutzeroberfläche wie einen Webbrowser genutzt werden
können (vgl. \cite{kappel1}, S.02).

Für den Betrieb einer Webanwendung werden mehrere
Computersystemkomponenten benötigt.Erstens das \gls{frontend},
das die grafische Oberfläche bezeichnet, mit der der
Benutzer interagiert, gefolgt vom \gls{backend}, das die
Logik der Anwendung enthält (z.B Datenbanken, Cloud).


Der Benutzer greift auf die Webanwendung über einen Computer
zu, der als Client bezeichnet wird. Der Client sendet eine
oder mehrere Anfragen über das Internet oder Intranet via
\acs{http} oder \acs{https}  an einen anderen Computer (Server),
auf dem die Webanwendung läuft. Der Server nimmt dann die
HTTP-Anfragen entgegen und verarbeitet sie. Je nach Anfrage
werden die angeforderten Daten entweder aus der Datenbank
abgerufen oder gespeichert. Die verarbeiteten Daten werden
dann vom Server in einer entsprechenden Antwort
(HTTP-Response) an den Client zurückgesendet und im
Webbrowser angezeigt. So funktionieren Webanwendungen grundsätzlich.

Um Webanwendungen zu testen, ist es wichtig, eine Testumgebung zu schaffen.
Diese Testumgebung ermöglicht es dem Tester, alle möglichen Schwachstellen in einer Webanwendung zu untersuchen,
ohne die Anwendung zu gefährden, wenn sie bereits in Produktion ist.

