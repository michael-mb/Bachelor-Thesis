\subsection{Nichtfunktionale Tests}

Nichtfunktionales Testen ist das Testen einer Softwareanwendung oder
eines Systems auf seine nichtfunktionalen Anforderungen. Es dient dazu,
die Bereitschaft eines Systems in Bezug auf nichtfunktionale Parameter zu
testen, die bei funktionalen Tests nicht berücksichtigt werden.
Nichtfunktionale Tests beziehen sich auf verschiedene Aspekte der Software
wie Leistung, Belastung, Stress, Skalierbarkeit, Sicherheit, Kompatibilität.
Der Schwerpunkt liegt auf der Verbesserung der Benutzererfahrung.


Es gibt viele Arten von Nichtfunktionalen Tests, von denen einige in den
folgenden Abschnitten behandelt werden:

\textbf{Performancetests}

Nach Vokolos et al. sind Performancetests die Gesamtheit aller Aktivitäten,
die an der Bewertung der Leistung einer Software in \gls{prod}
beteiligt sind (\cite{vokolos1998performance}, S. 80).
Diese Leistung wird aus der Sicht des Benutzers bewertet und in der
Regel in Form von Durchsatz, Reaktionszeit auf Stimuli oder einer
Kombination aus beiden bewertet. Performancetests können auch verwendet
werden, um den Grad der Verfügbarkeit bzw. Stabilität der Software zu
bewerten. Für viele Anwendungen in der Telekommunikation oder im
medizinischen Bereich ist es beispielsweise von entscheidender Bedeutung,
dass das System immer verfügbar ist.


Performancetests sind entscheidend, um festzustellen, ob eine Anwendung die
Leistungsanforderungen erfüllt (z.B.\ muss das System bis zu 1 000
gleichzeitige Benutzer verwalten können). Sie helfen auch dabei, Engpässe
(\gls{bottleneck}) in einer Anwendung zu lokalisieren. Sie werden
für den Vergleich von zwei oder mehr Systemen verwendet, um das
leistungsfähigste zu identifizieren (z. B. \gls{jexam_2009}  und \gls{jexam_new}).
Es ermöglicht auch eine effektive Messung der Stabilität in Bezug auf den
Datenverkehr. Performancetests unterteilen sich in weitere Unterkategorien:


\textbf{Lasttests}: Sie sind eine Art von Performancetests,
bei denen die Anwendung auf ihre Leistung bei normaler und Spitzenbelastung
getestet wird. Die Leistung einer Anwendung wird im Hinblick auf ihre Reaktion
auf Benutzeranfragen und ihre Fähigkeit, innerhalb einer akzeptierten Toleranz
bei unterschiedlichen Benutzerlasten konsistent zu reagieren, überprüft.


\textbf{Stresstests}: Sie werden eingesetzt, um Wege zu finden, das System zu
brechen. Der Test liefert auch den Bereich der maximalen Belastung,
die das System aushalten kann. In der Regel wird beim Stresstest ein
schrittweiser Ansatz verfolgt, bei dem die Belastung schrittweise
erhöht wird. Der Test wird mit einer Last begonnen, für die die Anwendung
bereits getestet wurde. Dann wird die Last langsam erhöht, um das System
zu belasten. Der Punkt, an dem wir feststellen, dass die Server nicht mehr
auf die Anfragen reagieren, wird als Sollbruchstelle betrachtet.


Dies sind die einzigen beiden Unterkategorien von Performancetests,
die in dieser Arbeit verwendet werden.



\textbf{Penetrationtests}

Ein Penetrationstest ist ein Versuch, die Sicherheit einer IT-Infrastruktur
zu bewerten, indem auf sichere Weise (in einer gesicherten Umgebung und unter sicheren Umständen)
versucht wird, Schwachstellen auszunutzen. Diese Schwachstellen können in Betriebssystemen, Diensten
und Anwendungsfehlern, unsachgemäßen Konfigurationen oder riskantem
Verhalten der Endbenutzer bestehen. Solche Bewertungen sind auch nützlich,
um die Wirksamkeit von Verteidigungsmechanismen und die Einhaltung von
Sicherheitsrichtlinien durch die Endbenutzer zu überprüfen.


Penetrationstests werden in der Regel mit manuellen oder automatisierten
Technologien durchgeführt, um systematisch Server, Endpunkte, Webanwendungen,
drahtlose Netzwerke, Netzwerkgeräte, mobile Geräte und andere potenzielle
Angriffspunkte zu \"uberpr\"ufen. Sobald die Schwachstellen in einem
bestimmten System erfolgreich ausgenutzt wurden, können die Tester versuchen,
das kompromittierte System für weitere Angriffe auf andere interne Ressourcen
zu nutzen. Insbesondere indem sie versuchen, schrittweise höhere
Sicherheitsstufen und einen tieferen Zugang zu elektronischen Ressourcen
und Informationen über die Ausweitung von Berechtigungen zu erreichen.


Laut Brad Arkin et al. sind Penetrationstests die am häufigsten und am
weitesten verbreiteten Best Practices im Bereich der Softwaresicherheit,
was zum Teil daran liegt, dass es sich um eine attraktive Aktivität am
Ende des Lebenszyklus handelt (vgl. \cite{1392709}, S.84).  Sobald eine
Anwendung fertiggestellt ist, sollen die Tester sie im Rahmen der Endabnahme
Penetrationstests unterziehen. Der Hauptzweck von Penetrationstests besteht
also darin, Schwachstellen in einem System zu identifizieren, um sie zu
schließen. Dies ermöglicht es den Testern, die Sicherheit der Anwendung
zu erhöhen und die Sicherheitsstrategie zu verbessern.


So wie funktionale Tests zeigen, wie gut eine Anwendung funktioniert,
so sind auch nicht-funktionale Tests für die Benutzererfahrung von
großer Bedeutung.

Funktionale Tests zeigen, wie gut eine Anwendung funktioniert.
Aber auch nicht-funktionale Tests sind sehr wichtig, da sie helfen,
bestimmte Kriterien der Nutzererfahrung zu messen, um diese zu verbessern.

