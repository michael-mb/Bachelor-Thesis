\subsection{Funktionale Tests}

Funktionale Tests werden durchgeführt, um zu überprüfen,
ob alle entwickelten Funktionen mit den funktionalen
Spezifikationen übereinstimmen. Dies wird durch die
Ausführung der funktionalen Testfälle erreicht. In der
Funktionstestphase wird das System getestet, indem
Eingaben gemacht, Ausgaben überprüft und die
tatsächlichen Ergebnisse mit den erwarteten
Ergebnissen verglichen werden. Bei diesen Tests werden
Benutzeroberfläche, \acs{api}s, Datenbank, Sicherheit,
Client/Server-Kommunikation und andere Funktionen der zu
testenden Anwendung überprüft. Die Tests können entweder
manuell oder durch Automatisierung durchgeführt werden.

\textbf{Unit-Tests}

James Whittaker erklärt, dass "Unit-Test einzelne Softwarekomponenten oder
eine Sammlung von Komponenten testet" (vgl. \cite{Whittaker2000}, S.70-79).
Unit-Testing ist die erste Stufe des Softwaretests, bei der einzelne
Komponenten eines Softwarepakets getestet werden, während der Rest des
Systems ignoriert wird. Auch Modul- oder Komponententest genannt,
wird Unit-Test während der Entwicklung einer Anwendung durchgeführt,
um zu prüfen, ob die einzelnen Einheiten oder Module einer Anwendung
ordnungsgemäß funktionieren.


Das Ziel von Unit-Tests ist es, Probleme in einem frühen Stadium des
Entwicklungszyklus zu finden. Dadurch werden die Testkosten gesenkt
(die Kosten für das frühzeitige Auffinden eines Fehlers sind wesentlich
geringer als die Kosten für das spätere Auffinden). Sie reduzieren Fehler
bei der Änderung bestehender Funktionen, so dass sie leicht gefunden und
behoben werden können. Dies vereinfacht den \gls{debug}-Prozess erheblich.


\textbf{Integration-Tests}

Es ist eine Erweiterung des Unit-Tests. Beim Integrationstest wird die
Konnektivität oder der Datentransfer zwischen den einzelnen getesteten
Modulen (unit tested modules) getestet. Nach Leung und White sind
Integrationstests die Tests, die durchgeführt werden, wenn alle
einzelnen Module zu einem funktionierenden Programm kombiniert
werden (vgl. \cite{131377}, S.290). Das Testen erfolgt auf Modulebene und nicht
auf Anweisungsebene wie beim Unit-Test. Beim Integrationstest liegt der
Schwerpunkt auf den Interaktionen zwischen den Modulen und ihren
Schnittstellen.


Mit Integrationstests wird überprüft, ob das funktionale und nicht-funktionale
Verhalten der Schnittstellen dem Softwaredesign und -spezifikationen
entspricht. Dies führt dazu, dass Fehler daran gehindert werden, in höhere
Teststufen zu gelangen.


\textbf{System-Tests}

Systemtests sind eine Stufe der Softwaretests über den Integrationstests.
Laut Briand et al. werden sie an einer vollständigen und voll integrierten
Anwendung durchgeführt, um die Übereinstimmung des Systems mit den
spezifizierten Anforderungen zu bewerten (vgl. \cite{briand2002uml}, S.10).
Bei dieser Art von funktionalen tests validieren die Tester das vollständige
und integrierte Softwarepaket. Damit wird sichergestellt, dass es den
Anforderungen entspricht. Bei Bedarf können die Tester Feedback zur
Funktionalität und Leistung der App oder Website geben, ohne vorher
zu wissen, wie sie programmiert wurde. Dies hilft den Teams bei der
Entwicklung von Testfällen, die in Zukunft verwendet werden sollen.
Systemtests werden auch als End-to-End-Tests bezeichnet.


\textbf{Automatisierte Benutzeroberfläche-Tests (\acs{ui} Automation Testing) }

Die erste Interaktion zwischen einem Benutzer und einer Software findet
über eine grafische Benutzeroberfläche (\acs{gui} acronym) statt. Beim
\acs{ui}-Testing wird überprüft, ob die Endbenutzeroberfläche korrekt
funktioniert. Bei der Durchführung von \acs{ui}-Tests wird geprüft, ob jedes
Stückchen Logik, jede \acs{ui}-Funktion oder jeder Aktionsablauf wie erwartet
funktioniert. Hier konzentrieren sich die Tester auf die Validierung jedes
Klicks auf eine Schaltfläche, der Dateneingabe, der Navigation, der
Berechnung von Werten und anderer Funktionalitäten, die für die
Benutzerinteraktion verwendet werden.


\acs{ui} Automation Testing ist eine Technik, bei der diese Testprozesse mithilfe
eines Automatisierungstools durchgeführt werden. Anstatt dass sich die
Tester durch die Anwendung klicken, um die Daten- und Aktionsflüsse
visuell zu überprüfen, werden für jeden Testfall Testskripte geschrieben.
Anschließend wird eine Reihe von Schritten hinzugefügt, die bei der
Überprüfung der Daten zu befolgen sind. Der Prozess der
\acs{ui}-Automatisierungstests vereinfacht die Erstellung von UI-Tests, die
Ausführung der Tests und die Anzeige der Ergebnisse (vgl.\cite{Perfecto2020}).
Es ermöglicht Testern, die Interaktion einer Anwendung mit den Endnutzern zu
simulieren und zu testen. Weitere Vorteile sind die Automatisierung aller
Testaktivitäten für die zu testende Anwendung und die Integration von
Benutzerschnittstellentests in den Entwicklungsprozess.


Automatisiertes Testen ist eine der Funktionen, die im Mittelpunkt dieser
Arbeit stehen. Es ist daher wichtig, dieses Konzept zu verstehen, um
weiterzukommen.


