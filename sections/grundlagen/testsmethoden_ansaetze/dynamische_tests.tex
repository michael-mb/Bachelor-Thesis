\subsection{Dynamisches Testen}

Dynamisches Testen ist eine Methode zur Bewertung
der Durchführbarkeit eines Softwareprogramms durch
Eingabe und Prüfung der Ausgabe. Die dynamische Methode
erfordert, dass der Code kompiliert und ausgeführt
wird. Dynamische Tests werden in zwei Kategorien unterteilt:
Whitebox- und  Blackboxtestverfahren.

\textbf{Whitebox-Testverfahren}

Whitebox-Testen ist eine Softwaretestmethode, bei
der dem Tester die interne Struktur/das Design
bekannt ist. Das Hauptziel  vom Whitebox-Testen ist
die Überprüfung der Korrektheit der Software
Anweisungen, Codepfade, Bedingungen, Schleifen
und Datenflüsse. Dieses Ziel wird oft als logische
Abdeckung bezeichnet (vgl. \cite{shultz2011software}, S.107).


\textbf{Blackbox-Testverfahren}

Blackbox-Testen ist eine Testmethode, bei der die
interne Struktur/der Code/das Design dem Tester nicht
bekannt ist. Das Hauptziel dieses Tests ist es, die
Funktionalität des zu testenden Systems zu überprüfen,
und diese Art von Tests erfordert die Ausführung der
kompletten Testsuite (vgl. \cite{shultz2011software}, S.112).


Die Blackbox-Methode ist diejenige, die in dieser
Arbeit entwickelt wird. Sie selbst ist in mehrere
Arten unterteilt, auf die im nächsten Kapitel
eingegangen wird.

