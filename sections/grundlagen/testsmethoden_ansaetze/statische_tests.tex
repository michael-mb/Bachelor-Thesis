\subsection{Statisches Testen}

Statisches Testen ist eine Softwaretestmethode,
bei der ein Programm zusammen mit den zugehörigen
Dokumenten untersucht wird, ohne dass das Programm
ausgeführt werden muss. Konkret besteht der Prozess aus
der Prüfung schriftlicher Dokumente, die insgesamt
einen Überblick über die zu testende Softwareanwendung
geben. Zu den geprüften Dokumenten gehören
Anforderungsspezifikationen, Designdokumente,
Benutzerdokumente, Webseiteninhalte, Quellcode,
Testfälle, Testdaten und Testskripte,
Benutzerdokumente, Spezifikations- und
Matrixdokumente. Statische Tests erleichtern die
Kommunikation zwischen den Teams und vermitteln
einen besseren Eindruck von den Qualitätsproblemen
in der Software. Es reduziert nicht nur die Kosten
in den frühen Entwicklungsphasen (in Bezug auf die
Menge an Arbeit, die neu gemacht werden muss, um
eventuelle Fehler zu korrigieren), sondern auch die
Entwicklungszeit.
