\chapter{Grundlagen}\label{ch:grundlagen}

In diesem Kapitel werden die notwendigen Definitionen und Methoden erläutert,
die für das Verständnis der Arbeit von Bedeutung sind. Zunächst werden die
Ziele und Grenzen des Softwaretestens definiert. Dann folgt eine kurze
Einführung in den Begriff der Webanwendung und  die Erklärung des Konzeptes
der Testumgebung. Um mehr über die Vorteile der Automatisierung zu erfahren,
wird im nächsten Teil das Konzept der Testautomatisierung erläutert, gefolgt
von der Vorstellung verschiedener Testmethoden und -ansätze. Das Konzept des
Testens im Bereich der Softwareentwicklung ist umfangreich und kann in dieser
Arbeit nicht vollständig behandelt werden. Daher werden die verschiedenen
Testarten, die in dieser Arbeit von Nutzen sein werden, am Ende vorgestellt.


\section{Ziele und Grenzen des Softwaretestens}

Die Standarddefinition des Testens nach dem
ANSI/IEEE 1059-Standard besagt, dass Testen der
Prozess der Analyse eines Softwareobjekts ist, um
Unterschiede zwischen bestehenden und erforderlichen
Bedingungen (d.h. Defekte/Fehler/Bugs) zu erkennen
und die Eigenschaften des Softwareobjekts zu bewerten (vgl. \cite{singh2012software}, S.07).
Das Softwaretesten ist daher eine Methode, um zu überprüfen,
ob das Softwareprodukt den erwarteten
Anforderungen entspricht und um sicherzustellen, dass
es frei von Fehlern ist.

Im 21. Jahrhundert ist der Einsatz von Software und
Anwendungen weit verbreitet und kein Bereich bleibt
davon verschont. Die Gesamtmenge der weltweit erstellten,
erfassten, kopierten und verbrauchten Daten ist laut
Statista (vgl. \cite{Statista2021}) bis 2020 rasant auf 64,2
Zettabyte \begin{math}(2^{70})\end{math} angestiegen. Die Nutzer vertrauen ihre sensiblen und privaten Daten
Plattformen an, deren Aufgabe ist es, sie zu schützen. Testen
ist wichtig, weil Softwarefehler teuer oder sogar gefährlich
sein können. Sie können finanzielle
und menschliche Verluste verursachen, und die Geschichte
ist voll von solchen Beispielen:

\noindent
\begin{enumerate}
    \item Im Mai 1996 führte ein Softwarefehler dazu, dass
     den Konten von 823 Kunden einer großen US-Bank 920
     Millionen US-Dollar gutgeschrieben wurden (vgl. \cite{Devi2015}).
    \item China Airlines Airbus A300 crashed due to a software bug on April 26,
     1994, killing 264 innocents live (vgl. \cite{Takeuch1996}).
\end{enumerate}

Das Testen einer Anwendung hat viele Vorteile. Zu den wichtigsten
gehören die folgenden:


 \textbf{Kosteneffektivität}: Das
rechtzeitige Testen eines IT-Projekts hilft auf
lange Sicht Geld zu sparen. Wenn die Fehler bereits in
der frühen Phase des Softwaretests entdeckt werden,
kostet es weniger, sie zu beheben. Es ist besser, mit
dem Testen früher zu beginnen und es in jeder Phase des
Lebenszyklus der Softwareentwicklung einzuführen.
Regelmäßige Tests sind erforderlich (vgl. \cite{kumar2010software}, S.53), um
sicherzustellen, dass die Anwendung gemäß den Anforderungen entwickelt wird.

\begin{figure}[H]
    \centering
    \includegraphics[scale=0.5]{images/Cost-of-fixing-bugs-in-different-phases}
    \caption{Kosten für die Behebung von Fehlern (Bugs) in verschiedenen Phasen (vgl. \cite{kumar2010software})} \label{fig:mof}
\end{figure}


\textbf{Erhöhung der Sicherheit}: Sicherheit ist der anfälligste und
sensibelste Teil des Softwaretestens. Durch Testen wird sichergestellt,
dass die Anwendung über einen minimalen Schutz verfügt. Testen hilft
dabei, Risiken und Probleme früher zu beseitigen. So wird die Anwendung für Nutzer attraktiv,
die vertrauenswürdige Produkte suchen (vgl. \cite{shultz2011software}, S.09).

\textbf{Produktqualität}: Sie ist eine wesentliche
Voraussetzung für jedes Softwareprodukt. Zu den sechs Gruppen von Software-Qualitätsindikatoren
in der ISO-Norm 9126 (vgl. \cite{AlainAbran2010}) gehört die Wartbarkeit, zu der auch die Untergruppe Testbarkeit gehört.
Durch Testen kann die Qualität einer Anwendung sowie ihre Wartbarkeit erhöht werden und so wird dem Kunden sichergestellt,
dass ein Qualitätsprodukt geliefert wird.


Aus diesen Gründen ist das Testen von Software ein
integraler Bestandteil des Softwareentwicklungsprozesses, jedoch hat es Grenzen.
Testen dient nur dazu, das Vorhandensein von potentiellen Fehlern
aufzudecken. Aber es kann nicht sicherstellen, dass
die Software keine Fehler oder Bugs enthält (vgl. \cite{kumar2010software}, S.55).
Dazu können Tests nicht nachweisen, dass ein Produkt unter allen
Bedingungen richtig funktioniert, sondern nur, dass es unter
bestimmten Bedingungen nicht richtig funktioniert (vgl. \cite{kumar2010software}, S.56).



Da das Ziel dieser Arbeit darin besteht, eine \Gls{TestSuite} für
eine Webanwendung einzurichten, ist es wichtig zu
definieren, was mit Webanwendung eigentlich gemeint ist.

\section{Webanwendungen}

Das weite Feld der Softwareentwicklung umfasst auch die
Entwicklung von Webanwendungen. Lange Zeit wurden
Anwendungen als kompakte, installierbare Programme
verkauft. Dabei handelt es sich um die so genannten
klassischen Anwendungen oder Computeranwendungen, die
lokal auf einem Computer, Mobiltelefon oder Tablet
installiert werden müssen. Im Gegensatz zu herkömmlichen
Anwendungen werden Webanwendungen nicht lokal auf dem Gerät
des Nutzers installiert, sondern auf einem Server, so dass
sie über eine bestimmte \acs{url} zugänglich sind.

Nach Kappel et al. ist eine Webanwendung ein Softwaresystem,
das auf den Spezifikationen des World Wide Web Consortiums
(\acs{w3c}) basiert und Webressourcen bereitstellt, die über
eine Benutzeroberfläche wie einen Webbrowser genutzt werden
können (vgl. \cite{kappel1}, S.02).

Für den Betrieb einer Webanwendung werden mehrere
Computersystemkomponenten benötigt.Erstens das \gls{frontend},
das die grafische Oberfläche bezeichnet, mit der der
Benutzer interagiert, gefolgt vom \gls{backend}, das die
Logik der Anwendung enthält (z.B Datenbanken, Cloud).


Der Benutzer greift auf die Webanwendung über einen Computer
zu, der als Client bezeichnet wird. Der Client sendet eine
oder mehrere Anfragen über das Internet oder Intranet via
\acs{http} oder \acs{https}  an einen anderen Computer (Server),
auf dem die Webanwendung läuft. Der Server nimmt dann die
HTTP-Anfragen entgegen und verarbeitet sie. Je nach Anfrage
werden die angeforderten Daten entweder aus der Datenbank
abgerufen oder gespeichert. Die verarbeiteten Daten werden
dann vom Server in einer entsprechenden Antwort
(HTTP-Response) an den Client zurückgesendet und im
Webbrowser angezeigt. So funktionieren Webanwendungen grundsätzlich.

Um Webanwendungen zu testen, ist es wichtig, eine Testumgebung zu schaffen.
Diese Testumgebung ermöglicht es dem Tester, alle möglichen Schwachstellen in einer Webanwendung zu untersuchen,
ohne die Anwendung zu gefährden, wenn sie bereits in Produktion ist.


\section{Testumgebung}

Im Moment ist die alte Version von jexam (\gls{jexam_2009})
in \gls{prod}, weil sie stabil und funktional ist.
Die Entwickler des Teams sorgen dafür, dass die
Funktionen der Anwendung ordnungsgemäß funktionieren.
Dies ist bereits eine Art manueller Test. Die neue
Version, die derzeit entwickelt wird, wurde noch nicht
vollständig getestet. Daher ist sie von unbekannter
Qualität und kann nicht als stabil für die \gls{prod}
gelten. Sie sollte nicht in die \gls{prod} aufgenommen
werden, damit die Benutzer sie testen können.  Wenn
die Entwickler dies täten, würden sie Schwachstellen
aufdecken, die von böswilligen Personen genutzt werden
könnten, um die Sicherheit der Nutzerdaten zu
gefährden. Bevor eine Anwendung in \gls{prod} geht,
sollte sie getestet werden. Da dies nicht in der
\gls{prod} geschehen sollte, ist es notwendig, eine
Umgebung zu schaffen, in der die Anwendung getestet
werden kann.


Nach Everett eine Testumgebung ist eine Umgebung,
die Hardware, Instrumente, Simulatoren, Software-Tools
und andere unterstützende Elemente enthält, die für
die Durchführung eines Tests erforderlich sind,
d.h. sie ermöglicht es, die Anwendung zu testen, ohne
den Kunden zu beeinträchtigen (vgl. \cite{shultz2011software}, S. 150-152).
Die Einrichtung einer Testumgebung hat eine Reihe von Vorteilen :

\noindent
\begin{enumerate}
    \item Die Nutzer nicht mit einer Anwendung von
    zweifelhafter Qualität zu belästigen. Ein
    Qualitätsabfall ist oft sehr nachteilig für eine
    Anwendung. Die Nutzer könnten sich beschweren,
    dass die alte Version besser ist und die ganze
    Arbeit der Entwickler, die die neue Version
    entwickelt haben, preisgeben.

    \item Die Einbeziehung aller Szenarien in die
    Tests, was die Zuverlässigkeit der Software
    erhöht. Sie hilft auch dabei, fehlende
    Implementierungen der Software zu finden. In der
    Produktion wäre dies eine komplizierte Aufgabe, da
    sich die Daten und die Anwendung ständig ändern.

    \item Die Senkung der Produktionskosten durch
    frühzeitige Erkennung von Fehlern.

\end{enumerate}

Testumgebungen sind für jede Softwareentwicklung
notwendig. Sie ermöglichen es, das Produkt zu testen,
bevor es auf den Markt kommt, ohne ein Risiko für die
Produktion einzugehen. Wenn sie jedoch wirklich
wirksam sein sollen, dürfen sie die Entwicklung nicht
behindern, sondern müssen auch zuverlässig sein. Aus
diesem Grund muss diesen Umgebungen besondere
Aufmerksamkeit geschenkt werden. Die endgültige
Qualität des Produkts hängt stark von der Qualität der
Testumgebung ab (vgl. \cite{shultz2011software}, S. 152).


\section{Testautomatisierung}

Nach Sharma et al. ist die Testautomatisierung eine Technik,
bei der Skripte geschrieben werden, um einen manuellen
Testprozess zu automatisieren (vgl. \cite{sharma2014web}, S.909). Bei der
Testautomatisierung werden vordefinierte Tests durchgeführt,
um verschiedene Funktionen in einer Anwendung zu überprüfen.
Dies bedeutet, dass Tools und Testskripte verwendet werden, um verschiedene
Zustände zu erzeugen und Daten vorzubereiten. Anschließend wird eine
Reihe von Schritten ausgeführt, um ein Szenario zu validieren.
Auf diese Weise können die Tester feststellen, ob eine Anwendung
wie erwartet funktioniert oder nicht. Entwicklungs- und Testteams
entscheiden sich aus mehreren Gründen für die Testautomatisierung.
Zu den wichtigsten gehören:

\textbf{Zeit}: Manuelle Tests sind langsam und können mit vielen Entwicklungsprozessen
nicht mithalten (vgl. \cite{sharma2014web}, S.910).


\textbf{Kosten}: Manuelle Tests sind ressourcenintensiv und kostspielig
(vgl. \cite{sharma2014web}, S.910).


\textbf{Genauigkeit}: Manuelle Tests sind bei der Durchführung sich wiederholender
Aufgaben fehleranfällig. Umgekehrt verringert die Automatisierung die
Wahrscheinlichkeit dieser Fehler (vgl. \cite{sharma2014web}, S.910).


\textbf{Umfang}: Bei der Durchführung komplexer Iterationen ist es schwierig,
sich auf manuelle Tests zu verlassen (vgl. \cite{sharma2014web}, S.910).


Laut dem Global Quality Report profitieren Unternehmen auf unterschiedliche
Weise von automatisierten Tests. Rund 60\% der Unternehmen gaben an,
dass sich die Fähigkeit zur Erkennung von Anwendungsfehlern durch eine höhere
Testabdeckung verbessert hat (vgl. \cite{Buenen201718}, S.30). Weitere 57\% stellten fest, dass die Wiederverwendung von Testfällen nach
dem Einsatz der Automatisierung zunahm. Gleichzeitig verzeichneten 54\%
eine Verringerung des Zeitaufwands für
Testzyklen (vgl. \cite{Buenen201718}, S.31).

\begin{figure}
    \centering
    \includegraphics[scale=1.3]{images/benefits_of_test_automation}
    \caption{Vorteile der Testautomatisierung (vgl. \cite{Buenen201718}, S.31)} \label{fig:mof}
\end{figure}

In den folgenden Abschnitten werden die verschiedenen Arten von Tests
behandelt. Diese Tests können sowohl manuell als auch automatisch
durchgeführt werden. Da der Schwerpunkt dieser Arbeit auf der
Automatisierung liegt, wird nur dieser Aspekt betrachtet.

\section{Testsmethoden und -ansätze}

Softwareobjekte können auf unterschiedliche Weise
getestet werden. Generell wird zwischen statischen und
dynamischen Tests unterschieden. In den folgenden
Fällen werden beide Konzepte in den folgenden
Abschnitten ausführlicher dargestellt.

\subsection{Statisches Testen}

Statisches Testen ist eine Softwaretestmethode,
bei der ein Programm zusammen mit den zugehörigen
Dokumenten untersucht wird, ohne dass das Programm
ausgeführt werden muss. Konkret besteht der Prozess aus
der Prüfung schriftlicher Dokumente, die insgesamt
einen Überblick über die zu testende Softwareanwendung
geben. Zu den geprüften Dokumenten gehören
Anforderungsspezifikationen, Designdokumente,
Benutzerdokumente, Webseiteninhalte, Quellcode,
Testfälle, Testdaten und Testskripte,
Benutzerdokumente, Spezifikations- und
Matrixdokumente. Statische Tests erleichtern die
Kommunikation zwischen den Teams und vermitteln
einen besseren Eindruck von den Qualitätsproblemen
in der Software. Es reduziert nicht nur die Kosten
in den frühen Entwicklungsphasen (in Bezug auf die
Menge an Arbeit, die neu gemacht werden muss, um
eventuelle Fehler zu korrigieren), sondern auch die
Entwicklungszeit.

\subsection{Dynamisches Testen}

Dynamisches Testen ist eine Methode zur Bewertung
der Durchführbarkeit eines Softwareprogramms durch
Eingabe und Prüfung der Ausgabe. Die dynamische Methode
erfordert, dass der Code kompiliert und ausgeführt
wird. Dynamische Tests werden in zwei Kategorien unterteilt:
Whitebox- und  Blackboxtestverfahren.

\textbf{Whitebox-Testverfahren}

Whitebox-Testen ist eine Softwaretestmethode, bei
der dem Tester die interne Struktur/das Design
bekannt ist. Das Hauptziel  vom Whitebox-Testen ist
die Überprüfung der Korrektheit der Software
Anweisungen, Codepfade, Bedingungen, Schleifen
und Datenflüsse. Dieses Ziel wird oft als logische
Abdeckung bezeichnet (vgl. \cite{shultz2011software}, S.107).


\textbf{Blackbox-Testverfahren}

Blackbox-Testen ist eine Testmethode, bei der die
interne Struktur/der Code/das Design dem Tester nicht
bekannt ist. Das Hauptziel dieses Tests ist es, die
Funktionalität des zu testenden Systems zu überprüfen,
und diese Art von Tests erfordert die Ausführung der
kompletten Testsuite (vgl. \cite{shultz2011software}, S.112).


Die Blackbox-Methode ist diejenige, die in dieser
Arbeit entwickelt wird. Sie selbst ist in mehrere
Arten unterteilt, auf die im nächsten Kapitel
eingegangen wird.




\section{Verschiedene Testarten}

Bei der Blackbox-Methode gibt es zwei Hauptarten von
Testen: funktionale Tests und nicht-funktionale Tests.

\subsection{Funktionale Tests}

Funktionale Tests werden durchgeführt, um zu überprüfen,
ob alle entwickelten Funktionen mit den funktionalen
Spezifikationen übereinstimmen. Dies wird durch die
Ausführung der funktionalen Testfälle erreicht. In der
Funktionstestphase wird das System getestet, indem
Eingaben gemacht, Ausgaben überprüft und die
tatsächlichen Ergebnisse mit den erwarteten
Ergebnissen verglichen werden. Bei diesen Tests werden
Benutzeroberfläche, \acs{api}s, Datenbank, Sicherheit,
Client/Server-Kommunikation und andere Funktionen der zu
testenden Anwendung überprüft. Die Tests können entweder
manuell oder durch Automatisierung durchgeführt werden.

\textbf{Unit-Tests}

James Whittaker erklärt, dass Unit-Test einzelne Softwarekomponenten oder
eine Sammlung von Komponenten testet (vgl. \cite{Whittaker2000}, S.70-79).
Unit-Testing ist die erste Stufe des Softwaretests, bei der einzelne
Komponenten eines Softwarepakets getestet werden, während der Rest des
Systems ignoriert wird. Auch Modul- oder Komponententest genannt,
wird Unit-Test während der Entwicklung einer Anwendung durchgeführt,
um zu prüfen, ob die einzelnen Einheiten oder Module einer Anwendung
ordnungsgemäß funktionieren.


Das Ziel von Unit-Tests ist es, Probleme in einem frühen Stadium des
Entwicklungszyklus zu finden. Dadurch werden die Testkosten gesenkt
(die Kosten für das frühzeitige Auffinden eines Fehlers sind wesentlich
geringer als die Kosten für das spätere Auffinden). Sie reduzieren Fehler
bei der Änderung bestehender Funktionen, so dass sie leicht gefunden und
behoben werden können. Dies vereinfacht den \gls{debug}-Prozess erheblich.


\textbf{Integration-Tests}

Es ist eine Erweiterung des Unit-Tests. Beim Integrationstest wird die
Konnektivität oder der Datentransfer zwischen den einzelnen getesteten
Modulen (unit tested modules) getestet. Nach Leung und White sind
Integrationstests die Tests, die durchgeführt werden, wenn alle
einzelnen Module zu einem funktionierenden Programm kombiniert
werden (vgl. \cite{131377}, S.290). Das Testen erfolgt auf Modulebene und nicht
auf Anweisungsebene wie beim Unit-Test. Beim Integrationstest liegt der
Schwerpunkt auf den Interaktionen zwischen den Modulen und ihren
Schnittstellen.


Mit Integrationstests wird überprüft, ob das funktionale und nicht-funktionale
Verhalten der Schnittstellen dem Softwaredesign und -spezifikationen
entspricht. Dies führt dazu, dass Fehler daran gehindert werden, in höhere
Teststufen zu gelangen.


\textbf{System-Tests}

Systemtests sind eine Stufe der Softwaretests über den Integrationstests.
Laut Briand et al. werden sie an einer vollständigen und voll integrierten
Anwendung durchgeführt, um die Übereinstimmung des Systems mit den
spezifizierten Anforderungen zu bewerten (vgl. \cite{briand2002uml}, S.10).
Bei dieser Art von funktionalen tests validieren die Tester das vollständige
und integrierte Softwarepaket. Damit wird sichergestellt, dass es den
Anforderungen entspricht. Bei Bedarf können die Tester Feedback zur
Funktionalität und Leistung der App oder Website geben, ohne vorher
zu wissen, wie sie programmiert wurde. Dies hilft den Teams bei der
Entwicklung von Testfällen, die in Zukunft verwendet werden sollen.
Systemtests werden auch als End-to-End-Tests bezeichnet.


\textbf{Automatisierte Benutzeroberfläche-Tests (\acs{ui} Automation Testing) }

Die erste Interaktion zwischen einem Benutzer und einer Software findet
über eine grafische Benutzeroberfläche (\acs{gui} acronym) statt. Beim
\acs{ui}-Testing wird überprüft, ob die Endbenutzeroberfläche korrekt
funktioniert. Bei der Durchführung von \acs{ui}-Tests wird geprüft, ob jedes
Stückchen Logik, jede \acs{ui}-Funktion oder jeder Aktionsablauf wie erwartet
funktioniert. Hier konzentrieren sich die Tester auf die Validierung jedes
Klicks auf eine Schaltfläche, der Dateneingabe, der Navigation, der
Berechnung von Werten und anderer Funktionalitäten, die für die
Benutzerinteraktion verwendet werden.


\acs{ui} Automation Testing ist eine Technik, bei der diese Testprozesse mithilfe
eines Automatisierungstools durchgeführt werden. Anstatt dass sich die
Tester durch die Anwendung klicken, um die Daten- und Aktionsflüsse
visuell zu überprüfen, werden für jeden Testfall Testskripte geschrieben.
Anschließend wird eine Reihe von Schritten hinzugefügt, die bei der
Überprüfung der Daten zu befolgen sind. Der Prozess der
\acs{ui}-Automatisierungstests vereinfacht die Erstellung von UI-Tests, die
Ausführung der Tests und die Anzeige der Ergebnisse (vgl.\cite{Perfecto2020}).
Es ermöglicht Testern, die Interaktion einer Anwendung mit den Endnutzern zu
simulieren und zu testen. Weitere Vorteile sind die Automatisierung aller
Testaktivitäten für die zu testende Anwendung und die Integration von
Benutzerschnittstellentests in den Entwicklungsprozess.


Automatisiertes Testen ist eine der Funktionen, die im Mittelpunkt dieser
Arbeit stehen. Es ist daher wichtig, dieses Konzept zu verstehen, um
weiterzukommen.



\subsection{Nichtfunktionale Tests}

Nichtfunktionales Testen ist das Testen einer Softwareanwendung oder
eines Systems auf seine nichtfunktionalen Anforderungen. Es dient dazu,
die Bereitschaft eines Systems in Bezug auf nichtfunktionale Parameter zu
testen, die bei funktionalen Tests nicht berücksichtigt werden.
Nichtfunktionale Tests beziehen sich auf verschiedene Aspekte der Software
wie Leistung, Belastung, Stress, Skalierbarkeit, Sicherheit, Kompatibilität.
Der Schwerpunkt liegt auf der Verbesserung der Benutzererfahrung.


Es gibt über 100 Arten von Nichtfunktionalen Tests, von denen einige in den
folgenden Abschnitten behandelt werden:

\textbf{Performancetests}

Nach Vokolos et al. sind Performancetests die Gesamtheit aller Aktivitäten,
die an der Bewertung der Leistung einer Software in \Gls{prod}
beteiligt sind (\cite{vokolos1998performance}, S. 80).
Diese Leistung wird aus der Sicht des Benutzers bewertet und in der
Regel in Form von Durchsatz, Reaktionszeit auf Stimuli oder einer
Kombination aus beiden bewertet. Performancetests können auch verwendet
werden, um den Grad der Verfügbarkeit bzw stabilität der Software zu
bewerten. Für viele Anwendungen in der Telekommunikation oder im
medizinischen Bereich ist es beispielsweise von entscheidender Bedeutung,
dass das System immer verfügbar ist.


Performancetests sind entscheidend, um festzustellen, ob eine Anwendung die
Leistungsanforderungen erfüllt (z.B.\ muss das System bis zu 1 000
gleichzeitige Benutzer verwalten können). Sie helfen auch dabei, Engpässe
(\Gls{bottleneck}) in einer Anwendung zu lokalisieren. Sie werden auch
für den Vergleich von zwei oder mehr Systemen verwendet, um das
leistungsfähigste zu identifizieren (z. B. \Gls{jexam_2009}  und \Gls{jexam_new}).
Es ermöglicht auch eine effektive Messung der Stabilität in Bezug auf den
Datenverkehr. Performancetests unterteilen sich in weitere Unterkategorien:


\textbf{Lasttests}: Sie sind eine Art von Performancetests,
bei denen die Anwendung auf ihre Leistung bei normaler und Spitzenbelastung
getestet wird. Die Leistung einer Anwendung wird im Hinblick auf ihre Reaktion
auf Benutzeranfragen und ihre Fähigkeit, innerhalb einer akzeptierten Toleranz
bei unterschiedlichen Benutzerlasten konsistent zu reagieren, überprüft.


\textbf{Stresstests}: Sie werden eingesetzt, um Wege zu finden, das System zu
brechen. Der Test liefert auch den Bereich der maximalen Belastung,
die das System aushalten kann. In der Regel wird beim Stresstest ein
schrittweiser Ansatz verfolgt, bei dem die Belastung schrittweise
erhöht wird. Der Test wird mit einer Last begonnen, für die die Anwendung
bereits getestet wurde. Dann wird die Last langsam erhöht, um das System
zu belasten. Der Punkt, an dem wir feststellen, dass die Server nicht mehr
auf die Anfragen reagieren, wird als Sollbruchstelle betrachtet.


Dies sind die einzigen beiden Unterkategorien von Performancetests,
die in dieser Arbeit verwendet werden.



\textbf{Penetrationtests}

Ein Penetrationstest ist ein Versuch, die Sicherheit einer IT-Infrastruktur
zu bewerten, indem auf sichere Weise versucht wird, Schwachstellen
auszunutzen. Diese Schwachstellen können in Betriebssystemen, Diensten
und Anwendungsfehlern, unsachgemäßen Konfigurationen oder riskantem
Verhalten der Endbenutzer bestehen. Solche Bewertungen sind auch nützlich,
um die Wirksamkeit von Verteidigungsmechanismen und die Einhaltung von
Sicherheitsrichtlinien durch die Endbenutzer zu überprüfen.


Penetrationstests werden in der Regel mit manuellen oder automatisierten
Technologien durchgeführt, um systematisch Server, Endpunkte, Webanwendungen,
drahtlose Netzwerke, Netzwerkgeräte, mobile Geräte und andere potenzielle
Angriffspunkte zu kompromittieren. Sobald die Schwachstellen in einem
bestimmten System erfolgreich ausgenutzt wurden, können die Tester versuchen,
das kompromittierte System für weitere Angriffe auf andere interne Ressourcen
zu nutzen. Insbesondere indem sie versuchen, schrittweise höhere
Sicherheitsstufen und einen tieferen Zugang zu elektronischen Ressourcen
und Informationen über die Ausweitung von Berechtigungen zu erreichen.


Laut Brad Arkin et al. sind Penetrationstests die am häufigsten und am
weitesten verbreiteten Best Practices im Bereich der Softwaresicherheit,
was zum Teil daran liegt, dass es sich um eine attraktive Aktivität am
Ende des Lebenszyklus handelt (vgl. \citte{1392709}, S.84).  Sobald eine
Anwendung fertiggestellt ist, sollen die Tester sie im Rahmen der Endabnahme
Penetrationstests unterziehen. Der Hauptzweck von Penetrationstests besteht
also darin, Schwachstellen in einem System zu identifizieren, um sie zu
schließen. Dies ermöglicht es den Testern, die Sicherheit der Anwendung
zu erhöhen und die Sicherheitsstrategie zu verbessern.


So wie funktionale Tests zeigen, wie gut eine Anwendung funktioniert,
so sind auch nicht-funktionale Tests für die Benutzererfahrung von
großer Bedeutung.





