\chapter{Grundlagen}\label{ch:grundlagen}

In diesem Kapitel werden die notwendigen Definitionen und Methoden erläutert,
die für das Verständnis der Arbeit von Bedeutung sind. Zunächst werden die
Ziele und Grenzen des Softwaretestens definiert. Dann folgt eine kurze
Einführung in den Begriff der Webanwendung und  die Erklärung des
Konzeptes der Testumgebung. Das Konzept des Testens im Bereich der
Softwareentwicklung ist sehr weit gefasst und kann in dieser Arbeit
nicht vollständig behandelt werden. Aus diesem Grund werden die
verschiedenen Arten von Tests vorgestellt, die in dieser Arbeit
nützlich sein werden.  Um herauszufinden, wie guten Testfälle
geschrieben werden, erlaütert der fünfte Teil die Merkmale einer
guten \Gls{TestSuite}.  Schließlich werden zwei wichtigste
Testmethoden vorgestellt.

\section{Ziele und Grenzen des Softwaretestens}

Die Standarddefinition des Testens nach dem
ANSI/IEEE 1059-Standard besagt, dass Testen der
Prozess der Analyse eines Softwareobjekts ist, um
Unterschiede zwischen bestehenden und erforderlichen
Bedingungen (d.h. Defekte/Fehler/Bugs) zu erkennen
und die Eigenschaften des Softwareobjekts zu bewerten (vgl. \cite{singh2012software}, S.07).
Das Softwaretesten ist daher eine Methode, um zu überprüfen,
ob das Softwareprodukt den erwarteten
Anforderungen entspricht und um sicherzustellen, dass
es frei von Fehlern ist.

Im 21. Jahrhundert ist der Einsatz von Software und
Anwendungen weit verbreitet und kein Bereich bleibt
davon verschont. Die Gesamtmenge der weltweit erstellten,
erfassten, kopierten und verbrauchten Daten ist laut
Statista (vgl. \cite{Statista2021}) bis 2020 rasant auf 64,2
Zettabyte \begin{math}(2^{70})\end{math} angestiegen. Die Nutzer vertrauen ihre sensiblen und privaten Daten
Plattformen an, deren Aufgabe ist es, sie zu schützen. Testen
ist wichtig, weil Softwarefehler teuer oder sogar gefährlich
sein können. Sie können finanzielle
und menschliche Verluste verursachen, und die Geschichte
ist voll von solchen Beispielen:

\noindent
\begin{enumerate}
    \item Im Mai 1996 führte ein Softwarefehler dazu, dass
     den Konten von 823 Kunden einer großen US-Bank 920
     Millionen US-Dollar gutgeschrieben wurden (vgl. \cite{Devi2015}).
    \item China Airlines Airbus A300 crashed due to a software bug on April 26,
     1994, killing 264 innocents live (vgl. \cite{Takeuch1996}).
\end{enumerate}

Das Testen einer Anwendung hat viele Vorteile. Zu den wichtigsten
gehören die folgenden:


 \textbf{Kosteneffektivität}: Das
rechtzeitige Testen eines IT-Projekts hilft auf
lange Sicht Geld zu sparen. Wenn die Fehler bereits in
der frühen Phase des Softwaretests entdeckt werden,
kostet es weniger, sie zu beheben. Es ist besser, mit
dem Testen früher zu beginnen und es in jeder Phase des
Lebenszyklus der Softwareentwicklung einzuführen.
Regelmäßige Tests sind erforderlich (vgl. \cite{kumar2010software}, S.53), um
sicherzustellen, dass die Anwendung gemäß den Anforderungen entwickelt wird.

\begin{figure}[H]
    \centering
    \includegraphics[scale=0.5]{images/Cost-of-fixing-bugs-in-different-phases}
    \caption{Kosten für die Behebung von Fehlern (Bugs) in verschiedenen Phasen (vgl. \cite{kumar2010software})} \label{fig:mof}
\end{figure}


\textbf{Erhöhung der Sicherheit}: Sicherheit ist der anfälligste und
sensibelste Teil des Softwaretestens. Durch Testen wird sichergestellt,
dass die Anwendung über einen minimalen Schutz verfügt. Testen hilft
dabei, Risiken und Probleme früher zu beseitigen. So wird die Anwendung für Nutzer attraktiv,
die vertrauenswürdige Produkte suchen (vgl. \cite{shultz2011software}, S.09).

\textbf{Produktqualität}: Sie ist eine wesentliche
Voraussetzung für jedes Softwareprodukt. Zu den sechs Gruppen von Software-Qualitätsindikatoren
in der ISO-Norm 9126 (vgl. \cite{AlainAbran2010}) gehört die Wartbarkeit, zu der auch die Untergruppe Testbarkeit gehört.
Durch Testen kann die Qualität einer Anwendung sowie ihre Wartbarkeit erhöht werden und so wird dem Kunden sichergestellt,
dass ein Qualitätsprodukt geliefert wird.


Aus diesen Gründen ist das Testen von Software ein
integraler Bestandteil des Softwareentwicklungsprozesses, jedoch hat es Grenzen.
Testen dient nur dazu, das Vorhandensein von potentiellen Fehlern
aufzudecken. Aber es kann nicht sicherstellen, dass
die Software keine Fehler oder Bugs enthält (vgl. \cite{kumar2010software}, S.55).
Dazu können Tests nicht nachweisen, dass ein Produkt unter allen
Bedingungen richtig funktioniert, sondern nur, dass es unter
bestimmten Bedingungen nicht richtig funktioniert (vgl. \cite{kumar2010software}, S.56).



Da das Ziel dieser Arbeit darin besteht, eine \Gls{TestSuite} für
eine Webanwendung einzurichten, ist es wichtig zu
definieren, was mit Webanwendung eigentlich gemeint ist.

\section{Webanwendungen}

Das weite Feld der Softwareentwicklung umfasst auch die
Entwicklung von Webanwendungen. Lange Zeit wurden
Anwendungen als kompakte, installierbare Programme
verkauft. Dabei handelt es sich um die so genannten
klassischen Anwendungen oder Computeranwendungen, die
lokal auf einem Computer, Mobiltelefon oder Tablet
installiert werden müssen. Im Gegensatz zu herkömmlichen
Anwendungen werden Webanwendungen nicht lokal auf dem Gerät
des Nutzers installiert, sondern auf einem Server, so dass
sie über eine bestimmte \acs{url} zugänglich sind.

Nach Kappel et al. ist eine Webanwendung ein Softwaresystem,
das auf den Spezifikationen des World Wide Web Consortiums
(\acs{w3c}) basiert und Webressourcen bereitstellt, die über
eine Benutzeroberfläche wie einen Webbrowser genutzt werden
können (vgl. \cite{kappel1}, S.02).

Für den Betrieb einer Webanwendung werden mehrere
Computersystemkomponenten benötigt.Erstens das \gls{frontend},
das die grafische Oberfläche bezeichnet, mit der der
Benutzer interagiert, gefolgt vom \gls{backend}, das die
Logik der Anwendung enthält (z.B Datenbanken, Cloud).


Der Benutzer greift auf die Webanwendung über einen Computer
zu, der als Client bezeichnet wird. Der Client sendet eine
oder mehrere Anfragen über das Internet oder Intranet via
\acs{http} oder \acs{https}  an einen anderen Computer (Server),
auf dem die Webanwendung läuft. Der Server nimmt dann die
HTTP-Anfragen entgegen und verarbeitet sie. Je nach Anfrage
werden die angeforderten Daten entweder aus der Datenbank
abgerufen oder gespeichert. Die verarbeiteten Daten werden
dann vom Server in einer entsprechenden Antwort
(HTTP-Response) an den Client zurückgesendet und im
Webbrowser angezeigt. So funktionieren Webanwendungen grundsätzlich.

Um Webanwendungen zu testen, ist es wichtig, eine Testumgebung zu schaffen.
Diese Testumgebung ermöglicht es dem Tester, alle möglichen Schwachstellen in einer Webanwendung zu untersuchen,
ohne die Anwendung zu gefährden, wenn sie bereits in Produktion ist.


\section{Testumgebung}

Im Moment ist die alte Version von jexam (\gls{jexam_2009})
in \gls{prod}, weil sie stabil und funktional ist.
Die Entwickler des Teams sorgen dafür, dass die
Funktionen der Anwendung ordnungsgemäß funktionieren.
Dies ist bereits eine Art manueller Test. Die neue
Version, die derzeit entwickelt wird, wurde noch nicht
vollständig getestet. Daher ist sie von unbekannter
Qualität und kann nicht als stabil für die \gls{prod}
gelten. Sie sollte nicht in die \gls{prod} aufgenommen
werden, damit die Benutzer sie testen können.  Wenn
die Entwickler dies täten, würden sie Schwachstellen
aufdecken, die von böswilligen Personen genutzt werden
könnten, um die Sicherheit der Nutzerdaten zu
gefährden. Bevor eine Anwendung in \gls{prod} geht,
sollte sie getestet werden. Da dies nicht in der
\gls{prod} geschehen sollte, ist es notwendig, eine
Umgebung zu schaffen, in der die Anwendung getestet
werden kann.


Nach Everett eine Testumgebung ist eine Umgebung,
die Hardware, Instrumente, Simulatoren, Software-Tools
und andere unterstützende Elemente enthält, die für
die Durchführung eines Tests erforderlich sind,
d.h. sie ermöglicht es, die Anwendung zu testen, ohne
den Kunden zu beeinträchtigen (vgl. \cite{shultz2011software}, S. 150-152).
Die Einrichtung einer Testumgebung hat eine Reihe von Vorteilen :

\noindent
\begin{enumerate}
    \item Die Nutzer nicht mit einer Anwendung von
    zweifelhafter Qualität zu belästigen. Ein
    Qualitätsabfall ist oft sehr nachteilig für eine
    Anwendung. Die Nutzer könnten sich beschweren,
    dass die alte Version besser ist und die ganze
    Arbeit der Entwickler, die die neue Version
    entwickelt haben, preisgeben.

    \item Die Einbeziehung aller Szenarien in die
    Tests, was die Zuverlässigkeit der Software
    erhöht. Sie hilft auch dabei, fehlende
    Implementierungen der Software zu finden. In der
    Produktion wäre dies eine komplizierte Aufgabe, da
    sich die Daten und die Anwendung ständig ändern.

    \item Die Senkung der Produktionskosten durch
    frühzeitige Erkennung von Fehlern.

\end{enumerate}

Testumgebungen sind für jede Softwareentwicklung
notwendig. Sie ermöglichen es, das Produkt zu testen,
bevor es auf den Markt kommt, ohne ein Risiko für die
Produktion einzugehen. Wenn sie jedoch wirklich
wirksam sein sollen, dürfen sie die Entwicklung nicht
behindern, sondern müssen auch zuverlässig sein. Aus
diesem Grund muss diesen Umgebungen besondere
Aufmerksamkeit geschenkt werden. Die endgültige
Qualität des Produkts hängt stark von der Qualität der
Testumgebung ab (vgl. \cite{shultz2011software}, S. 152).


