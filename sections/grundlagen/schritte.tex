\section{Verschiedene Schritte zur Erstellung einer Testsuite}\label{sec:verschiedene-schritte-zur-erstellung-einer-testsuite}

In der heutigen wettbewerbsorientierten und schnelllebigen
Welt ist das Internet zu einem integralen Bestandteil des Lebens
geworden. Heutzutage treffen die meisten ihre Entscheidungen,
indem sie im Internet nach Informationen suchen. Das Hosten einer
Website ist daher nicht mehr optional, sondern f\"ur alle Arten von
Unternehmen obligatorisch. Es ist der erste Schritt, um auf dem
Markt relevant zu werden und zu bleiben. Es reicht nicht mehr aus,
nur eine Website zu haben. Ein Unternehmen muss eine Website
entwickeln, die informativ, zug\"anglich und benutzerfreundlich ist.
Um all diese Qualit\"aten zu erhalten, sollte die Website getestet
werden, und dieser Prozess des Testens einer Website ist als Web-Testing
bekannt. Um diese Aufgabe zu bew\"altigen,
durchlaufen die Testerteams mehrere Schritte:

\textbf{Erstellung eines Testplan:} Ein Testplan ist ein Dokument, das den
Umfang, die Vorgehensweise und den Zeitplan der geplanten Testaktivit\"aten
festlegt (vgl. \cite{shultz2011software}, S. 79). Das Hauptziel eines Testplans
ist die Erstellung einer Dokumentation, die beschreibt, wie der Tester
\"uberpr\"uft, ob das System wie vorgesehen funktioniert. Das Dokument sollte
beschreiben, was getestet werden muss, wie es getestet werden soll und wer
daf\"ur verantwortlich ist. Er ist die Grundlage f\"ur jede Testarbeit
(vgl. \cite{shultz2011software}, S. 79). Der Testplan enth\"alt in der Regel die
folgenden Informationen:

\begin{enumerate}
    \item Das Gesamtziel des Testaufwands.
    \item Einen detaillierten \"Uberblick \"uber die Art und Weise, wie die Tests
    durchgef\"uhrt werden.
    \item Die zu pr\"ufenden Funktionen, Anwendungen oder Komponenten.
    \item Detaillierte Zeit- und Ressourcenzuweisungspl\"ane f\"ur Tester und
    Entwickler w\"ahrend aller Testphasen.
\end{enumerate}

\textbf{Erstellung einer Teststrategie :} Die Teststrategie beim Softwaretesten
ist definiert als eine Reihe von Leitprinzipien, die das Testdesign
bestimmen und regeln, wie der Softwaretestprozess durchgef\"uhrt wird. Das Ziel
der Teststrategie ist es, einen systematischen Ansatz f\"ur den
Softwaretestprozess zu liefern, um die Qualit\"at, R\"uckverfolgbarkeit,
Zuverl\"assigkeit und bessere Planung zu gew\"ahrleisten. Die
Teststrategie wird sehr oft mit dem Testplan verwechselt. Es gibt jedoch einen
bemerkenswerten Unterschied zwischen den beiden. Der Testplan konzentriert
sich auf die Organisation und das Management der Ressourcen sowie die
Definition der Ziele, w\"ahrend die Teststrategie sich auf die Umsetzung und die
Testtechniken konzentriert. Beide helfen bei der Planung, aber auf
unterschiedlicher Ebene.

\textbf{Testentwurf und implementierung :} Der Testentwurf ist ein Prozess,
der beschreibt, ``wie'' die Tests durchgef\"uhrt werden sollen und das
Schreiben von Testsuiten f\"ur das Testen einer Software. Er umfasst Prozesse
zur Identifizierung von Testf\"allen durch Aufz\"ahlung von Schritten der
definierten Testbedingungen. Die in der Teststrategie oder dem Testplan
definierten Testtechniken werden f\"ur die Aufz\"ahlung
der Schritte verwendet (vgl. \cite{shultz2011software}, S. 46).



Je nach den Zielen, die im Testplan festgelegt wurden, konzentrieren sich die
Entwickler auf bestimmte Arten von Tests, z. B. funktionale Tests,
Performancetests oder Sicherheitstests. F\"ur diese Arbeit wurde vorab ein
Testplan erstellt. Aus diesem Grund wird der Schwerpunkt auf der Implementierung
von Tests und der Einrichtung der Testinfrastruktur liegen.

