\section{Testumgebung}

Im Moment ist die alte Version von jexam (\gls{jexam_2009})
in \gls{prod}, weil sie stabil und funktional ist.
Die Entwickler des Teams sorgen dafür, dass die
Funktionen der Anwendung ordnungsgemäß funktionieren.
Dies ist bereits eine Art manueller Test. Die neue
Version, die derzeit entwickelt wird, wurde noch nicht
vollständig getestet. Daher ist sie von unbekannter
Qualität und kann nicht als stabil für die \gls{prod}
gelten. Sie sollte nicht in die \gls{prod} aufgenommen
werden, damit die Benutzer sie testen können.  Wenn
die Entwickler dies täten, würden sie Schwachstellen
aufdecken, die von böswilligen Personen genutzt werden
könnten, um die Sicherheit der Nutzerdaten zu
gefährden. Bevor eine Anwendung in \gls{prod} geht,
sollte sie getestet werden. Da dies nicht in der
\gls{prod} geschehen sollte, ist es notwendig, eine
Umgebung zu schaffen, in der die Anwendung getestet
werden kann.


Nach Everett eine Testumgebung ist eine Umgebung,
die Hardware, Instrumente, Simulatoren, Software-Tools
und andere unterstützende Elemente enthält, die für
die Durchführung eines Tests erforderlich sind,
d.h. sie ermöglicht es, die Anwendung zu testen, ohne
den Kunden zu beeinträchtigen (vgl. \cite{shultz2011software}, S. 150-152).
Die Einrichtung einer Testumgebung hat eine Reihe von Vorteilen :

\noindent
\begin{enumerate}
    \item Die Nutzer nicht mit einer Anwendung von
    zweifelhafter Qualität zu belästigen. Ein
    Qualitätsabfall ist oft sehr nachteilig für eine
    Anwendung. Die Nutzer könnten sich beschweren,
    dass die alte Version besser ist und die ganze
    Arbeit der Entwickler, die die neue Version
    entwickelt haben, preisgeben.

    \item Die Einbeziehung aller Szenarien in die
    Tests, was die Zuverlässigkeit der Software
    erhöht. Sie hilft auch dabei, fehlende
    Implementierungen der Software zu finden. In der
    Produktion wäre dies eine komplizierte Aufgabe, da
    sich die Daten und die Anwendung ständig ändern.

    \item Die Senkung der Produktionskosten durch
    frühzeitige Erkennung von Fehlern.

\end{enumerate}

Testumgebungen sind für jede Softwareentwicklung
notwendig. Sie ermöglichen es, das Produkt zu testen,
bevor es auf den Markt kommt, ohne ein Risiko für die
Produktion einzugehen. Wenn sie jedoch wirklich
wirksam sein sollen, dürfen sie die Entwicklung nicht
behindern, sondern müssen auch zuverlässig sein. Aus
diesem Grund muss diesen Umgebungen besondere
Aufmerksamkeit geschenkt werden. Die endgültige
Qualität des Produkts hängt stark von der Qualität der
Testumgebung ab (vgl. \cite{shultz2011software}, S. 152).

