\chapter{Problemanalyse und Modellierung}\label{ch:problemanalyse-und-modellierung}

\blindtext
\todo[inline]{Write some more}

\begin{lstlisting}[language=AST,label={lst:example-ast},caption={Example AST}]
RailwayContainer ::= Route* Region*;
abstract RailwayElement ::= <Id:int>;
Region : RailwayElement ::= TrackElement* Sensor*;
Semaphore : RailwayElement ::= <Signal:Signal>;
Route : RailwayElement ::= <Active:boolean> SwitchPosition*;
SwitchPosition : RailwayElement ::= <Position:Position>;
Sensor : RailwayElement;
abstract TrackElement:RailwayElement;
Segment : TrackElement ::= <Length:int> Semaphore*;
Switch : TrackElement ::= <CurrentPosition:Position>;
\end{lstlisting}
%
Das Listing~\ref{lst:example-ast} zeigt eine beispielhafte Grammatik, welche im Attribute im folgenden Listing genutzt wird:

\lstinputlisting[language=JRAG,style=unboxed]{code/requiredSensor.jrag}
