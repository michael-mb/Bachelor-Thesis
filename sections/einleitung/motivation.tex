\section{Motivation}

Die meisten Entwickler finden Testen langweilig und denken, dass Testen den
Entwicklungsprozess verlangsamt. Nach Murugesan  wird es oft nur zweitrangig
berücksichtigt, insbesondere in den letzten und entscheidenden Phasen des
Softwareentwicklungsprozesses (vgl. \cite{526394}, S.112). Jedoch ist diese
Herangehensweise fehlerbehaftet und kostspielig. “Testing is not a PIT, it is a
LADDER ...! “ (Anand vgl. \cite{anand12importance} , S.02). Nach dieser Auffassung von
Anand geht es beim Testen vielmehr darum, Fehler und Situationen zu vermeiden,
die den Entwicklungsprozess einer Anwendung verlangsamen könnten.
Dadurch wird die Qualität der Software erhöht und die zeitlichen Vorgaben
werden eingehalten. Dies reduziert die Produktionskosten.


An der Technischen Universität Dresden steht den Studierenden der Informatik eine
Plattform für die Anmeldung zu Lehrveranstaltungen und Prüfungen sowie für die
Bekanntgabe der Prüfungsergebnisse zur Verfügung. Es handelt sich um ein
studentisches Projekt, das im Laufe der Zeit mit relativ wenig Personal
und einer hohen Personalfluktuation entwickelt wurde.  Das erschwert die
Wartung der Plattform und hat das Entwicklerteam dazu gezwungen , sich auf
Entwicklungs- und Wartungsaufgaben zu konzentrieren. Aus diesem Grund liegt
die Plattform im Moment ohne automatische Tests vor.


Die Entwicklung der Plattform in ihrer jetzigen Form basiert auf Technologien aus
dem Jahr 2009 und früher. Aufgrund der zwischenzeitlichen Entwicklung dieser
Kerntechnologien sowie der potenziellen Sicherheitsrisiken, die in den älteren
Versionen vorhanden sein könnten, besteht ein Bedarf an Weiterentwicklung.
Diese Arbeit wird derzeit von einer studentischen Hilfskraft aus der jExam-Gruppe
durchgeführt.


Jexam ist jedoch eine kritische Plattform für Studenten, zum einen, weil sie
vertrauliche Informationen über Studenten und ihr Studium enthält, und zum
anderen, weil sie zu Beginn des Semesters für die Wahl der Fächer und am Ende
des Semesters für die Anmeldung zu Prüfungen unerlässlich ist. Um möglichen
Fehlern vorzubeugen und eine gute Wartung der Plattform zu gewährleisten, ist
es notwendig, zu testen.


