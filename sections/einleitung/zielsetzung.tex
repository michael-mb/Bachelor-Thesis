
\section{Zielsetzung und Abgrenzung}


Wie Azeem Uddin  sagte: ``Testen bedeutet herauszufinden, wie gut etwas funktioniert'' (vgl. \cite{anand12importance}, S.02).
Die aktuelle Version von jExam (die nun als \textbf{\gls{jexam_2009}} beschrieben wird) und die neue Version (\textbf{\gls{jexam_new}}),
die derzeit entwickelt wird, laufen ohne Tests. Das Testen von \gls{jexam_2009} wird dazu beitragen, die Plattform
besser zu warten, die Sicherheit zu erhöhen, die Qualität der Anwendungen zu verbessern und mögliche Fehler zu
vermeiden (vgl. \cite{shultz2011software}, S.21), während auf die Inbetriebnahme von \gls{jexam_new} noch aussteht.


Beide Versionen werden genau die gleichen Funktionalitäten haben, da die neue Version (\gls{jexam_new}) nur eine Kopie der
alten Version mit modernen Technologien ist. So kann eine Testinfrastruktur geschaffen werden, die mit beiden Versionen
kompatibel ist. Damit besteht die Möglichkeit zu testen, ob \gls{jexam_new} genau die gleichen Funktionalit\"aten wie
\gls{jexam_2009} hat und gleichzeitig alle Vorteile einer
getesteten Webanwendung zu haben.

Wegen des Mangels an Arbeitskräften im Entwicklungsteam ist es notwendig,
eine Infrastruktur zu schaffen, um das Schreiben zu erleichtern und die
Entwicklungszeit der Tests zu beschleunigen. Diese Arbeit zielt darauf ab,
das jExam-Entwicklungsteam bei der Wartung und schnellen Funktionsprüfung der
Plattform zu unterstützen. Dafür müssen die folgenden Ziele umgesetzt werden:


\begin{enumerate}
    \item Entwicklung einer Testsuite für die Neuentwicklung von jExam-Web.
    Dabei sollten mindestens folgende Funktionen abgedeckt sein:
    \begin{enumerate}
        \item Login
        \item Registrierung
        \item Abruf der Noten in der \"Ubersicht und als pdf
        \item Einschreibung in Pr\"ufungen
        \item Einschreibung in Lehrveranstaltungen
        \item Einschreibung in Seminargruppen
    \end{enumerate}
    \item Hohe Wartbarkeit der Testsuite
    \item Bereitstellung einer automatisierten Infrastruktur
    \item Erweiterung der Testsuite um Sicherheitstests
    \item Erweiterung der Testsuite um Performancetests
    \item Dokumentation
\end{enumerate}