\section{Aufbau der Arbeit}


Diese Arbeit ist in fünf Teile gegliedert. Zunächst wird das zum Verständnis der
Arbeit erforderliche Grundwissen erläutert. Darin werden verschiedene Konzepte
beschrieben, darunter Webanwendungen und die Grundlagen des Softwaretestens,
in denen grundlegende Fragen definiert werden. Außerdem erfolgt die Vorstellung
von verschiedenen Testarten und -methoden.


Dann folgt Kapitel drei, das der Beschreibung der zu testenden Plattformen
gewidmet ist, gefolgt von einer Phase der Analyse der Probleme, die die
Entwicklung von Tests erschweren und sogar die Verwendung bestimmter Werkzeuge
ausschließen können.  Am Ende desselben Kapitels werden verschiedene
Lösungsvorschläge erörtert.


Als nächstes kommt in Kapitel vier  der praktische Teil, in dem die verschiedenen
Technologien und Werkzeuge beschrieben werden, die während der Implementierung
verwendet wurden, wobei die Gründe für ihre Verwendung dargelegt werden,
gefolgt von einer detaillierten Beschreibung des Entwicklungsprozesses und
der verwendeten Architektur (Entwurfsmuster, Softwarearchitektur \ldots)


Die Ergebnisse des entwickelten Tools werden in Kapitel fünf vorgestellt und im
Hinblick auf verschiedene Aspekte wie Leistung und Fehler evaluiert. Schließlich
werden in Kapitel sechs Perspektiven für die Erweiterung der vorgestellten
Arbeit sowie die vollständige Dokumentation des Tools vorgestellt.

