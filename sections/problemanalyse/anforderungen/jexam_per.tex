\subsection{Jexam Performancestests}

\begin{center}
`` 2 Sekunden ist die Schwelle f\"ur die Akzeptanz einer E-Commerce-Website.
Bei Google streben wir eine Zeit unter einer halben Sekunde an.'' Maile Ohye
\end{center}
\>

Das obige Zitat stammt aus einem Video, das 2010 von Google Webmasters
ver\"offentlicht wurde (vgl. \cite{Ohye2010}). Wenn eine Seite l\"anger
als 2 Sekunden zum Laden braucht, verliert sie Pl\"atze in ihrer Google
\acs{seo}-Position. Es sollte sichergestellt werden, dass eine
Webanwendung dieses Kriterium erfüllt, daher ist es wichtig,
die Leistung zu testen. Die Performance einer Webanwendung spielt eine
entscheidende Rolle f\"ur die Benutzererfahrung. Jexam hat Dienste,
die immer in Echtzeit verf\"ugbar sein m\"ussen. Mithilfe von 
Performancetests kann die Anwendung unter bestimmten Bedingungen 
getestet werden, um sie zu bewerten. Die beiden Versionen sollten 
daher getestet und verglichen werden, um eine m\"ogliche Leistungssteigerung
zu untersuchen. Dies wird es auch erm\"oglichen, nach einer \"anderung 
zu beobachten, ob es einen Leistungsr\"uckgang oder Bottlenecks gegeben
hat.  Aus denselben Gr\"unden wie die Sicherheitstests sollten auch die 
Performancetests in Form von Skripten geschrieben und vollautomatisch
ausgef\"uhrt werden. 