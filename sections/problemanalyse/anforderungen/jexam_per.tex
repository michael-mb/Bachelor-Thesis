\subsection{JExam Performancestests}

\begin{center}
`` 2 Sekunden ist die Schwelle f\"ur die Akzeptanz einer E-Commerce-Website.
Bei Google streben wir eine Zeit unter einer halben Sekunde an.'' Maile Ohye
\end{center}
\>

Das obige Zitat stammt aus einem Video, das 2010 von Google Webmasters
ver\"offentlicht wurde (vgl. \cite{Ohye2010}). Wenn eine Seite l\"anger
als 2 Sekunden zum Laden braucht, verliert sie Pl\"atze in ihrer Google
\acs{seo}-Position. Es sollte sichergestellt werden, dass eine
Webanwendung dieses Kriterium erfüllt, daher ist es wichtig,
die Leistung zu testen. Die Performance einer Webanwendung spielt eine
entscheidende Rolle f\"ur die Benutzererfahrung. JExam hat Dienste,
die immer in Echtzeit verf\"ugbar sein m\"ussen. Mithilfe von 
Performancetests kann die Anwendung unter bestimmten Bedingungen 
getestet werden, um sie zu bewerten.

Die einzelnen Schritte des Leistungstests sind von Unternehmen
zu Unternehmen und von Anwendung zu Anwendung unterschiedlich.
Sie hängen davon ab, welche Leistungsindikatoren das Unternehmen für
besonders wichtig hält. Die allgemeinen Ziele von Leistungstests
sind jedoch weitgehend identisch, so dass es einen bestimmten
Arbeitsablauf gibt, dem die meisten Testpläne folgen. Daher ist es
wichtig, die Einschränkungen, Ziele und Schwellenwerte festzulegen,
die den Erfolg des Tests belegen. Die wichtigsten Kriterien,
die für die Tests von jExam verwendet werden, sind:

\noindent
\begin{enumerate}
    \item Beim Navigieren auf der Anwendung sollte die Webseite auch in
    weniger als 03 Sekunden geladen werden, unabhängig davon, welche Anfrage
    ein Benutzer ausführt (Load Testing).
    \item Die Seiten von jExam müssen in weniger als 3 Sekunden laden,
    wenn 200 Benutzer gleichzeitig eine Anfrage stellen (Stress Testing).
\end{enumerate}

Die beiden Versionen sollten daher getestet und verglichen werden,
um eine mögliche Leistungssteigerung zu untersuchen.
Dies wird es auch ermöglichen, nach einer Änderung zu beobachten, ob
es einen Leistungsrückgang gegeben oder zum Bottlenecks geführt hat.