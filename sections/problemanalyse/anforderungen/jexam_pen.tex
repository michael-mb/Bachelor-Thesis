\subsection{JExam Sicherheitstests}\label{ch:sicherheit}

Keine Software ist frei von Fehlern und Bugs. Dennoch muss eine qualitativ
hochwertige Anwendung ihren Nutzern Datenschutz garantieren. Das gilt
auch f\"ur jExam. Das Sicherheits- oder Penetrationstesten  ist eine schwierige
Aufgabe. Das Penetrationstesten ist eine haupts\"achlich manuelle Aktivit\"at.

Experten-Penetrationtester sind ein wesentlicher Bestandteil
von Penetrationtests. Zunächst verwenden sie Schwachstellenscanner.
Es handelt sich um automatisierte Tools , die eine Umgebung untersuchen
und nach Abschluss der Untersuchung einen Bericht über die
gefundenen Schwachstellen erstellen. Diese Scanner listen diese
Schwachstellen häufig mithilfe von \Gls{cve} auf, die Informationen
über bekannte Schwachstellen liefern. Da Scanner Tausende von
Schwachstellen entdecken können, kann es sein, dass es
genügend schwerwiegende Schwachstellen gibt, die eine zusätzliche
Priorisierung erforderlich machen.


Für komplexe Tests, bei denen tief in verschiedene Systeme und
Anwendungen eingedrungen werden muss oder Übungen mit mehreren
Angriffsketten durchgeführt werden müssen, wird immer eine
erfahrenere Person oder ein erfahrenes Team benötigt. Um ein
realistisches Angriffsszenario zu testen, ist ein Team erforderlich,
das ausgeklügelte Strategien und Lösungen verwendet, die den
Techniken der Bedrohungsakteure ähneln.


Aufgrund des Ressourcenmangels im jExam-Team muss den Prozess
automatisiert werden.  Daher sollte die Sicherheit nur
oberflächlich mit einem Schwachstellen-Scanner getestet werden.
Auf diese Weise ist es möglich hohe Vulnerabilitäten
herauszufinden, wenn sie vorhanden sind. Schwachstellen-Scans
können also auch beim Vergleich der beiden Versionen von jExam
helfen. Es wird erfahren, ob die neue Version mehr oder weniger
Schwachstellen hat als die alte. Dadurch können mögliche
Sicherheitslücken behoben werden.