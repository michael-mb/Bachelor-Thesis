\subsection{Jexam Penetrationtests}

Keine Software ist frei von Fehlern und Bugs. Dennoch muss eine qualitativ
hochwertige Anwendung ihren Nutzern Datenschutz garantieren. Das gilt
auch f\"ur Jexam. Sicherheits- oder Penetrationstests sind eine schwierige
Aufgabe. Der Penetrationstest ist eine haupts\"achlich manuelle Aktivit\"at,
aber aufgrund des Ressourcenmangels im Jexam-Team muss den Prozess
automatisiert werden. Es ist geplant, Skripte zu schreiben, die die
Jexam-Anwendung im Lichte der derzeit bekannten Web-Schwachstellen
untersuchen. Auf diese Weise ist es m\"oglich hohe Vulnerabilit\"aten
erkennen, wenn sie vorhanden sind.


Penetrationstests k\"onnen also auch beim Vergleich der beiden Versionen von
jexam helfen. Es wird erfahren, ob die neue Version mehr oder weniger
Schwachstellen hat als die alte. Dadurch k\"onnen m\"ogliche
Sicherheitsl\"ucken behoben werden.

