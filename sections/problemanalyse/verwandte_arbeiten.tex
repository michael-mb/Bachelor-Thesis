\section{Verwandte Arbeiten}

In der heutigen wettbewerbsorientierten und schnelllebigen
Welt ist das Internet zu einem integralen Bestandteil des Lebens
geworden. Heutzutage treffen die meisten von uns ihre Entscheidungen,
indem sie im Internet nach Informationen suchen. Das Hosten einer
Website ist daher nicht mehr optional, sondern für alle Arten von
Unternehmen obligatorisch. Es ist der erste Schritt, um auf dem
Markt relevant zu werden und zu bleiben. Es reicht nicht mehr aus,
nur eine Website zu haben. Ein Unternehmen muss eine Website
entwickeln, die informativ, zugänglich und benutzerfreundlich ist.
Um all diese Qualitäten zu erhalten, sollte die Website getestet
werden, und dieser Prozess des Testens einer Website ist als Web-Testing
bekannt (siehe \autoref{ch:grundlagen}). Um diese Aufgabe zu bewältigen,
durchlaufen die Testerteams mehrere Schritte:


\textbf{Erstellung eines Testplan:} Ein Testplan ist ein Dokument, das den
Umfang, die Vorgehensweise und den Zeitplan der geplanten Testaktivitäten
festlegt (vgl. \cite{shultz2011software}, S. 79). Das Hauptziel eines Testplans
ist die Erstellung einer Dokumentation, die beschreibt, wie der Tester
überprüft, ob das System wie vorgesehen funktioniert. Das Dokument sollte
beschreiben, was getestet werden muss, wie es getestet werden soll und wer
dafür verantwortlich ist. Er ist die Grundlage für jede Testarbeit
(vgl. \cite{shultz2011software}, S. 79). Der Testplan enthält in der Regel die
folgenden Informationen:

\begin{enumerate}
    \item Das Gesamtziel des Testaufwands.
    \item Einen detaillierten Überblick über die Art und Weise, wie die Tests
    durchgeführt werden.
    \item Die zu prüfenden Funktionen, Anwendungen oder Komponenten.
    \item Detaillierte Zeit- und Ressourcenzuweisungspläne für Tester und
    Entwickler während aller Testphasen.
\end{enumerate}

\textbf{Erstellung einer Teststrategie :} Die Teststrategie beim Softwaretesten
ist definiert als eine Reihe von Leitprinzipien, die das Testdesign
bestimmen und regeln, wie der Softwaretestprozess durchgeführt wird. Das Ziel
der Teststrategie ist es, einen systematischen Ansatz für den
Softwaretestprozess zu liefern, um die Qualität, Rückverfolgbarkeit,
Zuverlässigkeit und bessere Planung zu gewährleisten. Die
Teststrategie wird sehr oft mit dem Testplan verwechselt. Es gibt jedoch einen
bemerkenswerten Unterschied zwischen den beiden. Der Testplan konzentriert
sich auf die Organisation und das Management der Ressourcen sowie die
Definition der Ziele, während die Teststrategie sich auf die Umsetzung und die
Testtechniken konzentriert. Beide helfen bei der Planung, aber auf
unterschiedlicher Ebene.

\textbf{Testentwurf und implementierung :} Der Testentwurf ist ein Prozess,
der beschreibt, "wie" die Tests durchgeführt werden sollen und das
Schreiben von Testsuiten für das Testen einer Software. Er umfasst Prozesse
zur Identifizierung von Testfällen durch Aufzählung von Schritten der
definierten Testbedingungen. Die in der Teststrategie oder dem Testplan
definierten Testtechniken werden für die Aufzählung
der Schritte verwendet (vgl. \cite{shultz2011software}, S. 46).



Je nach den Zielen, die im Testplan festgelegt wurden, konzentrieren sich die
Entwickler auf bestimmte Arten von Tests, z. B. Funktionale Tests,
Performancetests oder Sicherheitstests. Für diese Arbeit wurde vorab ein
Testplan erstellt. Aus diesem Grund wird der Schwerpunkt auf der Implementierung
von Tests und der Einrichtung der Testinfrastruktur liegen.


