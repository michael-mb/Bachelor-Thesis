\section{Beschreibung von jExam}

JExam ist eine in der Programmiersprache Java entwickelte Software,
die seit dem Jahr 2000 an der Fakult\"at Informatik der Technischen
Universit\"at Dresden für die Anmeldung zu Lehrveranstaltungen und
Pr\"ufungen sowie für die Mitteilung von Pr\"ufungsergebnissen eingesetzt
wird. Die Entwicklung der Weboberfläche in der jetzigen Form basiert
auf Technologien, welche auf das Jahr 2009 und Frühere zur\"uckgehen.
Seit 2009 haben sich die Programmiersprachen und Technologien jedoch
stetig weiterentwickelt. Allein bei Java hat sich die Entwicklung von
Java SE 6, das 2009 verwendet wurde, zu Java SE 11, das heute verwendet
wird, fortgesetzt. Die Anwendung muss auch mit neuen Versionen von
Programmiersprachen und Frameworks aktualisiert werden, und das aus
mehreren Gründen:


\textbf{Schwachstellen verringern:} Das ist der Hauptgrund für ein
Software-Update. Wenn eine Plattform dem Internet ausgesetzt ist,
sind die Datenbanken, in denen alle Details der Nutzer gespeichert
sind, zunehmend Sicherheitsbedrohungen ausgesetzt. Böswillige
Personen, die über immer bessere Werkzeuge verfügen, finden
Schwachstellen in der Software.  Durch Software-Updates werden einige
dieser Sicherheitslücken überdeckt, sodass sie nicht ausgenutzt werden
können (vgl \cite{10.1145/605466.605479}, S.82).

\textbf{Behebung von Fehlern und Abstürzen:} Ausfälle, Probleme, Fehler
werden behoben, wenn ein Unternehmen eine aktualisierte Version eines
Programms erstellt. Jede entwickelte Software hat inhärente Fehler
oder Verbesserungsmöglichkeiten. Wenn Hersteller Sicherheitslücken
oder Bugs entdecken, kleinere Verbesserungen an Programmen vornehmen
oder Kompatibilitätsprobleme beheben, veröffentlichen sie Updates.
Durch die Aktualisierung wird sichergestellt, dass die neueste und
stabilste Version weniger Fehler aufweist (vgl \cite{10.1145/605466.605479}, S.82).

\textbf{Gewährleistung der Kompatibilität mit anderen aktualisierten
Technologien:} Anwendungen funktionieren heute nicht mehr völlig unabhängig
voneinander, sondern kommunizieren miteinander. Daher ist es wichtig, dass eine
Plattform auf dem neuesten Stand ist, um kompatibel zu sein und die
Vorteile anderer Bibliotheken, Frameworks oder Tools nutzen zu können.


\textbf{Gewinn an Leistung und Funktionalität:} Die Aktualisierung einer
Programmiersprache oder einer Software ist oft mit einem
Leistungszuwachs verbunden.

Aus diesen Gründen wird derzeit eine neue Version von jExam (\Gls{jexam_new})
mit neueren Technologien entwickelt. Um herauszufinden, ob die neue
Plattform funktional und effizienter als die alte Version (\Gls{jexam_2009})
ist, muss eine Testinfrastruktur eingerichtet werden. Dies erfordert
nicht nur das Testen der beiden Plattformen, sondern auch einen Vergleich
zwischen ihnen.


JExam ist eine kritische Plattform für Studierende der TU Dresden.
Sie speichert viele sensible persönliche Informationen über die
Nutzer. Außerdem muss sie immer online und schnell verfügbar sein,
um eine gute Softwarequalität zu gewährleisten. Aufgrund der vielen
Vorteile, die das Testen einer Plattform mit sich bringt
(siehe \autoref{ch:grundlagen}), sollte die Plattform mit einer Reihe von
Anforderungen getestet werden. Das ist das Ziel des nächsten Abschnitts.