% the following command is only required if the thesis is written in german
\RequirePackage[ngerman=ngerman-x-latest]{hyphsubst}

% change to english for english theses
\documentclass[ngerman,BCOR=1.5cm,twoside]{tudscrreprt}

\usepackage[T1]{fontenc}
\usepackage[utf8]{inputenc}
\usepackage[ngerman]{babel}
\usepackage{isodate}


\usepackage[
    style=numeric-comp,
    backend=biber,
    url=false,
    doi=false,
    isbn=false,
    hyperref,
]{biblatex}
\addbibresource{bibliography.bib}
\AtEveryBibitem{%
    \clearfield{note}%
}

\usepackage[hidelinks]{hyperref} % makes all links clickable but hides ugly boxes
\usepackage[capitalise,nameinlink,noabbrev]{cleveref} % automatically inserts Fig. X in the text with \cref{..}

\usepackage[colorinlistoftodos,prependcaption,textsize=tiny]{todonotes}

\usepackage{graphicx}
\graphicspath{ {./images/} }

% if you need mathy stuff
\newtheorem{lem}{Lemma}
\crefname{lem}{Lemma}{Lemmas}
\newtheorem{thm}{Theorem}
\crefname{thm}{Theorem}{Theorems}
\newtheorem{defs}{Definition}
\crefname{defs}{Def.}{Defs.}

\usepackage{blindtext}

%\usepackage{tudscrsupervisor} % if you want to copy the sources of the task description into the thesis

\usepackage{csquotes}

\usepackage{caption}
\captionsetup{font=sf,labelfont=bf,labelsep=space}
\usepackage{floatrow}
\usepackage{float}
\floatsetup[table]{font=sf,style=plaintop}
\captionsetup{singlelinecheck=off,format=hang,justification=raggedright}
\DeclareCaptionSubType[alph]{figure}
\DeclareCaptionSubType[alph]{table}
\captionsetup[subfloat]{labelformat=brace,list=off}

\usepackage{booktabs}
\usepackage{array}
\usepackage{tabularx}
\usepackage{tabulary}
\usepackage{tabu}
\usepackage{longtable}

\usepackage{quoting}

\usepackage[babel]{microtype}

\usepackage{xfrac}

\usepackage{enumitem}
\setlist[itemize]{noitemsep}

\usepackage{ellipsis}
\let\ellipsispunctuation\relax


\parskip=10pt
%\renewcommand{\baselinestretch}{1.8}
%\usepackage{setspace}
\usepackage{acronym}
\usepackage[toc]{glossaries}
\usepackage{textcomp}
\loadglsentries{sections/glossaries}
\makeglossaries

\usepackage{listings}
\usepackage{inconsolata}
\usepackage{color}

\definecolor{dkgreen}{rgb}{0,0.6,0}
\definecolor{gray}{rgb}{0.5,0.5,0.5}
\definecolor{mauve}{rgb}{0.58,0,0.82}

\lstset{frame=tb,
    language=Java,
    aboveskip=3mm,
    belowskip=3mm,
    showstringspaces=false,
    columns=flexible,
    basicstyle={\small\ttfamily},
    numbers=none,
    numberstyle=\tiny\color{gray},
    keywordstyle=\color{blue},
    commentstyle=\color{dkgreen},
    stringstyle=\color{mauve},
    breaklines=true,
    breakatwhitespace=true,
    tabsize=3
}

\newcommand{\lstbg}[3][0pt]{{\fboxsep#1\colorbox{#2}{\strut #3}}}
\lstdefinelanguage{diff}[]{java}{
  style=common-style,
  morecomment=[f][\lstbg{HKS07!30}]-,
  morecomment=[f][\lstbg{HKS65!30}]+,
  morecomment=[f][\textit]{@@},
  %morecomment=[f][\textit]{---},
  %morecomment=[f][\textit]{+++},
}

\begin{document}
    \doublespacing
    \faculty{Fakultät Informatik}
    \department{}
    \institute{Institut für Software- und Multimediatechnik}
    \chair{Lehrstuhl für Softwaretechnologie}
    \title{%
        Entwicklung einer Web-Testsuite am Beispiel von jExam
    }

% for a master thesis
    \thesis{Bachelor}
    \graduation[B.Sc.]{Bachelor of Science}

    \author{Michael Vivian Mboni Saha}
    \emailaddress[]{michael_vivian.mboni_saha@tu-dresden.de}
    \matriculationnumber{4757191}
    \matriculationyear{2018}
    \discipline{Bachelor Informatik}
    \supervisor{%
        M.Sc. Ronny Marx
    }
    \professor{Prof. Dr. rer. nat habil. Uwe Aßmann}
    \date{29.01.2022}
    \maketitle
    \newpage

    \tableofcontents
    \listoffigures
    \listoftables
    \newpage
\chapter*{Abkürzungsverzeichnis}
\begin{acronym}
    \acro{url}[URL]{Uniform Resource Locator}
    \acro{w3c}[W3C]{World Wide Web Consortium}
    \acro{gui}[GUI]{Graphical user interface}
    \acro{http}[HTTP]{Hypertext Transfer Protocol}
    \acro{https}[HTTPS]{Hypertext Transfer Protocol Secure}
\end{acronym}
    \printglossary[title=Glossar]

    \chapter{Einleitung}\label{ch:einleitung}

Das erste Kapitel dieser Arbeit befasst sich mit der grundlegenden Motivation und definiert dann das allgemeine Problem, das gelöst werden soll. Es folgt eine Erläuterung der Zielsetzung und zum Schluss
wird der Aufbau der Arbeit im Detail beschrieben.

\section{Motivation}

Die meisten Entwickler finden Testen langweilig und denken, dass Testen den
Entwicklungsprozess verlangsamt. Nach Murugesan  wird es oft nur zweitrangig
berücksichtigt, insbesondere in den letzten und entscheidenden Phasen des
Softwareentwicklungsprozesses (vgl. \cite{526394}, S.112). Jedoch ist diese
Herangehensweise fehlerbehaftet und kostspielig. “Testing is not a PIT, it is a
LADDER ...! “ (Anand vgl. \cite{anand12importance} , S.02). Nach dieser Auffassung von
Anand geht es beim Testen vielmehr darum, Fehler und Situationen zu vermeiden,
die den Entwicklungsprozess einer Anwendung verlangsamen könnten.
Dadurch wird die Qualität der Software erhöht und die zeitlichen Vorgaben
werden eingehalten. Dies reduziert die Produktionskosten.


An der Technischen Universität Dresden steht den Studierenden der Informatik eine
Plattform für die Anmeldung zu Lehrveranstaltungen und Prüfungen sowie für die
Bekanntgabe der Prüfungsergebnisse zur Verfügung. Es handelt sich um ein
studentisches Projekt, das im Laufe der Zeit mit relativ wenig Personal
und einer hohen Personalfluktuation entwickelt wurde.  Das erschwert die
Wartung der Plattform und hat das Entwicklerteam dazu gezwungen , sich auf
Entwicklungs- und Wartungsaufgaben zu konzentrieren. Aus diesem Grund liegt
die Plattform im Moment ohne automatische Tests vor.


Die Entwicklung der Plattform in ihrer jetzigen Form basiert auf Technologien aus
dem Jahr 2009 und früher. Aufgrund der zwischenzeitlichen Entwicklung dieser
Kerntechnologien sowie der potenziellen Sicherheitsrisiken, die in den älteren
Versionen vorhanden sein könnten, besteht ein Bedarf an Weiterentwicklung.
Diese Arbeit wird derzeit von einer studentischen Hilfskraft aus der jExam-Gruppe
durchgeführt.


jExam ist jedoch eine kritische Plattform für Studenten, zum einen, weil sie
vertrauliche Informationen über Studenten und ihr Studium enthält, und zum
anderen, weil sie zu Beginn des Semesters für die Wahl der Fächer und am Ende
des Semesters für die Anmeldung zu Prüfungen unerlässlich ist. Um möglichen
Fehlern vorzubeugen und eine gute Wartung der Plattform zu gewährleisten, ist
es notwendig, zu testen.




\section{Zielsetzung und Abgrenzung}


Wie Azeem Uddin  sagte: ``Testen bedeutet herauszufinden, wie gut etwas funktioniert'' (vgl. \cite{anand12importance}, S.02).
Die aktuelle Version von jExam (die nun als \textbf{\gls{jexam_2009}} beschrieben wird) und die neue Version (\textbf{\gls{jexam_new}}),
die derzeit entwickelt wird, laufen ohne Tests. Das Testen von \gls{jexam_2009} wird dazu beitragen, die Plattform
besser zu warten, die Sicherheit zu erhöhen, die Qualität der Anwendungen zu verbessern und mögliche Fehler zu
vermeiden (vgl. \cite{shultz2011software}, S.21), während auf die Inbetriebnahme von \gls{jexam_new} noch aussteht.


Beide Versionen werden genau die gleichen Funktionalitäten haben, da die neue Version (\gls{jexam_new}) nur eine Kopie der
alten Version mit modernen Technologien ist. So kann eine Testinfrastruktur geschaffen werden, die mit beiden Versionen
kompatibel ist. Damit besteht die Möglichkeit zu testen, ob \gls{jexam_new} genau die gleichen Funktionalit\"aten wie
\gls{jexam_2009} hat und gleichzeitig alle Vorteile einer
getesteten Webanwendung zu haben.

Wegen des Mangels an Arbeitskräften im Entwicklungsteam ist es notwendig,
eine Infrastruktur zu schaffen, um das Schreiben zu erleichtern und die
Entwicklungszeit der Tests zu beschleunigen. Diese Arbeit zielt darauf ab,
das jExam-Entwicklungsteam bei der Wartung und schnellen Funktionsprüfung der
Plattform zu unterstützen. Dafür müssen die folgenden Ziele umgesetzt werden:


\begin{enumerate}
    \item Entwicklung einer Testsuite für die Neuentwicklung von jExam-Web.
    Dabei sollten mindestens folgende Funktionen abgedeckt sein:
    \begin{enumerate}
        \item Login
        \item Registrierung
        \item Abruf der Noten in der \"Ubersicht und als pdf
        \item Einschreibung in Pr\"ufungen
        \item Einschreibung in Lehrveranstaltungen
        \item Einschreibung in Seminargruppen
    \end{enumerate}
    \item Hohe Wartbarkeit der Testsuite
    \item Bereitstellung einer automatisierten Infrastruktur
    \item Erweiterung der Testsuite um Sicherheitstests
    \item Erweiterung der Testsuite um Performancetests
    \item Dokumentation
\end{enumerate}
\section{Aufbau der Arbeit}


Diese Arbeit ist in fünf Teile gegliedert. Zunächst wird das zum Verständnis der
Arbeit erforderliche Grundwissen erläutert. Darin werden verschiedene Konzepte
beschrieben, darunter Webanwendungen und die Grundlagen des Softwaretestens,
in denen grundlegende Fragen definiert werden. Außerdem erfolgt die Vorstellung
von verschiedenen Testarten und -methoden.


Dann folgt Kapitel drei, das der Beschreibung der zu testenden Plattformen
gewidmet ist, gefolgt von einer Phase der Analyse der Probleme, die die
Entwicklung von Tests erschweren und sogar die Verwendung bestimmter Werkzeuge
ausschließen können.  Am Ende desselben Kapitels werden verschiedene
Lösungsvorschläge erörtert.


Als nächstes kommt in Kapitel vier  der praktische Teil, in dem die verschiedenen
Technologien und Werkzeuge beschrieben werden, die während der Implementierung
verwendet wurden, wobei die Gründe für ihre Verwendung dargelegt werden,
gefolgt von einer detaillierten Beschreibung des Entwicklungsprozesses und
der verwendeten Architektur (Entwurfsmuster, Softwarearchitektur \ldots)


Die Ergebnisse des entwickelten Tools werden in Kapitel fünf vorgestellt und im
Hinblick auf verschiedene Aspekte wie Leistung und Fehler evaluiert. Schließlich
werden in Kapitel sechs Perspektiven für die Erweiterung der vorgestellten
Arbeit sowie die vollständige Dokumentation des Tools vorgestellt.




    \chapter{Grundlagen}\label{ch:grundlagen}

In diesem Kapitel werden die notwendigen Definitionen und Methoden erläutert,
die für das Verständnis der Arbeit von Bedeutung sind. Zunächst werden die
Ziele und Grenzen des Softwaretestens definiert. Dann folgt eine kurze
Einführung in den Begriff der Webanwendung und  die Erklärung des
Konzeptes der Testumgebung. Als Nächstes werden verschiedenen
Testmethoden und -ansätze vorgestellt. Das Konzept des Testens im Bereich der
Softwareentwicklung ist sehr weit gefasst und kann in dieser Arbeit
nicht vollständig behandelt werden. Aus diesem Grund werden die
verschiedenen Arten von Tests vorgestellt, die in dieser Arbeit
nützlich sein werden.  Schließlich um herauszufinden, wie guten Testfälle
geschrieben werden, erläutert der letzte Teil die Merkmale einer
gute \Gls{TestSuite}.

\section{Ziele und Grenzen des Softwaretestens}

Die Standarddefinition des Testens nach dem
ANSI/IEEE 1059-Standard besagt, dass Testen der
Prozess der Analyse eines Softwareobjekts ist, um
Unterschiede zwischen bestehenden und erforderlichen
Bedingungen (d.h. Defekte/Fehler/Bugs) zu erkennen
und die Eigenschaften des Softwareobjekts zu bewerten (vgl. \cite{singh2012software}, S.07).
Das Softwaretesten ist daher eine Methode, um zu überprüfen,
ob das Softwareprodukt den erwarteten
Anforderungen entspricht und um sicherzustellen, dass
es frei von Fehlern ist.

Im 21. Jahrhundert ist der Einsatz von Software und
Anwendungen weit verbreitet und kein Bereich bleibt
davon verschont. Die Gesamtmenge der weltweit erstellten,
erfassten, kopierten und verbrauchten Daten ist laut
Statista (vgl. \cite{Statista2021}) bis 2020 rasant auf 64,2
Zettabyte \begin{math}(2^{70})\end{math} angestiegen. Die Nutzer vertrauen ihre sensiblen und privaten Daten
Plattformen an, deren Aufgabe ist es, sie zu schützen. Testen
ist wichtig, weil Softwarefehler teuer oder sogar gefährlich
sein können. Sie können finanzielle
und menschliche Verluste verursachen, und die Geschichte
ist voll von solchen Beispielen:

\noindent
\begin{enumerate}
    \item Im Mai 1996 führte ein Softwarefehler dazu, dass
     den Konten von 823 Kunden einer großen US-Bank 920
     Millionen US-Dollar gutgeschrieben wurden (vgl. \cite{Devi2015}).
    \item China Airlines Airbus A300 crashed due to a software bug on April 26,
     1994, killing 264 innocents live (vgl. \cite{Takeuch1996}).
\end{enumerate}

Das Testen einer Anwendung hat viele Vorteile. Zu den wichtigsten
gehören die folgenden:


 \textbf{Kosteneffektivität}: Das
rechtzeitige Testen eines IT-Projekts hilft auf
lange Sicht Geld zu sparen. Wenn die Fehler bereits in
der frühen Phase des Softwaretests entdeckt werden,
kostet es weniger, sie zu beheben. Es ist besser, mit
dem Testen früher zu beginnen und es in jeder Phase des
Lebenszyklus der Softwareentwicklung einzuführen.
Regelmäßige Tests sind erforderlich (vgl. \cite{kumar2010software}, S.53), um
sicherzustellen, dass die Anwendung gemäß den Anforderungen entwickelt wird.

\begin{figure}[H]
    \centering
    \includegraphics[scale=0.5]{images/Cost-of-fixing-bugs-in-different-phases}
    \caption{Kosten für die Behebung von Fehlern (Bugs) in verschiedenen Phasen (vgl. \cite{kumar2010software})} \label{fig:mof}
\end{figure}


\textbf{Erhöhung der Sicherheit}: Sicherheit ist der anfälligste und
sensibelste Teil des Softwaretestens. Durch Testen wird sichergestellt,
dass die Anwendung über einen minimalen Schutz verfügt. Testen hilft
dabei, Risiken und Probleme früher zu beseitigen. So wird die Anwendung für Nutzer attraktiv,
die vertrauenswürdige Produkte suchen (vgl. \cite{shultz2011software}, S.09).

\textbf{Produktqualität}: Sie ist eine wesentliche
Voraussetzung für jedes Softwareprodukt. Zu den sechs Gruppen von Software-Qualitätsindikatoren
in der ISO-Norm 9126 (vgl. \cite{AlainAbran2010}) gehört die Wartbarkeit, zu der auch die Untergruppe Testbarkeit gehört.
Durch Testen kann die Qualität einer Anwendung sowie ihre Wartbarkeit erhöht werden und so wird dem Kunden sichergestellt,
dass ein Qualitätsprodukt geliefert wird.


Aus diesen Gründen ist das Testen von Software ein
integraler Bestandteil des Softwareentwicklungsprozesses, jedoch hat es Grenzen.
Testen dient nur dazu, das Vorhandensein von potentiellen Fehlern
aufzudecken. Aber es kann nicht sicherstellen, dass
die Software keine Fehler oder Bugs enthält (vgl. \cite{kumar2010software}, S.55).
Dazu können Tests nicht nachweisen, dass ein Produkt unter allen
Bedingungen richtig funktioniert, sondern nur, dass es unter
bestimmten Bedingungen nicht richtig funktioniert (vgl. \cite{kumar2010software}, S.56).



Da das Ziel dieser Arbeit darin besteht, eine \Gls{TestSuite} für
eine Webanwendung einzurichten, ist es wichtig zu
definieren, was mit Webanwendung eigentlich gemeint ist.

\section{Webanwendungen}

Das weite Feld der Softwareentwicklung umfasst auch die
Entwicklung von Webanwendungen. Lange Zeit wurden
Anwendungen als kompakte, installierbare Programme
verkauft. Dabei handelt es sich um die so genannten
klassischen Anwendungen oder Computeranwendungen, die
lokal auf einem Computer, Mobiltelefon oder Tablet
installiert werden müssen. Im Gegensatz zu herkömmlichen
Anwendungen werden Webanwendungen nicht lokal auf dem Gerät
des Nutzers installiert, sondern auf einem Server, so dass
sie über eine bestimmte \acs{url} zugänglich sind.

Nach Kappel et al. ist eine Webanwendung ein Softwaresystem,
das auf den Spezifikationen des World Wide Web Consortiums
(\acs{w3c}) basiert und Webressourcen bereitstellt, die über
eine Benutzeroberfläche wie einen Webbrowser genutzt werden
können (vgl. \cite{kappel1}, S.02).

Für den Betrieb einer Webanwendung werden mehrere
Computersystemkomponenten benötigt.Erstens das \gls{frontend},
das die grafische Oberfläche bezeichnet, mit der der
Benutzer interagiert, gefolgt vom \gls{backend}, das die
Logik der Anwendung enthält (z.B Datenbanken, Cloud).


Der Benutzer greift auf die Webanwendung über einen Computer
zu, der als Client bezeichnet wird. Der Client sendet eine
oder mehrere Anfragen über das Internet oder Intranet via
\acs{http} oder \acs{https}  an einen anderen Computer (Server),
auf dem die Webanwendung läuft. Der Server nimmt dann die
HTTP-Anfragen entgegen und verarbeitet sie. Je nach Anfrage
werden die angeforderten Daten entweder aus der Datenbank
abgerufen oder gespeichert. Die verarbeiteten Daten werden
dann vom Server in einer entsprechenden Antwort
(HTTP-Response) an den Client zurückgesendet und im
Webbrowser angezeigt. So funktionieren Webanwendungen grundsätzlich.

Um Webanwendungen zu testen, ist es wichtig, eine Testumgebung zu schaffen.
Diese Testumgebung ermöglicht es dem Tester, alle möglichen Schwachstellen in einer Webanwendung zu untersuchen,
ohne die Anwendung zu gefährden, wenn sie bereits in Produktion ist.


\section{Testumgebung}

Im Moment ist die alte Version von jexam (\gls{jexam_2009})
in \gls{prod}, weil sie stabil und funktional ist.
Die Entwickler des Teams sorgen dafür, dass die
Funktionen der Anwendung ordnungsgemäß funktionieren.
Dies ist bereits eine Art manueller Test. Die neue
Version, die derzeit entwickelt wird, wurde noch nicht
vollständig getestet. Daher ist sie von unbekannter
Qualität und kann nicht als stabil für die \gls{prod}
gelten. Sie sollte nicht in die \gls{prod} aufgenommen
werden, damit die Benutzer sie testen können.  Wenn
die Entwickler dies täten, würden sie Schwachstellen
aufdecken, die von böswilligen Personen genutzt werden
könnten, um die Sicherheit der Nutzerdaten zu
gefährden. Bevor eine Anwendung in \gls{prod} geht,
sollte sie getestet werden. Da dies nicht in der
\gls{prod} geschehen sollte, ist es notwendig, eine
Umgebung zu schaffen, in der die Anwendung getestet
werden kann.


Nach Everett eine Testumgebung ist eine Umgebung,
die Hardware, Instrumente, Simulatoren, Software-Tools
und andere unterstützende Elemente enthält, die für
die Durchführung eines Tests erforderlich sind,
d.h. sie ermöglicht es, die Anwendung zu testen, ohne
den Kunden zu beeinträchtigen (vgl. \cite{shultz2011software}, S. 150-152).
Die Einrichtung einer Testumgebung hat eine Reihe von Vorteilen :

\noindent
\begin{enumerate}
    \item Die Nutzer nicht mit einer Anwendung von
    zweifelhafter Qualität zu belästigen. Ein
    Qualitätsabfall ist oft sehr nachteilig für eine
    Anwendung. Die Nutzer könnten sich beschweren,
    dass die alte Version besser ist und die ganze
    Arbeit der Entwickler, die die neue Version
    entwickelt haben, preisgeben.

    \item Die Einbeziehung aller Szenarien in die
    Tests, was die Zuverlässigkeit der Software
    erhöht. Sie hilft auch dabei, fehlende
    Implementierungen der Software zu finden. In der
    Produktion wäre dies eine komplizierte Aufgabe, da
    sich die Daten und die Anwendung ständig ändern.

    \item Die Senkung der Produktionskosten durch
    frühzeitige Erkennung von Fehlern.

\end{enumerate}

Testumgebungen sind für jede Softwareentwicklung
notwendig. Sie ermöglichen es, das Produkt zu testen,
bevor es auf den Markt kommt, ohne ein Risiko für die
Produktion einzugehen. Wenn sie jedoch wirklich
wirksam sein sollen, dürfen sie die Entwicklung nicht
behindern, sondern müssen auch zuverlässig sein. Aus
diesem Grund muss diesen Umgebungen besondere
Aufmerksamkeit geschenkt werden. Die endgültige
Qualität des Produkts hängt stark von der Qualität der
Testumgebung ab (vgl. \cite{shultz2011software}, S. 152).


\section{Testsmethoden und -ansätze}

Softwareobjekte können auf unterschiedliche Weise
getestet werden. Generell wird zwischen statischen und
dynamischen Tests unterschieden. In den folgenden
Fällen werden beide Konzepte in den folgenden
Abschnitten ausführlicher dargestellt.

\subsection{Statisches Testen}

Statisches Testen ist eine Softwaretestmethode,
bei der ein Programm zusammen mit den zugehörigen
Dokumenten untersucht wird, ohne dass das Programm
ausgeführt werden muss. Konkret besteht der Prozess aus
der Prüfung schriftlicher Dokumente, die insgesamt
einen Überblick über die zu testende Softwareanwendung
geben. Zu den geprüften Dokumenten gehören
Anforderungsspezifikationen, Designdokumente,
Benutzerdokumente, Webseiteninhalte, Quellcode,
Testfälle, Testdaten und Testskripte,
Benutzerdokumente, Spezifikations- und
Matrixdokumente. Statische Tests erleichtern die
Kommunikation zwischen den Teams und vermitteln
einen besseren Eindruck von den Qualitätsproblemen
in der Software. Es reduziert nicht nur die Kosten
in den frühen Entwicklungsphasen (in Bezug auf die
Menge an Arbeit, die neu gemacht werden muss, um
eventuelle Fehler zu korrigieren), sondern auch die
Entwicklungszeit.

\subsection{Dynamisches Testen}

Dynamisches Testen ist eine Methode zur Bewertung
der Durchführbarkeit eines Softwareprogramms durch
Eingabe und Prüfung der Ausgabe. Die dynamische Methode
erfordert, dass der Code kompiliert und ausgeführt
wird. Dynamische Tests werden in zwei Kategorien unterteilt:
Whitebox- und  Blackboxtestverfahren.

\textbf{Whitebox-Testverfahren}

Whitebox-Testen ist eine Softwaretestmethode, bei
der dem Tester die interne Struktur/das Design
bekannt ist. Das Hauptziel  vom Whitebox-Testen ist
die Überprüfung der Korrektheit der Software
Anweisungen, Codepfade, Bedingungen, Schleifen
und Datenflüsse. Dieses Ziel wird oft als logische
Abdeckung bezeichnet (vgl. \cite{shultz2011software}, S.107).


\textbf{Blackbox-Testverfahren}

Blackbox-Testen ist eine Testmethode, bei der die
interne Struktur/der Code/das Design dem Tester nicht
bekannt ist. Das Hauptziel dieses Tests ist es, die
Funktionalität des zu testenden Systems zu überprüfen,
und diese Art von Tests erfordert die Ausführung der
kompletten Testsuite (vgl. \cite{shultz2011software}, S.112).


Die Blackbox-Methode ist diejenige, die in dieser
Arbeit entwickelt wird. Sie selbst ist in mehrere
Arten unterteilt, auf die im nächsten Kapitel
eingegangen wird.




\section{Verschiedene Testarten}

Bei der Blackbox-Methode gibt es zwei Hauptarten von
Testen: funktionale Tests und nicht-funktionale Tests.

\subsection{Funktionale Tests}

Funktionale Tests werden durchgeführt, um zu überprüfen,
ob alle entwickelten Funktionen mit den funktionalen
Spezifikationen übereinstimmen. Dies wird durch die
Ausführung der funktionalen Testfälle erreicht. In der
Funktionstestphase wird das System getestet, indem
Eingaben gemacht, Ausgaben überprüft und die
tatsächlichen Ergebnisse mit den erwarteten
Ergebnissen verglichen werden. Bei diesen Tests werden
Benutzeroberfläche, \acs{api}s, Datenbank, Sicherheit,
Client/Server-Kommunikation und andere Funktionen der zu
testenden Anwendung überprüft. Die Tests können entweder
manuell oder durch Automatisierung durchgeführt werden.

\textbf{Unit-Tests}

James Whittaker erklärt, dass Unit-Test einzelne Softwarekomponenten oder
eine Sammlung von Komponenten testet (vgl. \cite{Whittaker2000}, S.70-79).
Unit-Testing ist die erste Stufe des Softwaretests, bei der einzelne
Komponenten eines Softwarepakets getestet werden, während der Rest des
Systems ignoriert wird. Auch Modul- oder Komponententest genannt,
wird Unit-Test während der Entwicklung einer Anwendung durchgeführt,
um zu prüfen, ob die einzelnen Einheiten oder Module einer Anwendung
ordnungsgemäß funktionieren.


Das Ziel von Unit-Tests ist es, Probleme in einem frühen Stadium des
Entwicklungszyklus zu finden. Dadurch werden die Testkosten gesenkt
(die Kosten für das frühzeitige Auffinden eines Fehlers sind wesentlich
geringer als die Kosten für das spätere Auffinden). Sie reduzieren Fehler
bei der Änderung bestehender Funktionen, so dass sie leicht gefunden und
behoben werden können. Dies vereinfacht den \gls{debug}-Prozess erheblich.


\textbf{Integration-Tests}

Es ist eine Erweiterung des Unit-Tests. Beim Integrationstest wird die
Konnektivität oder der Datentransfer zwischen den einzelnen getesteten
Modulen (unit tested modules) getestet. Nach Leung und White sind
Integrationstests die Tests, die durchgeführt werden, wenn alle
einzelnen Module zu einem funktionierenden Programm kombiniert
werden (vgl. \cite{131377}, S.290). Das Testen erfolgt auf Modulebene und nicht
auf Anweisungsebene wie beim Unit-Test. Beim Integrationstest liegt der
Schwerpunkt auf den Interaktionen zwischen den Modulen und ihren
Schnittstellen.


Mit Integrationstests wird überprüft, ob das funktionale und nicht-funktionale
Verhalten der Schnittstellen dem Softwaredesign und -spezifikationen
entspricht. Dies führt dazu, dass Fehler daran gehindert werden, in höhere
Teststufen zu gelangen.


\textbf{System-Tests}

Systemtests sind eine Stufe der Softwaretests über den Integrationstests.
Laut Briand et al. werden sie an einer vollständigen und voll integrierten
Anwendung durchgeführt, um die Übereinstimmung des Systems mit den
spezifizierten Anforderungen zu bewerten (vgl. \cite{briand2002uml}, S.10).
Bei dieser Art von funktionalen tests validieren die Tester das vollständige
und integrierte Softwarepaket. Damit wird sichergestellt, dass es den
Anforderungen entspricht. Bei Bedarf können die Tester Feedback zur
Funktionalität und Leistung der App oder Website geben, ohne vorher
zu wissen, wie sie programmiert wurde. Dies hilft den Teams bei der
Entwicklung von Testfällen, die in Zukunft verwendet werden sollen.
Systemtests werden auch als End-to-End-Tests bezeichnet.


\textbf{Automatisierte Benutzeroberfläche-Tests (\acs{ui} Automation Testing) }

Die erste Interaktion zwischen einem Benutzer und einer Software findet
über eine grafische Benutzeroberfläche (\acs{gui} acronym) statt. Beim
\acs{ui}-Testing wird überprüft, ob die Endbenutzeroberfläche korrekt
funktioniert. Bei der Durchführung von \acs{ui}-Tests wird geprüft, ob jedes
Stückchen Logik, jede \acs{ui}-Funktion oder jeder Aktionsablauf wie erwartet
funktioniert. Hier konzentrieren sich die Tester auf die Validierung jedes
Klicks auf eine Schaltfläche, der Dateneingabe, der Navigation, der
Berechnung von Werten und anderer Funktionalitäten, die für die
Benutzerinteraktion verwendet werden.


\acs{ui} Automation Testing ist eine Technik, bei der diese Testprozesse mithilfe
eines Automatisierungstools durchgeführt werden. Anstatt dass sich die
Tester durch die Anwendung klicken, um die Daten- und Aktionsflüsse
visuell zu überprüfen, werden für jeden Testfall Testskripte geschrieben.
Anschließend wird eine Reihe von Schritten hinzugefügt, die bei der
Überprüfung der Daten zu befolgen sind. Der Prozess der
\acs{ui}-Automatisierungstests vereinfacht die Erstellung von UI-Tests, die
Ausführung der Tests und die Anzeige der Ergebnisse (vgl.\cite{Perfecto2020}).
Es ermöglicht Testern, die Interaktion einer Anwendung mit den Endnutzern zu
simulieren und zu testen. Weitere Vorteile sind die Automatisierung aller
Testaktivitäten für die zu testende Anwendung und die Integration von
Benutzerschnittstellentests in den Entwicklungsprozess.


Automatisiertes Testen ist eine der Funktionen, die im Mittelpunkt dieser
Arbeit stehen. Es ist daher wichtig, dieses Konzept zu verstehen, um
weiterzukommen.



\subsection{Nichtfunktionale Tests}

Nichtfunktionales Testen ist das Testen einer Softwareanwendung oder
eines Systems auf seine nichtfunktionalen Anforderungen. Es dient dazu,
die Bereitschaft eines Systems in Bezug auf nichtfunktionale Parameter zu
testen, die bei funktionalen Tests nicht berücksichtigt werden.
Nichtfunktionale Tests beziehen sich auf verschiedene Aspekte der Software
wie Leistung, Belastung, Stress, Skalierbarkeit, Sicherheit, Kompatibilität.
Der Schwerpunkt liegt auf der Verbesserung der Benutzererfahrung.


Es gibt über 100 Arten von Nichtfunktionalen Tests, von denen einige in den
folgenden Abschnitten behandelt werden:

\textbf{Performancetests}

Nach Vokolos et al. sind Performancetests die Gesamtheit aller Aktivitäten,
die an der Bewertung der Leistung einer Software in \Gls{prod}
beteiligt sind (\cite{vokolos1998performance}, S. 80).
Diese Leistung wird aus der Sicht des Benutzers bewertet und in der
Regel in Form von Durchsatz, Reaktionszeit auf Stimuli oder einer
Kombination aus beiden bewertet. Performancetests können auch verwendet
werden, um den Grad der Verfügbarkeit bzw stabilität der Software zu
bewerten. Für viele Anwendungen in der Telekommunikation oder im
medizinischen Bereich ist es beispielsweise von entscheidender Bedeutung,
dass das System immer verfügbar ist.


Performancetests sind entscheidend, um festzustellen, ob eine Anwendung die
Leistungsanforderungen erfüllt (z.B.\ muss das System bis zu 1 000
gleichzeitige Benutzer verwalten können). Sie helfen auch dabei, Engpässe
(\Gls{bottleneck}) in einer Anwendung zu lokalisieren. Sie werden auch
für den Vergleich von zwei oder mehr Systemen verwendet, um das
leistungsfähigste zu identifizieren (z. B. \Gls{jexam_2009}  und \Gls{jexam_new}).
Es ermöglicht auch eine effektive Messung der Stabilität in Bezug auf den
Datenverkehr. Performancetests unterteilen sich in weitere Unterkategorien:


\textbf{Lasttests}: Sie sind eine Art von Performancetests,
bei denen die Anwendung auf ihre Leistung bei normaler und Spitzenbelastung
getestet wird. Die Leistung einer Anwendung wird im Hinblick auf ihre Reaktion
auf Benutzeranfragen und ihre Fähigkeit, innerhalb einer akzeptierten Toleranz
bei unterschiedlichen Benutzerlasten konsistent zu reagieren, überprüft.


\textbf{Stresstests}: Sie werden eingesetzt, um Wege zu finden, das System zu
brechen. Der Test liefert auch den Bereich der maximalen Belastung,
die das System aushalten kann. In der Regel wird beim Stresstest ein
schrittweiser Ansatz verfolgt, bei dem die Belastung schrittweise
erhöht wird. Der Test wird mit einer Last begonnen, für die die Anwendung
bereits getestet wurde. Dann wird die Last langsam erhöht, um das System
zu belasten. Der Punkt, an dem wir feststellen, dass die Server nicht mehr
auf die Anfragen reagieren, wird als Sollbruchstelle betrachtet.


Dies sind die einzigen beiden Unterkategorien von Performancetests,
die in dieser Arbeit verwendet werden.



\textbf{Penetrationtests}

Ein Penetrationstest ist ein Versuch, die Sicherheit einer IT-Infrastruktur
zu bewerten, indem auf sichere Weise versucht wird, Schwachstellen
auszunutzen. Diese Schwachstellen können in Betriebssystemen, Diensten
und Anwendungsfehlern, unsachgemäßen Konfigurationen oder riskantem
Verhalten der Endbenutzer bestehen. Solche Bewertungen sind auch nützlich,
um die Wirksamkeit von Verteidigungsmechanismen und die Einhaltung von
Sicherheitsrichtlinien durch die Endbenutzer zu überprüfen.


Penetrationstests werden in der Regel mit manuellen oder automatisierten
Technologien durchgeführt, um systematisch Server, Endpunkte, Webanwendungen,
drahtlose Netzwerke, Netzwerkgeräte, mobile Geräte und andere potenzielle
Angriffspunkte zu kompromittieren. Sobald die Schwachstellen in einem
bestimmten System erfolgreich ausgenutzt wurden, können die Tester versuchen,
das kompromittierte System für weitere Angriffe auf andere interne Ressourcen
zu nutzen. Insbesondere indem sie versuchen, schrittweise höhere
Sicherheitsstufen und einen tieferen Zugang zu elektronischen Ressourcen
und Informationen über die Ausweitung von Berechtigungen zu erreichen.


Laut Brad Arkin et al. sind Penetrationstests die am häufigsten und am
weitesten verbreiteten Best Practices im Bereich der Softwaresicherheit,
was zum Teil daran liegt, dass es sich um eine attraktive Aktivität am
Ende des Lebenszyklus handelt (vgl. \citte{1392709}, S.84).  Sobald eine
Anwendung fertiggestellt ist, sollen die Tester sie im Rahmen der Endabnahme
Penetrationstests unterziehen. Der Hauptzweck von Penetrationstests besteht
also darin, Schwachstellen in einem System zu identifizieren, um sie zu
schließen. Dies ermöglicht es den Testern, die Sicherheit der Anwendung
zu erhöhen und die Sicherheitsstrategie zu verbessern.


So wie funktionale Tests zeigen, wie gut eine Anwendung funktioniert,
so sind auch nicht-funktionale Tests für die Benutzererfahrung von
großer Bedeutung.




\section{Merkmale einer guten Testsuite}



    \chapter{Problemanalyse}\label{ch:problemanalyse}


Nachdem die f\"ur das Verständnis notwendigen Grundlagen gekl\"art wurden,
wird in diesem Kapitel eine Analyse der Jexam-Anwendung vorgenommen.
Ziel dieses Abschnitts ist es, die Anwendung vorzustellen und zu untersuchen,
um das n\"otige Wissen zu erlangen, um eine vollautomatische Testsuite
einrichten zu können.

\section{Beschreibung von jExam}

jExam ist eine in der Programmiersprache Java entwickelte Software,
die seit dem Jahr 2000 an der Fakult\"at Informatik der Technischen
Universit\"at Dresden für die Anmeldung zu Lehrveranstaltungen und
Pr\"ufungen sowie für die Mitteilung von Pr\"ufungsergebnissen eingesetzt
wird. Die Entwicklung der Weboberfläche in der jetzigen Form basiert
auf Technologien, welche auf das Jahr 2009 und Frühere zur\"uckgehen.
Seit 2009 haben sich die Programmiersprachen und Technologien jedoch
stetig weiterentwickelt. Allein bei Java hat sich die Entwicklung von
Java SE 6, das 2009 verwendet wurde, zu Java SE 11, das heute verwendet
wird, fortgesetzt. Die Anwendung muss auch mit neuen Versionen von
Programmiersprachen und Frameworks aktualisiert werden, und das aus
mehreren Gründen:


\textbf{Schwachstellen verringern:} Das ist der Hauptgrund für ein
Software-Update. Wenn eine Plattform dem Internet ausgesetzt ist,
sind die Datenbanken, in denen alle Details der Nutzer gespeichert
sind, zunehmend Sicherheitsbedrohungen ausgesetzt. Böswillige
Personen, die über immer bessere Werkzeuge verfügen, finden
Schwachstellen in der Software.  Durch Software-Updates werden einige
dieser Sicherheitslücken überdeckt, sodass sie nicht ausgenutzt werden
können (vgl \cite{10.1145/605466.605479}, S.82).

\textbf{Behebung von Fehlern und Abstürzen:} Ausfälle, Probleme, Fehler
werden behoben, wenn ein Unternehmen eine aktualisierte Version eines
Programms erstellt. Jede entwickelte Software hat inhärente Fehler
oder Verbesserungsmöglichkeiten. Wenn Hersteller Sicherheitslücken
oder Bugs entdecken, kleinere Verbesserungen an Programmen vornehmen
oder Kompatibilitätsprobleme beheben, veröffentlichen sie Updates.
Durch die Aktualisierung wird sichergestellt, dass die neueste und
stabilste Version weniger Fehler aufweist (vgl \cite{10.1145/605466.605479}, S.82).

\textbf{Gewährleistung der Kompatibilität mit anderen aktualisierten
Technologien:} Anwendungen funktionieren heute nicht mehr völlig unabhängig
voneinander, sondern kommunizieren miteinander. Daher ist es wichtig, dass eine
Plattform auf dem neuesten Stand ist, um kompatibel zu sein und die
Vorteile anderer Bibliotheken, Frameworks oder Tools nutzen zu können.


\textbf{Gewinn an Leistung und Funktionalität:} Die Aktualisierung einer
Programmiersprache oder einer Software ist oft mit einem
Leistungszuwachs verbunden.

Aus diesen Gründen wird derzeit eine neue Version von jExam (\gls{jexam_new})
mit neueren Technologien entwickelt. Um herauszufinden, ob die neue
Plattform funktional und effizienter als die alte Version (\gls{jexam_2009})
ist, muss eine Testinfrastruktur eingerichtet werden. Dies erfordert
nicht nur das Testen der beiden Plattformen, sondern auch einen Vergleich
zwischen ihnen.


JExam ist eine kritische Plattform für Studierende der TU Dresden.
Sie speichert viele sensible persönliche Informationen über die
Nutzer. Außerdem muss sie immer online und schnell verfügbar sein,
um eine gute Softwarequalität zu gewährleisten. Aufgrund der vielen
Vorteile, die das Testen einer Plattform mit sich bringt
(siehe \autoref{ch:grundlagen}), sollte die Plattform mit einer Reihe von
Anforderungen getestet werden. Das ist das Ziel des nächsten Abschnitts.
\input{sections/problemanalyse/herausforderungen.tex}
    \chapter{Implementierung der automatisierten Testinfrastruktur}

Das Endziel dieser Arbeit ist die Einrichtung einer automatisierten
Testinfrastruktur, um die Webanwendung jExam und ihre zukünftige
Version, die sich noch in der Entwicklung befindet, zu testen.
Dabei sollen nicht nur die wichtigsten
Funktionen getestet werden, sondern auch die Sicherheit und die
Performance. Dieses Kapitel konzentriert sich auf die Einrichtung
dieser Testinfrastruktur sowie auf das Design und die Entwicklung
der Tests. Außerdem werden die verschiedenen verwendeten Technologien
vorgestellt und ihre Verwendung begründet.


\section{Vorstellung der Testansatzes}

Die jExam-Webanwendung muss auf drei verschiedenen Ebenen getestet
werden: Funktionalitäten, Performance und Sicherheit. Es ist jedoch
nicht möglich, diese drei Ebenen in einer einzigen Testsuite zu
kombinieren. Dies erfordert die Einrichtung einer Infrastruktur,
die die Erstellung und Ausführung der Tests verwaltet. Die Anwendung,
die verwendet wird, um die verschiedenen Dienste zu trennen, ist Docker \cite{docker}
(wird in den folgenden Kapiteln behandelt). Docker  wird nicht nur für
die Trennung der Testebenen verwendet. Sie ermöglicht es auch, beide
Versionen von jExam zu deployen, sodass sie effizient und lokal
getestet werden können. Über diese Infrastruktur wird es möglich
sein, Tests auszuführen und am Ende ihrer Ausführung Berichte und
Metriken zu erhalten. Über diese Infrastruktur wird es möglich sein,
Tests auszuführen und am Ende ihrer Ausführung Berichte und Metriken
zu erhalten. Diese ermöglichen , die Leistung der beiden Plattformen
zu beobachten und zu vergleichen. In den nächsten Kapiteln werden
alle diese Konzepte im Detail behandelt.


\section{Entwicklung von Sicherheitstests}\label{sec:entwicklung-von-sicherheitstests}

\subsection{\acs{owasp}}

\acs{owasp} ist das Akronym für \textbf{O}pen \textbf{W}eb
\textbf{A}pplication \textbf{S}ecurity \textbf{P}roject und
beschreibt eine offene Organisation, deren Hauptziel ist, die
Sicherheit von Anwendungen, Diensten und Software zu verbessern.
Sie wurde am 1. Dezember 2001 gegründet und am 21. April 2004 als
gemeinnützige Organisation offiziell anerkannt.  Sie ermöglicht
Organisationen, Unternehmen oder Einzelpersonen, sichere Anwendungen zu
entwickeln und zu warten. Die \acs{owasp} hat eine Reihe von
Sicherheitstools und -richtlinien entwickelt. Dazu gehören die \acs{owasp}
Top 10 und der Zed Attack Proxy (\acs{zap}). Ein wichtiger
Grundsatz der OWASP ist, dass jeder, der sich für die Sicherheit
von Webanwendungen interessiert, weltweit kostenlos über das
nötige Wissen und die Werkzeuge verfügen kann (vgl. \cite{owasp}).
In diesem Zusammenhang hat sie die OWASP TOP 10 erstellt, die eine
Liste der zehn häufigsten Sicherheitslücken im Internet
darstellt. Ihr Hauptzweck ist die Schulung all jener, die mit der
Entwicklung sicherer Anwendungen zu tun haben.  Sie sollen über die
üblichsten Sicherheitslücken informiert werden und dadurch diese
vermeiden.  Im Folgenden werden die zehn Angriffe der OWASP TOP
10 auf der Basis der OWASP TOP 10 2017 
Release Candidate 2 (vgl. \cite{Stock2017}) beschrieben.

\begin{table}[h]
    \begin{tabulary}{\textwidth}{@{}L@{}}
        \toprule
        \textbf{OWASP TOP10 2017} \tabularnewline\midrule
        1"~ Injektion (Injection)
        \tabularnewline
        2"~ Fehler in der Authentifizierung (Broken Authentication)
        \tabularnewline
        3"~ Verlust der Vertraulichkeit von Daten (Sensitive Data Exposure)
        \tabularnewline
        4"~ XML External Entities (XML)
        \tabularnewline
        5"~ Fehler in der Zugriffskontrolle (Broken Access Control)
        \tabularnewline
        6"~ Sicherheitsrelevante Fehlkonfiguration (Security Misconfiguration)
        \tabularnewline
        7"~ Cross-Site-Scripting (XSS)
        \tabularnewline
        8"~ Unsichere Deserialisierung (Insecure Deserialization)
        \tabularnewline
        9"~ Nutzung von Komponenten mit bekannten Schwachstellen (Using Components with Known Vulnerabilities)
        \tabularnewline
        10"~ Unzureichendes Logging und Monitoring (Insufficient Logging and Monitoring)
        \tabularnewline\bottomrule
    \end{tabulary}
    \caption{OWASP TOP 10 2017}\label{tab:OWASP_TOP_10}
\end{table}

\input{sections/implementierung/sicherheit/top10/injection.tex}
\input{sections/implementierung/sicherheit/top10/auth.tex}
\input{sections/implementierung/sicherheit/top10/vertraulichkeit.tex}
\input{sections/implementierung/sicherheit/top10/xml.tex}
\input{sections/implementierung/sicherheit/top10/zugriff.tex}
\input{sections/implementierung/sicherheit/top10/Fehlkonfiguration.tex}
\input{sections/implementierung/sicherheit/top10/xss.tex}
\input{sections/implementierung/sicherheit/top10/deserialisierung.tex}


\subsection{ZAP Proxy}

Wie in \autoref{ch:sicherheit} erwähnt, müssen beide Versionen von jExam
auf größere Sicherheitslücken gescannt werden. Schwachstellen-Scanner für
Webanwendungen sind automatisierte Tools, die Webanwendungen - normalerweise
von außen - auf Sicherheitslücken wie Cross-Site-Scripting, SQL Injektion,
Command Injektion, Path Traversal und unsichere Serverkonfiguration
untersuchen (vgl. \cite{vulscan}). Diese Kategorie von Tools wird häufig
als Dynamic Application Security Testing (\acs{dast}) Tools bezeichnet.
Es gibt eine große Anzahl kommerzieller und Open-Source-Tools dieser Art,
und alle diese Tools haben ihre eigenen Stärken und Schwächen. Die Analyse
der verschiedenen Tools zum Aufspüren von Schwachstellen wird in dieser
Arbeit nicht behandelt. Es gibt jedoch das OWASP Benchmark-Projekt
(vgl \cite{benchmark}), das die Effektivität aller Arten von Tools zum
Aufspüren von Schwachstellen, einschließlich \asc{dast}, wissenschaftlich
misst. Das Tool zum Scannen von Schwachstellen, das in dieser Arbeit verwendet wird,
ist ZAP Proxy.

\acs{zap} steht für Zed Attack Proxy und bezeichnet eine
Open-Source-Sicherheitssoftware, die in der Programmiersprache Java
geschrieben und 2010 veröffentlicht wurde. Sie wird verwendet, um
Webanwendungen auf Schwachstellen zu scannen. Es wurde als kleines
Projekt vom Open Web Application Security Project (OWASP) gestartet und
ist heute das aktivste Projekt, das von Tausenden von Menschen auf der
ganzen Welt betreut wird. Es ist der am häufigsten verwendete
Webanwendungsscanner (vgl. \cite{zap}). \acs{zap} ist für Linux, Windows und Mac
in 29 Sprachen verfügbar. Zed Attack Proxy wird verwendet, um
Schwachstellen auf beliebigen Webservern zu erkennen. Zu den wichtigsten
Schwachstellen, die von \asc{zap} erkannt werden können, gehören :

\begin{enumerate}
    \item SQL injektion (Injection)
    \item Fehler in der Authentifizierung (Broken Authentication)
    \item Verlust der Vertraulichkeit von Daten (Sensitive data exposure)
    \item Fehler in der Zugriffskontrolle (Broken Access control)
    \item Sicherheitsrelevante Fehlkonfiguration (Security misconfiguration)
    \item Cross Site Scripting (XSS)
    \item Unsichere Deserialisierung (Insecure Deserialization)
    \item Nutzung von Komponenten mit bekannten Schwachstellen (Components with known vulnerabilities)
    \item Fehlende Sicherheitsheader (Missing security headers)
\end{enumerate}

Was Zap zum meistgenutzten Werkzeug für Sicherheitsprüfungen macht,
ist in erster Linie die Tatsache, dass es nicht nur für die Verwendung
durch erfahrene Penetrationstester, sondern auch für Anfänger auf diesem
Gebiet konzipiert wurde. ZAP ist ein kostenloses Open-Source-Tool, das
einfach einzurichten und zu verwenden ist. Da es von einer breiten Community
verwendet wird, gibt es online im ZAP-Blog und in anderen Artikeln eine Menge
Hilfe, die bei der Einrichtung und Verwendung des Tools hilft.
ZAP kann in einem Docker-Container ausgeführt werden. Außerdem ist die
Funktionalität skalierbar mit vielen verschiedenen Erweiterungen, die
auf GitHub veröffentlicht wurden.


ZAP ist ein so genannter "Man-in-the-middle-Proxy". Er wird zwischen den
Browser und die Webanwendung geschaltet. Während der Tester durch alle
Funktionen der Website navigiert, erfasst er alle Aktionen. Anschließend
greift er die Website mit bekannten Techniken an, um Sicherheitslücken zu
finden.  ZAP ist ein Werkzeug, das menschliche Intelligenz benötigt,
um richtig eingesetzt zu werden.
Penetrationstester verwenden es zunächst, um automatisch nach Schwachstellen
in einer Webanwendung zu scannen. Sobald sie eine Schwachstelle gefunden
haben, die sie ausnutzen können, verwenden sie ZAP, um die Anwendung
anzugreifen. Wie bereits in \autoref{ch:sicherheit} erwähnt, ist die einzige Funktion,
die jExam verwenden wird, der Scanner. Es geht darum, die Anwendung zu
scannen, um größere Sicherheitslücken zu finden.  Wenn die Anwendung solche
Lücken enthält, ist es für die Entwickler unerlässlich, zu handeln und
herauszufinden, wie man sie beheben kann. Auf diese Weise ist es möglich,
ZAP vollautomatisch zu verwenden.


\begin{figure}[H]
    \centering
    \includegraphics[scale=0.5]{images/zap-interface}
    \caption{Grafische Benutzeroberfläche von Zap} \label{fig:zap-interface}
\end{figure}



\subsection{Tests und Ergebnisse}

Die Testinfrastruktur von jExam wurde mit Docker unter Verwendung
von docker-compose entwickelt. Dies bietet die Möglichkeit, die
Testdienste in verschiedene Container aufzuteilen (dieses Konzept
wird in den nächsten Kapiteln ausführlich behandelt). Zu den
Testdiensten gehört auch ein Container, der speziell für Sicherheitstests
vorgesehen ist und automatisch bestimmte Befehle ausführt, um eine Version
von jExam zu scannen. Zunächst ist es notwendig, einige Begriffe zu
erklären, die im Folgenden verwendet werden.

\input{sections/implementierung/sicherheit/begriffe/spider.tex}
\input{sections/implementierung/sicherheit/begriffe/ajax.tex}
\input{sections/implementierung/sicherheit/begriffe/passive.tex}
\input{sections/implementierung/sicherheit/begriffe/active.tex}


Auf der jExam-Plattform wurden zwei Skripte für ihre Ausführung
implementiert. Es handelt sich dabei um die Skripte ZAP Baseline und
ZAP Full scan.  Da es zwei Versionen von jExam gibt, muss bei der
Ausführung der Skripte entschieden werden, welche der beiden Plattformen
getestet werden soll.

\input{sections/implementierung/sicherheit/begriffe/baseline.tex}
\input{sections/implementierung/sicherheit/begriffe/fullscan.tex}



Die Sicherheit einer Anwendung zu testen ist eine schwierige Aufgabe,
aber Werkzeuge wie Zap geben vielen Menschen die Möglichkeit, sich in
diesem Bereich leicht einzuarbeiten. Mit Hilfe der Scanner ist es möglich,
Sicherheitslücken zu finden, die den Entwicklern ermöglichen, sie zu
beheben. Dadurch wird eine gute Softwarequalität für die Zukunft gewährleistet.









\section{Entwicklung von Performancetests}

Performancetests sind eine nicht-funktionale Art von Tests, die
durchgeführt werden, um die Leistung einer Anwendung zu bestimmen.
Die Tests werden anhand von Metriken wie Geschwindigkeit, Stabilität
und Skalierbarkeit durchgeführt. Die Leistung einer Anwendung zu testen,
ist eine recht umfangreiche und für jede Anwendung spezifische Aufgabe.
Tester müssen die wichtigsten Leistungsindikatoren der Anwendung bestimmen
und eine Reihe von Tests entwerfen. Der Zugang zu diesem Bereich des
Testens wurde jedoch durch das Aufkommen von automatischen Werkzeuge für
Anfänger stark vereinfacht.  Es gibt mehrere kostenpflichtige und kostenlose
Tools, die alle ihre eigenen Vor- und Nachteile haben. In diesem Kapitel
werden keine Vergleiche zwischen diesen Tools angestellt, aber es gibt
Ressourcen, die diesen Vergleich detailliert durchgeführt haben
(vgl. \cite{perfComp}).  Zu den bekanntesten Performancetests Werkzeuge
gehören WebLoad, LoadNinja , LoadView und Apache JMeter. Apache JMeter
ist das Werkzeug, das zum Testen von jExam verwendet wird.

\subsection{Apache JMeter}


Nach der offiziellen Dokumentation ist Apache JMeter eine reine
Open-Source-Software, eine 100 \% reine Java-Anwendung, die von Stefano
Mazzocchi von der Apache Software Foundation entwickelt wurde, um das
Funktionsverhalten zu testen und die Leistung zu messen (vgl \cite{jmeter}). JMeter kann
verwendet werden, um die Leistung von Webanwendungen oder einer Vielzahl
von Diensten zu analysieren und zu messen. Diese Software wurde ursprünglich zum
Testen von Webanwendungen oder FTP-Anwendungen verwendet. Heutzutage wird
es für funktionale Tests, Datenbankserver-Tests. Apache JMeter kann
verwendet werden, um die Leistung statischer und dynamischer Ressourcen
sowie dynamischer Webanwendungen zu testen. Es kann verwendet werden, um
eine starke Belastung eines Servers, einer Gruppe von Servern, eines
Netzwerks oder eines Objekts zu simulieren, um dessen Stärke zu testen
oder um die Gesamtleistung unter verschiedenen Lasttypen zu analysieren.
Die wichtigsten Vorteile von JMeter sind :

\begin{enumerate}

    \item \textbf{Open-Source-Lizenz}: JMeter ist völlig kostenlos und erlaubt
     Entwicklern die Verwendung des Quellcodes für die Entwicklung.

    \item \textbf{Freundliche GUI}: JMeter ist extrem einfach zu bedienen und es
    dauert nicht lange, sich damit vertraut zu machen.

    \item \textbf{Plattformunabhängig}: JMeter ist eine 100\% reine
    Java-Desktop-Anwendung. Daher kann es auf mehreren Plattformen laufen.

    \item \textbf{Vollständiges Multithreading-Framework}: JMeter ermöglicht die
    gleichzeitige und simultane Abtastung verschiedener Funktionen durch
    eine separate Thread-Gruppe.

    \item \textbf{Visualisierung des Testergebnisses}: Das Testergebnis kann in
    verschiedenen Formaten wie Diagramm, Tabelle, Baum und Protokolldatei
    angezeigt werden.

    \item \textbf{Skalierbarkeit}: JMeter  unterstützt auch Visualisierungs-Plugins,
    mit denen Tests erweitert werden können.

    \item \textbf{Mehrere Teststrategien}: JMeter unterstützt viele
    Teststrategien wie Lasttests, verteilte Tests und funktionale Tests.

    \item \textbf{Simulation}: JMeter kann mehrere Benutzer mit gleichzeitigen Threads
    simulieren und eine hohe Last auf die zu testende Webanwendung erzeugen.

    \item \textbf{Unterstützung mehrerer Protokolle}: JMeter unterstützt nicht
    nur das Testen von Webanwendungen, sondern bewertet auch die
    Leistung von Datenbankservern. Alle Basisprotokolle wie HTTP, JDBC,
    LDAP, SOAP, JMS und FTP werden von JMeter unterstützt.

    \item \textbf{Aufzeichnung und Wiedergabe}: JMeter ermöglicht das Aufzeichnen
    von Benutzeraktivität im Browser und hilft dabei sie in einer
    Webanwendung zu simulieren.

    \item \textbf{Skript-Test}: JMeter kann mit Bean Shell für automatisierte Tests
    integriert werden.

\end{enumerate}

Im Allgemeinen hat JMeter ein vereinfachtes Funktionsprinzip. Es simuliert
eine Gruppe von Benutzern, die Anfragen an einen Zielserver senden, und
gibt Statistiken zurück, die die Leistung/Funktionalität des Zielservers/der
Zielanwendung in Tabellen und Diagrammen zeigen (siehe \Cref{fig:jmeter-prinzip}).

\begin{figure}[H]
    \centering
    \includegraphics[scale=0.5]{images/jmeter-princip}
    \caption{Funktionsweise von JMeter} \label{fig:jmeter-prinzip}
\end{figure}

JMeter wird normalerweise  über seine grafische Benutzeroberfläche
bedient. Es ist  jedoch möglich, Tests mit der grafischen
Benutzeroberfläche zu erstellen und sie in einem Docker-Container auf einer
Anwendung auszuführen. Dieses  Prinzip wurde zum Testen der beiden Versionen
von jExam verwendet und wird  im nächsten Abschnitt vorgestellt.







\subsection{Implementierung der Performancetests}

Die Performancetests, die für beide Versionen von jExam geschrieben wurden,
basieren im Wesentlichen auf den Kriterien in  \autoref{ch:perform}. Die Tests, die
in diesem Kapitel durchgeführt wurden, sind nur eine bescheidene Demonstration
dessen, was mit Jmeter möglich ist. Diese Tests könnten im Laufe der Zeit
weiterentwickelt und verbessert werden, wenn die Kriterien für die Bewertung
der Plattformen spezifischer werden. Sie werden daher als Grundlage für die
Entwicklung zukünftiger Tests dienen. Jmeter wird in diesem Fall zum Testen
einer Webanwendung verwendet. Deshalb wird HTTP als Kommunikationsprotokoll
verwendet. Zunächst ist es notwendig, einige Begriffe zu erklären, die im
Folgenden verwendet werden:

\textbf{Thread-Gruppe} : Sie ist eine Gruppe von Threads, die das gleiche
Szenario ausführen. Sie ist das Basiselement für jeden Jmeter-Testplan. Es
stehen mehrere Thread-Gruppen zur Verfügung, die konfiguriert werden können,
um zu simulieren, wie die Benutzer mit der Anwendung interagieren, wie die
Last aufrechterhalten wird und über welchen Zeitraum. Jeder Thread führt
den Testplan in seiner Gesamtheit und völlig unabhängig von anderen
Test-Threads aus. Mehrere Threads werden verwendet, um gleichzeitige
Verbindungen zu Ihrer Serveranwendung zu simulieren.

\textbf{Assertions} : Sie sind die Komponente eines Tests, mit der ein
Benutzer bestätigen kann, dass die von Jmeter erhaltene Antwort den
erwarteten Kriterien entspricht. Sie stellen sicher, dass der Benutzer sich
einer inkonsistenten oder unerwarteten Antwort der Zielanwendung bewusst
ist. Die Assertions von Jmeter sind ein mächtiges Werkzeug, aber ihre
Wirksamkeit hängt sowohl von den ausgewählten Assertionskriterien als auch
von der Genauigkeit der vorherigen Anfrage ab. Die Assertion validiert, dass
die Antwort der Anwendung wie erwartet empfangen wird, aber diese Antwort
beruht in der Regel auf der korrekten Formulierung einer vorherigen Anfrage.


\textbf{Resultslisteners} : Ein Listener ist eine Komponente, die die
Ergebnisse der Proben anzeigt. Die Ergebnisse können in einer Baumstruktur,
in Tabellen oder Diagrammen angezeigt oder einfach in eine Protokolldatei
geschrieben werden.


Für die Integration von Performancetests wurde eine Testsuite entwickelt,
die zwei Tests enthält, die in zwei Thread-Gruppen aufgeteilt sind. Der erste
Test (erste Thread-Gruppe) dient dazu, das Navigieren eines Benutzers auf den
Plattformen zu simulieren und so die Antwortzeiten des Servers zu überwachen.
Auf diese Weise ist es möglich, die beiden Versionen anhand verschiedener
Kriterien zu vergleichen, wie z.B. der schnellsten Antwortzeit. Diese
Thread-Gruppe besteht aus einem einzigen Benutzer und ist in vier Phasen
unterteilt:


\textbf{Login-Phase}: Zunächst führt der Test eine GET-Anfrage an die Login-Seite der
Anwendung durch. Danach führt er auf derselben Seite eine POST-Anfrage durch,
um sich zu registrieren und das Recht zu erhalten, auf gesicherte Seiten
zuzugreifen. Außerdem wurden zwei weitere Assertionen hinzugefügt: die
Duration Assersion, um herauszufinden, ob die Seite in weniger als drei
Sekunden empfangen wird, und die Response Assersion, die sicherstellt,
dass der Test einen korrekten HTTP-Response-Code (200) erhält.


\textbf{Home-Phase}: In diesem Teil führt der Test eine GET-Anfrage an die Home Page
der Anwendung aus. Wenn die Authentifizierung erfolgreich war, sollte der
Test eine HTTP-Response von 200 zurückgeben. Um dies sicherzustellen, wurden
die Assersion Duration (Für die drei Sekunden) und Reponse hinzugefügt.


\textbf{Veranstaltungsphase}: In diesem Teil führt der Test eine GET-Anfrage an die
Veranstaltungsseite aus. Wenn die Authentifizierung erfolgreich war, sollte
der Test eine HTTP-Response 200 zurückgeben. Um dies sicherzustellen, wurden
auch die Assersion Duration und Reponse hinzugefügt.

\textbf{Logout}: In dieser Phase führt der Test eine HTTP-POST-Anfrage aus, um sich
abzumelden, und wie die anderen Phasen verfügt auch diese Phase über die
Assertionen Duration und Response.



Der zweite Test ist viel einfacher. Die Thread-Gruppe besteht aus 200
Benutzern, die in Zeitintervallen von 10 Millisekunden eine GET-Anfrage
an die Home-Seite der Anwendung ausführen. Auf diese Weise können die
Antwortzeiten des Servers bei einer gleichzeitigen Überlastung durch
Anfragen beobachtet werden. Assersion Duration und Response wurden ebenfalls
hinzugefügt.


Nachdem der Tester die Skripte für beide Versionen von jExam geschrieben
und angepasst hat, kann er sie kompilieren und so .jmx-Dateien erzeugen.
Diese Dateien werden dann in Docker Container kopiert, die für die
automatische Ausführung der Tests zuständig sind. Nach der Ausführung der
Tests werden Berichte für \Gls{jexam_2009} und \Gls{jexam_new} generiert. Der Tester
kann sich diese Berichte ansehen und sie mit den Zielkriterien vergleichen.

\begin{lstlisting}[language=Dockerfile,label={lst:jmeter},caption={JMeter Ausführungsbefehl}]
    command: [ "./wait-for-it.sh", "web:8080/",
"/bin/sh" ,"-c",
"jmeter -f -n -t jExam_New.jmx -l results.csv
-e -o reports/html" ]

# jExam_New.jmx : jExam_new Generierter Jmeter-Test
# reports/html : Ordner, der den HTML-Bericht enthalten wird
\end{lstlisting}


    \chapter{Evaluation}\label{ch:evaluation}

Das Ziel dieser Arbeit war es, eine Testinfrastruktur f\"ur 
die jExam-Plattform zu erstellen. Diese Infrastruktur sollte den
Testern als Basis dienen und den Prozess der Testentwicklung
erleichtern. Dieses Kapitel konzentriert sich auf die Bewertung,
ob diese Ziele erreicht wurden, und geht auf die Einschr\"ankungen 
ein, w\"ahrend potenzielle Verbesserungen vorgeschlagen werden.

\section{Erreichte Ziele}

In einem ersten Schritt sollten die Funktionen von \gls{jexam_2009}
getestet werden. Danach sollte herausgefunden werden, ob es möglich 
ist, die Tests von \gls{jexam_2009} zum Testen von \gls{jexam_new} 
wiederzuverwenden und sogar eine Testsuite für \gls{jexam_new} 
bereitzustellen, bevor die Entwicklung abgeschlossen ist. Dies wurde
durch die Verwendung des Selenium-Tools ermöglicht, da 
\gls{jexam_2009} und \gls{jexam_new} praktisch die gleiche grafische 
Benutzeroberfläche haben. Die in \autoref{subsec:jexam-ui-tests}beschriebenen
Funktionen wurden für \gls{jexam_2009} getestet und nur die Login-Funktion
für \gls{jexam_new} (weil die anderen Funktionen noch nicht entwickelt
wurden). Der gleiche Login-Test funktioniert also für beide
Plattformen. Das Ziel für UI-Tests wurde erreicht.

Zweitens wurde die Frage gestellt, inwieweit es möglich ist,
Sicherheitstests zu integrieren. Diese Frage wurde in
\autoref{sec:entwicklung-von-sicherheitstests} behandelt und es
zeigte sich, dass es tatsächlich möglich ist,
Sicherheitstests zu integrieren. Zu diesem Zweck wurden
Schwachstellen-Scanner eingebaut, die beide Versionen von jExam
auf ausnutzbare Sicherheitslücken scannen.


Drittens ging es um die Frage, ob Bottleneck-Performance in beiden
Versionen von jExam festgestellt werden konnte. Dies war ein schwer
zu lösendes Problem, da sich \gls{jexam_new} noch in der Entwicklung
befindet. Das jMeter-Tool wurde jedoch in die Testinfrastruktur
integriert und mit Skripten versehen, um beide Versionen zu testen.
Bei der Analyse der von diesen Skripten erstellten Berichte wurden
kürzere Antwortzeiten für \gls{jexam_new} festgestellt, was auf eine
mögliche Verbesserung hindeuten könnte. Die verwendeten jMeter-Skripte
sind jedoch recht einfach und eignen sich nicht zum genauen Testen
von Anwendungen. Sie sollten daher verbessert werden.


Die Testinfrastruktur wurde so konzipiert, dass sie auf allen
Systemen tragbar, leicht installierbar, vollautomatisch und für
die Durchführung von Tests konfigurierbar ist. Daher kann man
sagen, dass die Testinfrastruktur die in
\autoref{sec:anforderungen-an-die-jexam-testsuite} beschriebenen
Anforderungen erfüllt.





\section{Begrenzung und mögliche Lösungen}
    \chapter{Zusammenfassung und Ausblick}\label{ch:conclusion}


\input{sections/figures}
\input{sections/install}

\section{Was ist ABC?}

\blindtext

\blindtext


    \printbibliography[heading=bibintoc]\label{sec:bibliography}%

    \appendix
    \chapter{Appendix}

\section{Präsentation der grafischen Oberfläche der Berichtstools}

\begin{figure}[H]
    \centering
    \includegraphics[scale=0.4]{images/extentReport2}
    \caption{Von Extent erstellter Bericht nach der Durchführung von UI-Tests} \label{fig:extent-report}
\end{figure}


\begin{figure}[H]
    \centering
    \includegraphics[scale=0.6]{images/jmeter-ui}
    \caption{Grafische Benutzeroberfläche von Jmeter} \label{fig:jmeter}
\end{figure}

\begin{figure}[H]
    \centering
    \includegraphics[scale=0.5]{images/jmeter-report}
    \caption{Vom Jmeter-Skript erstellter Bericht nach der Durchführung der Tests} \label{fig:jmeter-report}
\end{figure}


\section{UML-Klassendiagramm und globale Struktur für einige Pakete der UI-Tests}

\begin{figure}[H]
    \centering
    \includegraphics[scale=0.5]{images/system-test-uml}
    \caption{UML-Diagramm des Pakets systemTests} \label{fig:system-package}
\end{figure}

\begin{figure}[H]
    \centering
    \includegraphics[scale=0.4]{images/ui-test}
    \caption{Genereller Struktur der UI-Tests} \label{fig:ui-test}
\end{figure}

\begin{figure}[H]
    \centering
    \includegraphics[scale=0.4]{images/pages_diagram}
    \caption{UML-Diagramm des Pakets pages} \label{fig:page-package}
\end{figure}

\section{Globale Struktur der Testinfrastruktur}

\begin{figure}[H]
    \centering
    \includegraphics[scale=0.5]{images/global-structur}
    \caption{Globale Struktur der Testinfrastruktur} \label{fig:global-structur}
\end{figure}


\section{Einige Beispiele für Code Sources, die zum Verständnis beitragen}

\begin{lstlisting}[label={lst:docker-compose}, caption={Docker-Compose-Datei des Launcher-Dienstes}]

# Docker Compose File 
version: "3"

# docker network create jexam_network
# important command !
# docker-compose up
# docker-compose -f  docker-compose-penetration.yaml up
# docker-compose -f  docker-compose-performance.yaml up
# docker-compose -f  docker-compose-guitests.yaml up
# Must connect on TU Dresden vpn

services:

  jboss:
    build:
      context: ./
      dockerfile: ./Dockerfile-jboss
    container_name: jboss
    ports:
      - "1090:1090"
      - "1091:1091"
      - "1099:1099"
      - "40003:40003"
      - "4446:4446"
      - "4457:4457"
      - "4712:4712"
      - "4713:4713"
      - "4714:4714"
      - "9009:9009"

  oldweb:
    build:
      context: ./
      dockerfile: Dockerfile-oldweb
    container_name: oldweb
    stdin_open: true
    tty: true
    depends_on:
      - jboss
    ports:
      - "8085:8080"


  webservice:
    build:
      context: ./
      dockerfile: ./Dockerfile-webservice
    container_name: webservice
    stdin_open: true
    tty: true
    depends_on:
      - jboss
    ports:
      - "8081:8081"

  web:
    build:
      context: ./
      dockerfile: Dockerfile-web
    container_name: web
    stdin_open: true
    tty: true
    depends_on:
      - webservice
    ports:
      - "8080:8080"

  initializer:
    build:
      context: ./
      dockerfile: Dockerfile-initializer
    container_name: initializer
    stdin_open: true
    tty: true
    depends_on:
      - jboss
    volumes:
      - ./jexam-test/initializer-csv:/initializer/data/

networks:
  default:
    external:
      name: jexam_network
    
\end{lstlisting}


\begin{lstlisting}[label={lst:docker-file-web}, caption={Dockerfile Datei des jExam_New Container}]
# Dockerfile-Web (For jExam_New)

FROM openjdk:17-alpine
RUN apk --update add bash && apk --no-cache add dos2unix

COPY ./new_jexam/jexam-web /webapp

WORKDIR /webapp

# JARS kopieren

COPY ./new_jexam/jexam-webservice/. /webapp
COPY ./jboss-6.1.0.Final/server/jExamV5/lib/bos.jar /webapp
COPY ./jboss-6.1.0.Final/server/jExamV5/lib/common.jar /webapp
COPY ./jboss-6.1.0.Final/server/jExamV5/lib/csapis.jar /webapp

RUN rm -rf target
RUN rm -rf de.jexam.webservice

# CHANGE BASE_URL
RUN sed -i "s/localhost/webservice/g"  src/main/java/de/jexam/web/data/WebserviceConnector.java

RUN dos2unix mvnw

# RUN INSTALL JARS
RUN ./mvnw install:install-file -Dfile=bos.jar -DgroupId=de.jexam -DartifactId=bos -Dversion=1.0 -Dpackaging=jar
RUN ./mvnw install:install-file -Dfile=common.jar -DgroupId=de.jexam -DartifactId=common -Dversion=1.0 -Dpackaging=jar
RUN ./mvnw install:install-file -Dfile=csapis.jar -DgroupId=de.jexam -DartifactId=csapis -Dversion=1.0 -Dpackaging=jar

# INSTALL WEB CLASSES
RUN ./mvnw install:install-file -Dfile=/webapp/de.jexam.web.classes/target/classes-0.0.1-SNAPSHOT.jar -DgroupId=de.jexam -DartifactId=web-classes -Dversion=1.0 -Dpackaging=jar


RUN ./mvnw clean compile

ENTRYPOINT sleep 40;./mvnw spring-boot:run
\end{lstlisting}





    \confirmation

\end{document}
